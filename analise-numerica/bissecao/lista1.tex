\documentclass{article}

\usepackage[inline]{enumitem}

\newcommand{\ds}{\displaystyle}

\begin{document}

\noindent{}\rule{\textwidth}{0.4pt}
\begin{center}
	Universidade Federal da Grande Dourados\\
	An\'alise Num\'erica --- Lista 1 \\
	Engenharia Mec\^anica --- 2016.2 \\
	Prof.\ Adriano Barbosa
\end{center}
\noindent{}\rule{\textwidth}{0.4pt}

\begin{enumerate}
	\item Calcule o erro absoluto e relativo das aproxima\c{c}\~oes de $p$ por $p^*$:
		\begin{enumerate}
			\item $p = \pi,\ p^*  = 22/7$
			\item $p = \pi,\ p^* = 3.1416$
			\item $p = e,\ p^* = 2.718$
			\item $p = \sqrt{2},\ p^* = 1.414$
			\item $p = 8!,\ p^* = 39900$
		\end{enumerate}
	
	\item Encontre o maior intervalo ao qual $p^*$ deve pertencer para
		aproximar $p$ com erro relativo de pelo menos $10^{-3}$.
		\begin{enumerate}
			\item $p = \pi$
			\item $p = e$
			\item $p = 150$
			\item $p = 1500$
		\end{enumerate}
	
	\item O n\'umero $e$ pode ser definido por $\ds\sum_{n=0}^\infty
		\frac{1}{n!}$.  Calcule o erro absoluto e relativo da aproxima\c{c}\~ao
		$\ds\sum_{n=0}^5 \frac{1}{n!}$.

	\item Um sistema linear da forma
		\[\left\{\begin{array}{lcl}
			ax + by & = & e \\
			cx + dy & = & f
		\end{array}\right.\]
		pode ser resolvido utilizando o seguinte algoritmo:
		\[\begin{array}{ll}
		calcule & m = \ds\frac{c}{a},\ se\ a \neq{} 0; \\
				& d_1 = d - mb; \\
				& f_1 = f - me; \\
				& y = \ds\frac{f_1}{d_1}; \\
				& x = \ds\frac{e - by}{a}.
		\end{array}\]
		Resolva o sistema abaixo utilizando aritim\'etica computacional com
		quatro d\'{\i}gitos e arredondamento.

		% \begin{enumerate}
			% \item
			\[\left\{\begin{array}{lcl}
					1.130x - 6.990y & = & 14.20 \\
					1.013x - 6.099y & = & 14.22
				\end{array}\right.\]
			% \item $\left\{\begin{array}{rcl}
			% 		8.110x + 12.20y & = & -0.1370 \\
			% 		-18.11x + 112.2y & = & -0.1376
			% 	\end{array}\right.$
		% \end{enumerate}

	\item Use o m\'etodo da Bisse\c{c}\~ao e encontre $p_3$ para $f (x) = \sqrt{x} -
		\cos{x}$ em $[0,1]$.

	\item Use o m\'etodo da Bisse\c{c}\~ao para encontrar a solu\c{c}\~ao da equa\c{c}\~ao $x^3 -
		7x^2 + 14x - 6 = 0$ com precis\~ao de $10^{-2}$ em cada intervalo:
		\begin{enumerate}
			\item $[0,1]$
			\item $[1, 3.2]$
			\item $[3.2, 4]$
		\end{enumerate}
	
	\item Esboce o gr\'afico de $y = x$ e $y = 2\sin{x}$. Use o m\'etodo da
		Bisse\c{c}\~ao para encontrar uma a proxima\c{c}\~ao com precis\~ao de $10^{-2}$ da
		primeira raiz positiva de $x = 2\sin{x}$.
	
	% \item Esboce o gr\'afico de $y = e^x - 2$ e $y = \cos{(e^x - 2)}$. Use o m\'etodo
	% 	da Bisse\c{c}\~ao para encontrar uma a proxima\c{c}\~ao com precis\~ao de $10^{-2}$
	% 	em $[0.5, 1.5]$ de uma raiz de $e^x - 2 = \cos{(e^x - 2)}$.

	\item Encontre uma aproxima\c{c}\~ao de $\sqrt[3]{25}$ com precis\~ao de $10^{-2}$
		usando o m\'etodo da Bisse\c{c}\~ao.
\end{enumerate}

Respostas:

\noindent{}1. (a) $0.001264$, $4.025\times{}10^{-4}$\\
(b) $7.346\times{}10^{-6}$, $2.338\times{}10^{-6}$ \\
(c) $2.818\times{}10^{-4}$, $1.037\times{}10^{-4}$ \\
(d) $2.136\times{}10^{-4}$, $1.1510\times{}10^{-4}$ \\
(e) $420$, $1.042\times{}10^{-2}$

\noindent{}2. (a) $[3.138451061, 3.144734246]$ \\
(b) $[1.412799349, 1.415627776]$ \\
(c) $[149.85, 150.15]$ \\
(d) $[1498.5, 1501.5]$

\noindent{}3. aproxima\c{c}\~ao: $2.7166667$ \\
erro absoluto: $0.0016152$ \\
erro relativo: $5.9418\times{}10^{-4}$

\noindent{}4. $x = 67.42$, $y = 8.869$

\noindent{}5. $p_3 = 0.625$

\noindent{}6. (a) $p_7 = 0.5859$ (b) $p_8 = 3.002$ (c) $p_7 = 3.419$

\noindent{}7. Usando $[1.5, 2]$, $p_6 = 1.8984375$

\noindent{}8. Usando $[2, 3]$, $p_7 = 2.9921875$
\end{document}  % chktex 16
