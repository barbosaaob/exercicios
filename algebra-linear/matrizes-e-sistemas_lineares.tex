\documentclass{article}

\usepackage[inline]{enumitem}

\begin{document}
\noindent{}\rule{\textwidth}{0.4pt}
\begin{center}
	\'{A}lgebra Linear\\
	Lista 1 --- Matrizes e sistemas lineares \\
	\vspace{0.2cm}
	Prof. Adriano Barbosa
\end{center}
\noindent{}\rule{\textwidth}{0.4pt}

\begin{enumerate}
%%%%%%%%%%%%%%%%%%%%%%%%%%%%%%%%%%%%%%%%%%%%%
\item Sejam

$A = \left[
\begin{array}{ccc}
	1 & 2 & 3 \\
	2 & 1 & -1
\end{array}\right]$,
$B = \left[
\begin{array}{ccc}
	-2 & 0 & 1 \\
	3 & 0 & 1
\end{array}\right]$,
$C = \left[
\begin{array}{c}
	-1 \\ 2 \\ 4
\end{array}\right]$ e
$D = \left[
\begin{array}{cc}
	2 & 1
\end{array}\right]$

Encontre:

\begin{enumerate*}[label=\alph*.]
	\item $A + B$ \hfill{}
	\item $AC$ \hfill{}
	\item $BC$ \hfill{}
	\item $CD$ \hfill{}
	\item $DA$ \hfill{}
	\item $DB$ \hfill{}
	\item $-A$ \hfill{}
	\item $-D$
\end{enumerate*}

%%%%%%%%%%%%%%%%%%%%%%%%%%%%%%%%%%%%%%%%%%%%%
\item Seja
$A = \left[
\begin{array}{cc}
	2 & x^2 \\
	2x-1 & 0
\end{array}\right]$.
Se $A^T = A$, qual o valor de $x$?

%%%%%%%%%%%%%%%%%%%%%%%%%%%%%%%%%%%%%%%%%%%%%
\item Verdadeiro ou falso?

\begin{enumerate}[label=\alph*.]
	\item ${(-A)}^T = -{(A)}^T$
	\item ${(A + B)}^T = B^T + A^T$
	\item Se $AB = 0$, ent\~ao $A = 0$ ou $B = 0$.
	\item ${(k_1A)}{(k_2B)} = {(k_1k_2)}AB$
	\item ${(-A)}{(-B)} = -{(AB)}$
	\item Se $A$ e $B$ s\~ao matrizes sim\'etricas, ent\~ao $AB = BA$.
	\item Se $AB = 0$, ent\~ao $BA = 0$.
	\item Se podemos efetuar o produto $AA$, ent\~ao $A$ \'e uma matriz quadrada.
\end{enumerate}

%%%%%%%%%%%%%%%%%%%%%%%%%%%%%%%%%%%%%%%%%%%%%
\item Ache $x$, $y$, $z$ e $w$ se
$\left[\begin{array}{cc}
	x & y \\
	z & w
\end{array}\right]
\left[\begin{array}{cc}
	2 & 3 \\
	3 & 4
\end{array}\right] = 
\left[\begin{array}{cc}
	1 & 0 \\
	0 & 1
\end{array}\right]$.

%%%%%%%%%%%%%%%%%%%%%%%%%%%%%%%%%%%%%%%%%%%%%
\item Suponha que $A \neq{} 0$ e $AB = AC$ onde $A$, $B$ e $C$ s\~ao
matrizes tais que a multiplica\c{c}\~ao seja poss\'ivel.

\begin{enumerate}[label=\alph*.]
	\item $B = C$?
	\item Se existir uma matriz $Y$, tal que $YA = I$, onde $I$ \'e a matriz
		identidade, ent\~ao $B = C$?
\end{enumerate}

%%%%%%%%%%%%%%%%%%%%%%%%%%%%%%%%%%%%%%%%%%%%%
\item Resolva o sistema de equa\c{c}\~oes, escrevendo as matrizes aumentadas,
	associadas aos novos sistemas

$\left\{
\begin{array}{ccccccc}
	2x & - & y & + & 3z & = & 11 \\
	4x & - & 3y & + & 2z & = & 0 \\
	x & + & y & + & z & = & 6 \\
	3x & + & y & + & z & = & 4
\end{array}
\right.$

%%%%%%%%%%%%%%%%%%%%%%%%%%%%%%%%%%%%%%%%%%%%%
\item Reduza as matrizes \`a forma escalonada

\begin{enumerate*}[label=\alph*.]
	\item
		$\left[\begin{array}{cccc}
			1 & -2 & 3 & -1 \\
			2 & -1 & 2 & 3 \\
			3 & 1 & 2 & 3
		\end{array}\right]$
	\item
		$\left[\begin{array}{ccc}
			0 & 2 & 2 \\
			1 & 1 & 3 \\
			3 & -4 & 2 \\
			2 & -3 & 1
		\end{array}\right]$
	\item
		$\left[\begin{array}{cccc}
			0 & 1 & 3 & -2 \\
			2 & 1 & -4 & 3 \\
			2 & 3 & 2 & -1
		\end{array}\right]$
\end{enumerate*}

%%%%%%%%%%%%%%%%%%%%%%%%%%%%%%%%%%%%%%%%%%%%%
\item Calcule o posto das matrizes da quest\~ao anterior.

%%%%%%%%%%%%%%%%%%%%%%%%%%%%%%%%%%%%%%%%%%%%%
\item Determine $k$, para que o sistema admita solu\c{c}\~ao

$\left\{\begin{array}{ccccc}
	-4x & + & 3y & = & 2 \\
	5x & - & 4y & = & 0 \\
	2x & - & y & = & k
\end{array}\right.$

%%%%%%%%%%%%%%%%%%%%%%%%%%%%%%%%%%%%%%%%%%%%%
\item Determinar os valores de $a$ e $b$ de modo que o sistema

$\left\{\begin{array}{ccccl}
	3x & - & 7y & = & a \\
	x & + & y & = & b \\
	5x & + & 3y & = & 5a + 2b \\
	x & + & 2y & = & a + b -1
\end{array}\right.$

possua uma \'unica solu\c{c}\~ao. Em seguida resolver o sistema.

%%%%%%%%%%%%%%%%%%%%%%%%%%%%%%%%%%%%%%%%%%%%%
\item Foram estudados tr\^es tipos de alimentos. Fixada a mesma quantidade (1g)
	determinou-se que:

\begin{enumerate}[label=\roman*.]
	\item O alimento I tem 1 unidade de vitamina A, 3 unidades de vitamina B e
		4 unidades de vitamina C.
	\item O alimento II tem 2, 3 e 5 unidades respectivamente, das vitaminas A,
		B e C.
	\item O alimento III tem 3 unidades de vitaminas A e C, e n\~ao cont\'em
		vitamina B.
\end{enumerate}

Se s\~ao necess\'arias 11 unidades de vitamina A, 9 de vitamina B e 20 de vitamina C,

\begin{enumerate}[label=\alph*.]
	\item Encontre todas as poss\'iveis quantidades dos alimentos I, II e III,
		que fornecem a quantidade de vitaminas desejada.
	\item Se o alimento I custa R\$~0,60 por grama e os outros dois custam
		R\$~0,10, existe uma solu\c{c}\~ao custando exatamente R\$~1,00?
\end{enumerate}


%%%%%%%%%%%%%%%%%%%%%%%%%%%%%%%%%%%%%%%%%%%%%
\item Sendo
$A = \left[\begin{array}{cc}
	1 & 0 \\
	0 & 2
\end{array}\right]$ e 
$B = \left[\begin{array}{cc}
	4 & 0 \\
	0 & 2
\end{array}\right]$,
determine as matrizes $X_{2\times{} 2}$ e $Y_{2\times{} 2}$ tais que 

$\left\{\begin{array}{ccccl}
	2X & - & Y & = & A + B \\
	X & + & Y & = & A - B
\end{array}\right.$

\end{enumerate}
\end{document}
