\documentclass[a4paper,5pt]{amsbook}
%%%%%%%%%%%%%%%%%%%%%%%%%%%%%%%%%%%%%%%%%%%%%%%%%%%%%%%%%%%%%%%%%%%%%

\usepackage{booktabs}
\usepackage{graphicx}
\usepackage{multicol}
\usepackage{textcomp}
\usepackage{systeme}
\usepackage{amssymb}
\usepackage[]{amsmath}
\usepackage{subcaption}
\usepackage[inline]{enumitem}

%%%%%%%%%%%%%%%%%%%%%%%%%%%%%%%%%%%%%%%%%%%%%%%%%%%%%%%%%%%%%%

\newcommand{\sen}{\,\mbox{sen}\,}
\newcommand{\tg}{\,\mbox{tg}\,}
\newcommand{\cosec}{\,\mbox{cosec}\,}
\newcommand{\cotg}{\,\mbox{cotg}\,}
\newcommand{\tr}{\,\mbox{tr}\,}
\newcommand{\ds}{\displaystyle}

%%%%%%%%%%%%%%%%%%%%%%%%%%%%%%%%%%%%%%%%%%%%%%%%%%%%%%%%%%%%%%%%%%%%%%%%

\setlength{\textwidth}{16cm} %\setlength{\topmargin}{-1.3cm}
\setlength{\textheight}{30cm}
\setlength{\leftmargin}{1.2cm} \setlength{\rightmargin}{1.2cm}
\setlength{\oddsidemargin}{0cm}\setlength{\evensidemargin}{0cm}

%%%%%%%%%%%%%%%%%%%%%%%%%%%%%%%%%%%%%%%%%%%%%%%%%%%%%%%%%%%%%%%%%%%%%%%%

% \renewcommand{\baselinestretch}{1.6}
% \renewcommand{\thefootnote}{\fnsymbol{footnote}}
% \renewcommand{\theequation}{\thesection.\arabic{equation}}
% \setlength{\voffset}{-50pt}
% \numberwithin{equation}{chapter}

%%%%%%%%%%%%%%%%%%%%%%%%%%%%%%%%%%%%%%%%%%%%%%%%%%%%%%%%%%%%%%%%%%%%%%%

\begin{document}
\thispagestyle{empty}
\pagestyle{empty}
\begin{minipage}[h]{0.14\textwidth}
	\includegraphics[scale=0.24]{../../ufgd.png}
\end{minipage}
\begin{minipage}[h]{\textwidth}
\begin{tabular}{c}
{{\bf UNIVERSIDADE FEDERAL DA GRANDE DOURADOS}}\\
{{\bf \'{A}lgebra Linear e Geometria Anal\'{\i}tica --- Lista 13}}\\
{{\bf Prof.\ Adriano Barbosa}}\\
\end{tabular}
\vspace{-0.45cm}
%
\end{minipage}

%------------------------

\vspace{1cm}
%%%%%%%%%%%%%%%%%%%%%%%%%%%%%%%%   formulario  inicio  %%%%%%%%%%%%%%%%%%%%%%%%%%%%%%%%
\begin{enumerate}
	\vspace{0.5cm}
	\item Determine se as matrizes abaixo s\~ao diagonaliz\'aveis.

		\begin{enumerate*}
			\item $\begin{bmatrix}
					2 & -3 \\
					1 & -1
				\end{bmatrix}$
			\hspace{0.2cm}
			\hspace{0.2cm}
			\item $\begin{bmatrix}
					3 & 0 & 0 \\
					0 & 2 & 0 \\
					0 & 1 & 2
				\end{bmatrix}$
			\hspace{0.2cm}
			\hspace{0.2cm}
			\item $\begin{bmatrix}
					-1 & 0 & 1 \\
					-1 & 3 & 0 \\
					-4 & 13 & -13
				\end{bmatrix}$
			\hspace{0.2cm}
			\hspace{0.2cm}
			\item $\begin{bmatrix}
					2 & -1 & 0 & 1 \\
					0 & 2 & 1 & -1 \\
					0 & 0 & 3 & 2 \\
					0 & 0 & 0 & 3
				\end{bmatrix}$
		\end{enumerate*}

	\vspace{0.5cm}
	\item Encontre, se poss\'{\i}vel, uma matriz $P$ que diagonaliza $A$ e calcule $P^{-1}AP$.

		\begin{enumerate*}
			\item $\begin{bmatrix}
					-1 & 4 & -2 \\
					-3 & 4 & 0 \\
					-3 & 1 & 3
				\end{bmatrix}$
			\item $\begin{bmatrix}
					0 & 0 & 0 \\
					0 & 0 & 0 \\
					3 & 0 & 1
				\end{bmatrix}$
			\item $\begin{bmatrix}
					-2 & 0 & 0 & 0 \\
					0 & -2 & 5 & -5 \\
					0 & 0 & 3 & 0 \\
					0 & 0 & 0 & 3
				\end{bmatrix}$
		\end{enumerate*}

	\vspace{0.5cm}
	\item Calcule $A^{10}$, onde
		\[\begin{bmatrix}
			-1 & 7 & -1 \\
			0 & 1 & 0 \\
			0 & 15 & -2
		\end{bmatrix}\]

	\vspace{0.5cm}
	\item Mostre que se $(a-d)^2 + 4bc > 0$, ent\~ao
		$A = \begin{bmatrix}
			a & b \\
			c & d
		\end{bmatrix}$
		\'e diagonaliz\'avel.
\end{enumerate}

\end{document}
