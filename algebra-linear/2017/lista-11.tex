\documentclass[a4paper,5pt]{amsbook}
%%%%%%%%%%%%%%%%%%%%%%%%%%%%%%%%%%%%%%%%%%%%%%%%%%%%%%%%%%%%%%%%%%%%%

\usepackage{booktabs}
\usepackage{graphicx}
\usepackage{multicol}
\usepackage{textcomp}
\usepackage{systeme}
\usepackage{amssymb}
\usepackage[]{amsmath}
\usepackage{subcaption}
\usepackage[inline]{enumitem}
\usepackage{gensymb}

%%%%%%%%%%%%%%%%%%%%%%%%%%%%%%%%%%%%%%%%%%%%%%%%%%%%%%%%%%%%%%

\newcommand{\sen}{\,\mbox{sen}\,}
\newcommand{\tg}{\,\mbox{tg}\,}
\newcommand{\cosec}{\,\mbox{cosec}\,}
\newcommand{\cotg}{\,\mbox{cotg}\,}
\newcommand{\tr}{\,\mbox{tr}\,}
\newcommand{\ds}{\displaystyle}

%%%%%%%%%%%%%%%%%%%%%%%%%%%%%%%%%%%%%%%%%%%%%%%%%%%%%%%%%%%%%%%%%%%%%%%%

\setlength{\textwidth}{16cm} \setlength{\topmargin}{-1.3cm}
\setlength{\textheight}{30cm}
\setlength{\leftmargin}{1.2cm} \setlength{\rightmargin}{1.2cm}
\setlength{\oddsidemargin}{0cm}\setlength{\evensidemargin}{0cm}

%%%%%%%%%%%%%%%%%%%%%%%%%%%%%%%%%%%%%%%%%%%%%%%%%%%%%%%%%%%%%%%%%%%%%%%%

% \renewcommand{\baselinestretch}{1.6}
% \renewcommand{\thefootnote}{\fnsymbol{footnote}}
% \renewcommand{\theequation}{\thesection.\arabic{equation}}
% \setlength{\voffset}{-50pt}
% \numberwithin{equation}{chapter}

%%%%%%%%%%%%%%%%%%%%%%%%%%%%%%%%%%%%%%%%%%%%%%%%%%%%%%%%%%%%%%%%%%%%%%%

\begin{document}
\thispagestyle{empty}
\pagestyle{empty}
\begin{minipage}[h]{0.14\textwidth}
	\includegraphics[scale=0.24]{../../ufgd.png}
\end{minipage}
\begin{minipage}[h]{\textwidth}
\begin{tabular}{c}
{{\bf UNIVERSIDADE FEDERAL DA GRANDE DOURADOS}}\\
{{\bf \'{A}lgebra Linear e Geometria Anal\'{\i}tica --- Lista 11}}\\
{{\bf Prof.\ Adriano Barbosa}}\\
\end{tabular}
\vspace{-0.45cm}
%
\end{minipage}

%------------------------

\vspace{1cm}
%%%%%%%%%%%%%%%%%%%%%%%%%%%%%%%%   formulario  inicio  %%%%%%%%%%%%%%%%%%%%%%%%%%%%%%%%
\begin{enumerate}
	\vspace{0.5cm}
	\item Encontre a matriz can\^onica para cada composi\c{c}\~ao abaixo:
		\begin{enumerate}
			\item Uma rota\c{c}\~ao de $90\degree$ seguida de uma reflex\~ao em torno
				do eixo $y$.
			\item Uma reflex\~ao em torno do eixo $x$ seguida de uma escala de
				raz\~ao $k=3$.
			\item Uma rota\c{c}\~ao de $60\degree$, seguida de uma proje\c{c}\~ao ortogonal
				sobre o eixo $x$, seguida de uma reflex\~ao em torno do eixo $y$.
			\item Uma rota\c{c}\~ao de $15\degree$, seguida de uma rota\c{c}\~ao de
				$105\degree$, seguida de uma rota\c{c}\~ao de $60\degree$.
		\end{enumerate}

	\vspace{0.5cm}
	\item Determine se $T_1\circ T_2 = T_2 \circ T_1$:
		\begin{enumerate}
			\item $T_1:\mathbb{R}^2 \rightarrow \mathbb{R}^2$ \'e a proje\c{c}\~ao
				ortogonal sobre o eixo $x$ e $T_2: \mathbb{R}^2 \rightarrow
				\mathbb{R}^2$ \'e a proje\c{c}\~ao ortogonal sobre o eixo $y$.
			\item $T_1:\mathbb{R}^2 \rightarrow \mathbb{R}^2$ \'e a rota\c{c}\~ao por
				um \^angulo $\theta_1$ e $T_2: \mathbb{R}^2 \rightarrow
				\mathbb{R}^2$ \'e a rota\c{c}\~ao por um \^angulo $\theta_2$.
			\item $T_1:\mathbb{R}^2 \rightarrow \mathbb{R}^2$ \'e a rota\c{c}\~ao por
				um \^angulo $\theta$ e $T_2: \mathbb{R}^2 \rightarrow
				\mathbb{R}^2$ \'e a proje\c{c}\~ao ortogonal sobre o eixo $x$.
		\end{enumerate}

	\vspace{0.5cm}
	\item Definimos as proje\c{c}\~oes ortogonais de $\mathbb{R}^3$ sobre os eixos
		$x$, $y$ e $x$, respectivamente, por
		\[T_x(x,y,z) = (x,0,0),\ T_y(x,y,z) = (0,y,0)\ e\ T_z(x,y,z) =
			(0,0,z).\]
	Mostre que as proje\c{c}\~oes acima s\~ao transforma\c{c}\~oes lineares.

	\vspace{0.5cm}
	\item Mostre que a reflex\~ao de vetores de $\mathbb{R}^2$ em torno da
		reta $y=x$ \'e uma transforma\c{c}\~ao linear e encontre sua matriz can\^onica.

	\vspace{0.5cm}
	\item Mostre que a proje\c{c}\~ao ortogonal de vetores de $\mathbb{R}^2$ sobre a
		reta $y=x$ \'e uma transforma\c{c}\~ao linear e encontre sua matriz can\^onica.

	\vspace{0.5cm}
	\item Mostre que os vetores $T(v)$ e $v - T(v)$ s\~ao ortogonais, onde
		$T:\mathbb{R}^2 \rightarrow \mathbb{R}^2$ \'e uma proje\c{c}\~ao ortogonal
		sobre os eixos coordenados ou sobre a reta $y=x$.

	\vspace{0.5cm}
	\item Calcule o n\'ucleo e a imagem das transforma\c{c}\~oes lineares abaixo:
		\begin{enumerate}
			\item $T:\mathbb{R}^2 \rightarrow \mathbb{R}^2$,
				$T(x,y) = (2x-3y, 3x)$
			\item $T:\mathbb{R}^2 \rightarrow \mathbb{R}^4$,
				$T(x,y) = (x-y, x, y, y - x)$
			\item $T:\mathbb{R}^3 \rightarrow \mathbb{R}$,
				$T(x,y,z) = x - y + z$
		\end{enumerate}

	\vspace{0.5cm}
	\item Determine se as transforma\c{c}\~oes lineares do exerc\'{\i}cio anterior s\~ao
		injetivas e se s\~ao sobrejetivas.

	\vspace{0.5cm}
	\item O operador linear $T:\mathbb{R}^2 \rightarrow \mathbb{R}^2$, $T(x,y)
		= (2x+y, 3x+4y)$ \'e invert\'{\i}vel? Encontre sua inversa se poss\'{\i}vel.

	\vspace{0.5cm}
	\item Determine se os conjuntos de vetores abaixo s\~ao LI ou LD
		\begin{enumerate}
			\item $\{(1,2), (-2, 1)\}$ em $\mathbb{R}^2$
			\item $\{(1,1,1), (1,1,0), (1,0,0)\}$ em $\mathbb{R}^3$
			\item $\{(1,2,3), (1,1,1)\}$ em $\mathbb{R}^3$
			\item $\{(1,0), (1,1), (1,2)\}$ em $\mathbb{R}^2$
		\end{enumerate}

	\vspace{0.5cm}
	\item Para quais conjuntos de vetores do exerc\'{\i}cio anterior \'e poss\'{\i}vel
		escrever qualquer vetor dos espa\c{c}os vetoriais dados como combina\c{c}\~ao
		linear de seus elementos?

	\vspace{0.5cm}
	\item Determine quais dos conjuntos de vetores do exerc\'{\i}cio $(10)$ s\~ao
		base dos espa\c{c}os vetoriais dados.
\end{enumerate}

\end{document}
