\documentclass[a4paper,5pt]{amsbook}
%%%%%%%%%%%%%%%%%%%%%%%%%%%%%%%%%%%%%%%%%%%%%%%%%%%%%%%%%%%%%%%%%%%%%

\usepackage{booktabs}
\usepackage{graphicx}
\usepackage{multicol}
\usepackage{textcomp}
\usepackage{systeme}
\usepackage{amssymb}
\usepackage[]{amsmath}
\usepackage{subcaption}
\usepackage[inline]{enumitem}

%%%%%%%%%%%%%%%%%%%%%%%%%%%%%%%%%%%%%%%%%%%%%%%%%%%%%%%%%%%%%%

\newcommand{\sen}{\,\mbox{sen}\,}
\newcommand{\tg}{\,\mbox{tg}\,}
\newcommand{\cosec}{\,\mbox{cosec}\,}
\newcommand{\cotg}{\,\mbox{cotg}\,}
\newcommand{\tr}{\,\mbox{tr}\,}
\newcommand{\ds}{\displaystyle}

%%%%%%%%%%%%%%%%%%%%%%%%%%%%%%%%%%%%%%%%%%%%%%%%%%%%%%%%%%%%%%%%%%%%%%%%

\setlength{\textwidth}{16cm} \setlength{\topmargin}{-1.3cm}
\setlength{\textheight}{30cm}
\setlength{\leftmargin}{1.2cm} \setlength{\rightmargin}{1.2cm}
\setlength{\oddsidemargin}{0cm}\setlength{\evensidemargin}{0cm}

%%%%%%%%%%%%%%%%%%%%%%%%%%%%%%%%%%%%%%%%%%%%%%%%%%%%%%%%%%%%%%%%%%%%%%%%

% \renewcommand{\baselinestretch}{1.6}
% \renewcommand{\thefootnote}{\fnsymbol{footnote}}
% \renewcommand{\theequation}{\thesection.\arabic{equation}}
% \setlength{\voffset}{-50pt}
% \numberwithin{equation}{chapter}

%%%%%%%%%%%%%%%%%%%%%%%%%%%%%%%%%%%%%%%%%%%%%%%%%%%%%%%%%%%%%%%%%%%%%%%

\begin{document}
\thispagestyle{empty}
\pagestyle{empty}
\begin{minipage}[h]{0.14\textwidth}
	\includegraphics[scale=0.24]{../../ufgd.png}
\end{minipage}
\begin{minipage}[h]{\textwidth}
\begin{tabular}{c}
{{\bf UNIVERSIDADE FEDERAL DA GRANDE DOURADOS}}\\
{{\bf \'{A}lgebra Linear e Geometria Anal\'{\i}tica --- Lista 8}}\\
{{\bf Prof.\ Adriano Barbosa}}\\
\end{tabular}
\vspace{-0.45cm}
%
\end{minipage}

%------------------------

%%%%%%%%%%%%%%%%%%%%%%%%%%%%%%%%   formulario  inicio  %%%%%%%%%%%%%%%%%%%%%%%%%%%%%%%%
\begin{enumerate}
	\vspace{0.5cm}
	\item Dado o plano $\pi: 3x + y -z = 4$, calcule:
	\begin{enumerate}
		\item O ponto de $\pi$ que tem coordenadas $x=1$ e $y=3$.
		\item O ponto de $\pi$ que tem coordenadas $x=0$ e $z=2$.
		\item O valor de $k$ para que o ponto $P = (k, 2, k-1)$ perten\c{c}a a
			$\pi$.
		\item O ponto de coordenada $x=2$ cuja coordenada $y$ \'e o dobro da
			coordenada $z$.
		\item O valor de $k$ para que o plano $\pi_1:kx-4y+4z = 7$ seja
			paralelo a $\pi$.
	\end{enumerate}

	\vspace{0.5cm}
	\item Dada a equa\c{c}\~ao impl\'{\i}cita do plano $\pi: 3x - 2y - z = 6$, encontre as
		equa\c{c}\~oes param\'etricas de $\pi$.

	\vspace{0.5cm}
	\item Encontre a equa\c{c}\~ao impl\'{\i}cita do plano 
	$\left\{\begin{array}{l}
		x = 1 + h - 2t \\
		y = 1 - t \\
		z = 4 + 2h - 2t
	\end{array}\right.$
	
	\vspace{0.5cm}
	\item Encontre a equa\c{c}\~ao impl\'{\i}cita do plano que cont\'em as retas
	\begin{enumerate}
		\vspace{0.3cm}
		\item
			$r_1:\left\{\begin{array}{l}
				y = 2x - 3 \\
				z = -x + 2
			\end{array}\right.$
		\ \ \ \ \ e\ \ \ \ \ 
			$r_2:\left\{\begin{array}{l}
				\frac{x-1}{3} = z-1 \\
				y = -1
			\end{array}\right.$
		\vspace{0.3cm}
		\item
			$r_1:\left\{\begin{array}{l}
				x = 1 + 2t \\
				y = -2 + 3t \\
				z = 3 -t
			\end{array}\right.$
		\ \ \ \ \ e\ \ \ \ \ 
			$r_2:\left\{\begin{array}{l}
				x = 1 - 2t \\
				y = -2 - t \\
				z = 3 + 2t
			\end{array}\right.$
	\end{enumerate}
	
	\vspace{0.5cm}
	\item Determine a equa\c{c}\~ao impl\'{\i}cita do plano que cont\'em
	\begin{enumerate}
		\vspace{0.3cm}
		\item o ponto $A = (4, 3, 2)$ e a reta
			$r:\left\{\begin{array}{l}
				x = t \\
				y = 2 -t \\
				z = 3 + 2t
			\end{array}\right.$
		\vspace{0.3cm}
		\item o ponto $A = (1, -1, 2)$ e o eixo $z$
	\end{enumerate}
	
	\vspace{0.5cm}
	\item Verifique se a reta $r$ est\'a contida no plano $\pi$
	\begin{enumerate}
		\vspace{0.3cm}
		\item 
			$r:\left\{\begin{array}{l}
				y = 4x + 1 \\
				z = 2x - 1
			\end{array}\right.$
		\ \ \ \ \ e\ \ \ \ \ 
			$\pi:2x + y - 3z - 4 = 0$
		\vspace{0.3cm}
		\item
			$r: x-2 = \frac{y+2}{2} = z+3$
		\ \ \ \ \ e\ \ \ \ \ 
			$\pi:\left\{\begin{array}{l}
				x = h + t \\
				y = -1 + 2h - 3t \\
				z = -3 + h -t
			\end{array}\right.$
	\end{enumerate}
	
	\vspace{0.5cm}
	\item Encontre a equa\c{c}\~ao param\'etrica do plano paralelo ao eixo $z$ e
		que intersecta o eixo $x$ em $-3$ e o eixo $y$ em $4$.
	
	\vspace{0.5cm}
	\item Encontre a equa\c{c}\~ao param\'etrica do plano paralelo ao plano $xz$ e que
		intersecta o eixo dos $y$ em $-7$.

	\vspace{0.5cm}
	\item Encontre as equa\c{c}\~oes param\'etricas da reta que \'e a interse\c{c}\~ao entre os
		planos $3x+2y-4z=6$ e $x-3y-2z=4$.

	\vspace{0.5cm}
	\item Dados os planos $\pi_1: x+y+z=6$, $\pi_2: y=3-x$ e
		$\pi_3: \left\{\begin{array}{l}
			x = 2+2t-2s \\
			y = 1-t-s \\
			z = 2-2t+2s
		\end{array}\right.$
	\begin{enumerate}
		\item Calcule a interse\c{c}\~ao entre $\pi_1$ e $\pi_2$.
		\item Calcule a interse\c{c}\~ao entre $\pi_1$, $\pi_2$ e $\pi_3$.
	\end{enumerate}
\end{enumerate}

\end{document}
