\documentclass[a4paper,5pt]{amsbook}
%%%%%%%%%%%%%%%%%%%%%%%%%%%%%%%%%%%%%%%%%%%%%%%%%%%%%%%%%%%%%%%%%%%%%

\usepackage{booktabs}
\usepackage{graphicx}
\usepackage{multicol}
\usepackage{textcomp}
\usepackage{systeme}
\usepackage{amssymb}
\usepackage[]{amsmath}
\usepackage{subcaption}
\usepackage[inline]{enumitem}

%%%%%%%%%%%%%%%%%%%%%%%%%%%%%%%%%%%%%%%%%%%%%%%%%%%%%%%%%%%%%%

\newcommand{\sen}{\,\mbox{sen}\,}
\newcommand{\tg}{\,\mbox{tg}\,}
\newcommand{\cosec}{\,\mbox{cosec}\,}
\newcommand{\cotg}{\,\mbox{cotg}\,}
\newcommand{\tr}{\,\mbox{tr}\,}
\newcommand{\ds}{\displaystyle}

%%%%%%%%%%%%%%%%%%%%%%%%%%%%%%%%%%%%%%%%%%%%%%%%%%%%%%%%%%%%%%%%%%%%%%%%

\setlength{\textwidth}{16cm} \setlength{\topmargin}{-1cm}
\setlength{\textheight}{25cm}
\setlength{\leftmargin}{1.2cm} \setlength{\rightmargin}{1.2cm}
\setlength{\oddsidemargin}{0cm}\setlength{\evensidemargin}{0cm}

%%%%%%%%%%%%%%%%%%%%%%%%%%%%%%%%%%%%%%%%%%%%%%%%%%%%%%%%%%%%%%%%%%%%%%%%

% \renewcommand{\baselinestretch}{1.6}
% \renewcommand{\thefootnote}{\fnsymbol{footnote}}
% \renewcommand{\theequation}{\thesection.\arabic{equation}}
% \setlength{\voffset}{-50pt}
% \numberwithin{equation}{chapter}

%%%%%%%%%%%%%%%%%%%%%%%%%%%%%%%%%%%%%%%%%%%%%%%%%%%%%%%%%%%%%%%%%%%%%%%

\begin{document}
\thispagestyle{empty}
\pagestyle{empty}
\begin{minipage}[h]{0.14\textwidth}
	\includegraphics[scale=0.24]{../../ufgd.png}
\end{minipage}
\begin{minipage}[h]{\textwidth}
\begin{tabular}{c}
{{\bf UNIVERSIDADE FEDERAL DA GRANDE DOURADOS}}\\
{{\bf \'{A}lgebra Linear e Geometria Anal\'{\i}tica --- Lista 4}}\\
{{\bf Prof.\ Adriano Barbosa}}\\
\end{tabular}
\vspace{-0.45cm}
%
\end{minipage}

%------------------------
\vspace{1cm}

%%%%%%%%%%%%%%%%%%%%%%%%%%%%%%%%   formulario  inicio  %%%%%%%%%%%%%%%%%%%%%%%%%%%%%%%%
\begin{enumerate}
	\vspace{0.5cm}
	\item Encontre $A$ que satisfaz as igualdades abaixo:

		\begin{enumerate*}
			\item $A^{-1} =
				\begin{bmatrix}
					2 & -1 \\
					3 & 5
				\end{bmatrix}$
			\hspace{1cm}
			\hspace{1cm}
			\item $ {(I+2A)}^{-1} =
				\begin{bmatrix}
					-1 & 2 \\
					4 & 5
				\end{bmatrix}$
			\hspace{1cm}
			\hspace{1cm}
			\item $A^{-1} =
				\begin{bmatrix}
					\cos{\theta} & \sin{\theta} \\
					-\sin{\theta} & \cos{\theta}
				\end{bmatrix}$
		\end{enumerate*}

	\vspace{0.5cm}
	\item Resolva o sistema
		\[\systeme[xyz]{
				x + 2y + z = b_1,
				x - y + z = b_2,
				x + y = b_3
			}\]
		invertendo a matriz dos coeficientes para

		\begin{enumerate*}
			\item $b_1=-1, b_2=3, b_3 = 4$
			\hspace{1cm}
			\hspace{1cm}
			\item $b_1=5, b_2=0, b_3=0$
		\end{enumerate*}

	\vspace{0.5cm}
	\item Sejam
		\[A = \begin{bmatrix}
				2 & 1 & 2 \\
				2 & 2 & -2 \\
				3 & 1 & 1
			\end{bmatrix}
		\hspace{1cm}
		x = \begin{bmatrix}
			x_1 \\ x_2 \\ x_3
			\end{bmatrix}\]
			\begin{enumerate}
				\item Mostre que a equa\c{c}\~ao $Ax=x$ pode ser reescrita como
					$(A-I)x = 0$ e use este resultado para resolver $Ax=x$ em
					$x$.
				\item Resolva $Ax=4x$.
			\end{enumerate}

	\vspace{0.5cm}
	\item Apenas observando as matrizes aumentadas abaixo, determine se o
		sistema correspondente tem solu\c{c}\~ao e se a solu\c{c}\~ao \'e \'unica. Justifique
		sua resposta.

		\begin{enumerate*}
			\item $\begin{bmatrix}
					1 & 2 & 3 & 4 \\
					0 & 5 & 6 & 7 \\
					0 & 0 & 8 & 9
				\end{bmatrix}$
			\hspace{0.5cm}
			\hspace{0.5cm}
			\item $\begin{bmatrix}
					1 & 0 & 0 & 1 \\
					0 & 1 & 0 & 1 \\
					0 & 0 & 0 & 1
				\end{bmatrix}$
			\hspace{0.5cm}
			\hspace{0.5cm}
			\item $\begin{bmatrix}
					1 & 0 & 0 & 0 & 1 \\
					0 & 2 & 3 & -1 & 2 \\
					0 & 0 & 3 & 9 & 3 \\
					0 & 0 & 0 & 0 & 0 \\
				\end{bmatrix}$
		\end{enumerate*}

	\vspace{0.5cm}
	\item Dada $A =
		\begin{bmatrix}
			1 & 0 \\
			0 & -2
		\end{bmatrix}$.
		Encontre $A^2$, $A^{-2}$ e $A^{-k}$ por inspe\c{c}\~ao.

	\vspace{0.5cm}
	\item Encontre todos os valores de $a$, $b$ e $c$ tais que $A$ \'e sim\'etrica
		\[A = \begin{bmatrix}
				2 & a-2b+2c & 2a+b+c \\
				3 & 5 & a+c \\
				0 & -2 & 7
			\end{bmatrix}\]

	\vspace{0.5cm}
	\item Se a matriz $A_{n\times{}n}$ pode ser expressa como $A=LU$, onde $L$
		\'e uma matriz triangular inferior e $U$ \'e uma matriz triangular
		superior, ent\~ao o sistema $Ax=b$ pode ser expresso como $LUx=b$ e
		resolvido em dois passos:
		\begin{enumerate}
			\item Chame $y = Ux$ e resolva o sistema $Ly=b$ para $y$.
			\item Sabendo o valor de $y$, resolva $Ux=y$ para $x$.
		\end{enumerate}
		Verifique que $A=LU$ e use este m\'etodo para resolver o sistema $Ax=b$, onde:
		\[A = \begin{bmatrix}
				2 & -1 & 3 \\
				-4 & 5 & 0 \\
				4 & 2 & 18
			\end{bmatrix},\ 
		L = \begin{bmatrix}
				1 & 0 & 0 \\
				-2 & 3 & 0 \\
				2 & 4 & 1
			\end{bmatrix},\ 
		U = \begin{bmatrix}
				2 & -1 & 3 \\
				0 & 1 & 2 \\
				0 & 0 & 4
			\end{bmatrix},\ 
		b = \begin{bmatrix}
			1 \\ -2 \\ 0
		\end{bmatrix}\]
\end{enumerate}

\end{document}
