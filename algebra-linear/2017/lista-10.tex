\documentclass[a4paper,5pt]{amsbook}
%%%%%%%%%%%%%%%%%%%%%%%%%%%%%%%%%%%%%%%%%%%%%%%%%%%%%%%%%%%%%%%%%%%%%

\usepackage{booktabs}
\usepackage{graphicx}
\usepackage{multicol}
\usepackage{textcomp}
\usepackage{systeme}
\usepackage{amssymb}
\usepackage[]{amsmath}
\usepackage{subcaption}
\usepackage[inline]{enumitem}

%%%%%%%%%%%%%%%%%%%%%%%%%%%%%%%%%%%%%%%%%%%%%%%%%%%%%%%%%%%%%%

\newcommand{\sen}{\,\mbox{sen}\,}
\newcommand{\tg}{\,\mbox{tg}\,}
\newcommand{\cosec}{\,\mbox{cosec}\,}
\newcommand{\cotg}{\,\mbox{cotg}\,}
\newcommand{\tr}{\,\mbox{tr}\,}
\newcommand{\ds}{\displaystyle}

%%%%%%%%%%%%%%%%%%%%%%%%%%%%%%%%%%%%%%%%%%%%%%%%%%%%%%%%%%%%%%%%%%%%%%%%

\setlength{\textwidth}{16cm} %\setlength{\topmargin}{-1.3cm}
\setlength{\textheight}{30cm}
\setlength{\leftmargin}{1.2cm} \setlength{\rightmargin}{1.2cm}
\setlength{\oddsidemargin}{0cm}\setlength{\evensidemargin}{0cm}

%%%%%%%%%%%%%%%%%%%%%%%%%%%%%%%%%%%%%%%%%%%%%%%%%%%%%%%%%%%%%%%%%%%%%%%%

% \renewcommand{\baselinestretch}{1.6}
% \renewcommand{\thefootnote}{\fnsymbol{footnote}}
% \renewcommand{\theequation}{\thesection.\arabic{equation}}
% \setlength{\voffset}{-50pt}
% \numberwithin{equation}{chapter}

%%%%%%%%%%%%%%%%%%%%%%%%%%%%%%%%%%%%%%%%%%%%%%%%%%%%%%%%%%%%%%%%%%%%%%%

\begin{document}
\thispagestyle{empty}
\pagestyle{empty}
\begin{minipage}[h]{0.14\textwidth}
	\includegraphics[scale=0.24]{../../ufgd.png}
\end{minipage}
\begin{minipage}[h]{\textwidth}
\begin{tabular}{c}
{{\bf UNIVERSIDADE FEDERAL DA GRANDE DOURADOS}}\\
{{\bf \'{A}lgebra Linear e Geometria Anal\'{\i}tica --- Lista 10}}\\
{{\bf Prof.\ Adriano Barbosa}}\\
\end{tabular}
\vspace{-0.45cm}
%
\end{minipage}

%------------------------

\vspace{1cm}
%%%%%%%%%%%%%%%%%%%%%%%%%%%%%%%%   formulario  inicio  %%%%%%%%%%%%%%%%%%%%%%%%%%%%%%%%
\begin{enumerate}
	\vspace{0.5cm}
	\item Encontre o dom\'{\i}nio e o contra-dom\'{\i}nio das transforma\c{c}\~oes abaixo e
		determine se s\~ao lineares:
		\begin{enumerate}
			\item $T(x,y,z) = (3x-2y+4z, 5x-8y+z)$
			\item $T(x,y) = (2xy-y, x+3xy, x+y)$
			\item $T(x,y,z,t) = (x^2-3y+z-2t, 3x-4y-z^2+t)$
		\end{enumerate}

	\vspace{0.5cm}
	\item Encontre a matriz can\^onica das transforma\c{c}\~oes lineares abaixo
		\begin{enumerate}
			\item $T: \mathbb{R}^4 \rightarrow \mathbb{R}2$, $T(x,y,z,t) =
				2x-3y+t, 3x+5y-t)$
			\item $T: \mathbb{R}^4 \rightarrow \mathbb{R}^4$, $T(x,y,z,t) = (x,
				x+y, x+y+z, x+y+z+t)$
			\item $T: \mathbb{R}^2 \rightarrow \mathbb{R}^3$, $T(x,y) = (-x+y,
				3x-2y, 5x-7y)$
		\end{enumerate}

	\vspace{0.5cm}
	\item Encontre a transforma\c{c}\~ao linear cuja matriz can\^onica \'e dada abaixo:

		\begin{enumerate*}
			\item $\begin{bmatrix}
					1 & 2 \\
					3 & 4
				\end{bmatrix}$
			\hspace{0.5cm}
			\hspace{0.5cm}
			\item $\begin{bmatrix}
					-1 & 2 & 0 \\
					3 & 1 & 5
				\end{bmatrix}$
			\hspace{0.5cm}
			\hspace{0.5cm}
			\item $\begin{bmatrix}
					-1 & 1 \\
					2 & 4 \\
					7 & 8
				\end{bmatrix}$
		\end{enumerate*}

	\vspace{0.5cm}
	\item Use a matriz can\^onica $[T]$ para obter $T(v)$ e em seguida confira
		o resultado calculando $T(v)$ diretamente:
		\begin{enumerate}
			\item $T: \mathbb{R}^2 \rightarrow \mathbb{R}^3$, $T(x,y) = (-x+y,
				y, x-y)$ avaliada em $(1,2)$
			\item $T: \mathbb{R}^3 \rightarrow \mathbb{R}^3$, $T(x,y,z) =
				(-x+2y, y-3z, x-y-z)$ avaliada em $(2,3,0)$
		\end{enumerate}
\end{enumerate}

\end{document}
