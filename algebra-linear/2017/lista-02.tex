\documentclass[a4paper,5pt]{amsbook}
%%%%%%%%%%%%%%%%%%%%%%%%%%%%%%%%%%%%%%%%%%%%%%%%%%%%%%%%%%%%%%%%%%%%%

\usepackage{booktabs}
\usepackage{graphicx}
\usepackage{multicol}
\usepackage{textcomp}
\usepackage{systeme}
\usepackage{amssymb}
\usepackage[]{amsmath}
\usepackage{subcaption}
\usepackage[inline]{enumitem}

%%%%%%%%%%%%%%%%%%%%%%%%%%%%%%%%%%%%%%%%%%%%%%%%%%%%%%%%%%%%%%

\newcommand{\sen}{\,\mbox{sen}\,}
\newcommand{\tg}{\,\mbox{tg}\,}
\newcommand{\cosec}{\,\mbox{cosec}\,}
\newcommand{\cotg}{\,\mbox{cotg}\,}
\newcommand{\ds}{\displaystyle}

%%%%%%%%%%%%%%%%%%%%%%%%%%%%%%%%%%%%%%%%%%%%%%%%%%%%%%%%%%%%%%%%%%%%%%%%

\setlength{\textwidth}{16cm} % \setlength{\topmargin}{-2cm}
% \setlength{\textheight}{25cm}
\setlength{\leftmargin}{1.2cm} \setlength{\rightmargin}{1.2cm}
\setlength{\oddsidemargin}{0cm}\setlength{\evensidemargin}{0cm}

%%%%%%%%%%%%%%%%%%%%%%%%%%%%%%%%%%%%%%%%%%%%%%%%%%%%%%%%%%%%%%%%%%%%%%%%

% \renewcommand{\baselinestretch}{1.6}
% \renewcommand{\thefootnote}{\fnsymbol{footnote}}
% \renewcommand{\theequation}{\thesection.\arabic{equation}}
% \setlength{\voffset}{-50pt}
% \numberwithin{equation}{chapter}

%%%%%%%%%%%%%%%%%%%%%%%%%%%%%%%%%%%%%%%%%%%%%%%%%%%%%%%%%%%%%%%%%%%%%%%

\begin{document}
\thispagestyle{empty}
\begin{minipage}[h]{0.14\textwidth}
	\includegraphics[scale=0.24]{../../ufgd.png}
\end{minipage}
\begin{minipage}[h]{\textwidth}
\begin{tabular}{c}
{{\bf UNIVERSIDADE FEDERAL DA GRANDE DOURADOS}}\\
{{\bf \'{A}lgebra Linear e Geometria Anal\'{\i}tica --- Lista 2}}\\
{{\bf Prof.\ Adriano Barbosa}}\\
\end{tabular}
\vspace{-0.45cm}
%
\end{minipage}

%------------------------
\vspace{1cm}

%%%%%%%%%%%%%%%%%%%%%%%%%%%%%%%%   formulario  inicio  %%%%%%%%%%%%%%%%%%%%%%%%%%%%%%%%
\begin{enumerate}
	\vspace{0.5cm}
	\item Encontre a matriz aumentada de cada um dos sistemas lineares abaixo:

		\vspace{0.3cm}
		\begin{enumerate*}
			\item
				$\systeme[x_1x_2x_3]{
					x_1 + 2x_2 - 4x_4 + x_5 = 1,
					3x_2 + x_3 - x_5 = 2,
					x_3 + 7x_4 = 1
				}$
			\hspace{1cm}
			\hspace{1cm}
			\item
				$\systeme[x_1x_2x_3]{
					2x_1 + 2x_3 = 1,
					3x_1 - x_2 + 4x_3 = 7,
					6x_1 + x_2 - x_3 = 0
				}$
		\end{enumerate*}

	\vspace{0.5cm}
	\item Encontre o sistema de equa\c{c}\~oes lineares correspondendo \`as matrizes
		aumentadas abaixo:

		\vspace{0.3cm}
		\begin{enumerate*}
			\item
				$\begin{bmatrix}
					2 & 0 & 0 \\
					3 & -4 & 0 \\
					0 & 1 & 1
				\end{bmatrix}$
			\hspace{1cm}
			\hspace{1cm}
			\item
				$\begin{bmatrix}
					7 & 2 & 1 & -3 & 5 \\
					1 & 2 & 4 & 0 & 1
				\end{bmatrix}$
		\end{enumerate*}

	\vspace{0.5cm}
	\item Determine se as matrizes abaixo est\~ao na forma escalonada. Resolva o sistema.

		\vspace{0.3cm}
		\begin{enumerate*}
			\item
				$\begin{bmatrix}
					1 & 0 & 0 & -7 & 8 \\
					0 & 1 & 0 & 3 & 2 \\
					0 & 0 & 1 & 1 & -5
				\end{bmatrix}$
			\hspace{1cm}
			\hspace{1cm}
			\item 
				$\begin{bmatrix}
					1 & -3 & 0 & 0 \\
					0 & 0 & 1 & 0 \\
					0 & 0 & 0 & 1
				\end{bmatrix}$
		\end{enumerate*}

	\vspace{0.5cm}
	\item Resolva o sistema abaixo:
		\[\systeme[abc]{
				-2b + 3c = 1,
				3a + 6b - 3c = -2,
				6a + 6b + 3c = 5
			}\]

% 	\vspace{0.5cm}
% 	\item Um sistema linear \'e dito \textbf{homog\^eneo} se os coeficientes
% 		independentes s\~ao todos zero, ou seja, a \'ultima coluna da matriz
% 		aumentada do sistema \'e nula. \'{E} poss\'{\i}vel que um sistema linear
% 		homog\^eneo n\~ao possua solu\c{c}\~ao?

	\vspace{0.5cm}
	\item Para quais valores de $a$ o sistema abaixo n\~ao admite solu\c{c}\~ao?
		\[\systeme[xyz]{
				x + 2y - 3z = 4,
				3x - y + 5z = 2,
				4x + y + {(a^2-14)}z = a+2
			}\]
\end{enumerate}

\end{document}
