\documentclass[a4paper,5pt]{amsbook}
%%%%%%%%%%%%%%%%%%%%%%%%%%%%%%%%%%%%%%%%%%%%%%%%%%%%%%%%%%%%%%%%%%%%%

\usepackage{booktabs}
\usepackage{graphicx}
\usepackage{multicol}
\usepackage{textcomp}
\usepackage{systeme}
\usepackage{amssymb}
\usepackage[]{amsmath}
\usepackage{subcaption}
\usepackage[inline]{enumitem}

%%%%%%%%%%%%%%%%%%%%%%%%%%%%%%%%%%%%%%%%%%%%%%%%%%%%%%%%%%%%%%

\newcommand{\sen}{\,\mbox{sen}\,}
\newcommand{\tg}{\,\mbox{tg}\,}
\newcommand{\cosec}{\,\mbox{cosec}\,}
\newcommand{\cotg}{\,\mbox{cotg}\,}
\newcommand{\tr}{\,\mbox{tr}\,}
\newcommand{\ds}{\displaystyle}

%%%%%%%%%%%%%%%%%%%%%%%%%%%%%%%%%%%%%%%%%%%%%%%%%%%%%%%%%%%%%%%%%%%%%%%%

\setlength{\textwidth}{16cm} %\setlength{\topmargin}{-1.3cm}
\setlength{\textheight}{30cm}
\setlength{\leftmargin}{1.2cm} \setlength{\rightmargin}{1.2cm}
\setlength{\oddsidemargin}{0cm}\setlength{\evensidemargin}{0cm}

%%%%%%%%%%%%%%%%%%%%%%%%%%%%%%%%%%%%%%%%%%%%%%%%%%%%%%%%%%%%%%%%%%%%%%%%

% \renewcommand{\baselinestretch}{1.6}
% \renewcommand{\thefootnote}{\fnsymbol{footnote}}
% \renewcommand{\theequation}{\thesection.\arabic{equation}}
% \setlength{\voffset}{-50pt}
% \numberwithin{equation}{chapter}

%%%%%%%%%%%%%%%%%%%%%%%%%%%%%%%%%%%%%%%%%%%%%%%%%%%%%%%%%%%%%%%%%%%%%%%

\begin{document}
\thispagestyle{empty}
\pagestyle{empty}
\begin{minipage}[h]{0.14\textwidth}
	\includegraphics[scale=0.24]{../../ufgd.png}
\end{minipage}
\begin{minipage}[h]{\textwidth}
\begin{tabular}{c}
{{\bf UNIVERSIDADE FEDERAL DA GRANDE DOURADOS}}\\
{{\bf \'{A}lgebra Linear e Geometria Anal\'{\i}tica --- Lista 12}}\\
{{\bf Prof.\ Adriano Barbosa}}\\
\end{tabular}
\vspace{-0.45cm}
%
\end{minipage}

%------------------------

\vspace{1cm}
%%%%%%%%%%%%%%%%%%%%%%%%%%%%%%%%   formulario  inicio  %%%%%%%%%%%%%%%%%%%%%%%%%%%%%%%%
\begin{enumerate}
	\vspace{0.5cm}
	\item Calcule as equa\c{c}\~oes caracter\'{\i}sticas das matrizes can\^onicas das
		transforma\c{c}\~oes lineares abaixo:
		\begin{enumerate}
			\item $T:\mathbb{R}^2 \rightarrow \mathbb{R}^2$, $T(x,y) = (x+2y, 2x+y)$
			\item $T:\mathbb{R}^2 \rightarrow \mathbb{R}^2$, $T(x,y) = (2x+3y, 4x+3y)$
			\item $T:\mathbb{R}^2 \rightarrow \mathbb{R}^2$, $T(x,y) = (3x+y, -5x-3y)$
			\item $T:\mathbb{R}^3 \rightarrow \mathbb{R}^3$, $T(x,y,z) = (x, y, 0)$
			\item $T:\mathbb{R}^3 \rightarrow \mathbb{R}^3$, $T(x,y,z) = (4x+z, -2x+y, -2x+z)$
			\item $T:\mathbb{R}^3 \rightarrow \mathbb{R}^3$, $T(x,y,z) = (5x+z, x+y, -7x+y)$
		\end{enumerate}

	\vspace{0.5cm}
	\item Calcule os autovalores e autovetores das transforma\c{c}\~oes do exerc\'{\i}cio
		anterior.

	\vspace{0.5cm}
	\item Calcule os auto-espa\c{c}os associados aos autovalores das transforma\c{c}\~oes
		lineares no exerc\'{\i}cio (1).
\end{enumerate}

\end{document}
