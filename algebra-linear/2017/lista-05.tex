\documentclass[a4paper,5pt]{amsbook}
%%%%%%%%%%%%%%%%%%%%%%%%%%%%%%%%%%%%%%%%%%%%%%%%%%%%%%%%%%%%%%%%%%%%%

\usepackage{booktabs}
\usepackage{graphicx}
\usepackage{multicol}
\usepackage{textcomp}
\usepackage{systeme}
\usepackage{amssymb}
\usepackage[]{amsmath}
\usepackage{subcaption}
\usepackage[inline]{enumitem}

%%%%%%%%%%%%%%%%%%%%%%%%%%%%%%%%%%%%%%%%%%%%%%%%%%%%%%%%%%%%%%

\newcommand{\sen}{\,\mbox{sen}\,}
\newcommand{\tg}{\,\mbox{tg}\,}
\newcommand{\cosec}{\,\mbox{cosec}\,}
\newcommand{\cotg}{\,\mbox{cotg}\,}
\newcommand{\tr}{\,\mbox{tr}\,}
\newcommand{\ds}{\displaystyle}

%%%%%%%%%%%%%%%%%%%%%%%%%%%%%%%%%%%%%%%%%%%%%%%%%%%%%%%%%%%%%%%%%%%%%%%%

\setlength{\textwidth}{16cm} \setlength{\topmargin}{-2cm}
\setlength{\textheight}{30cm}
\setlength{\leftmargin}{1.2cm} \setlength{\rightmargin}{1.2cm}
\setlength{\oddsidemargin}{0cm}\setlength{\evensidemargin}{0cm}

%%%%%%%%%%%%%%%%%%%%%%%%%%%%%%%%%%%%%%%%%%%%%%%%%%%%%%%%%%%%%%%%%%%%%%%%

% \renewcommand{\baselinestretch}{1.6}
% \renewcommand{\thefootnote}{\fnsymbol{footnote}}
% \renewcommand{\theequation}{\thesection.\arabic{equation}}
% \setlength{\voffset}{-50pt}
% \numberwithin{equation}{chapter}

%%%%%%%%%%%%%%%%%%%%%%%%%%%%%%%%%%%%%%%%%%%%%%%%%%%%%%%%%%%%%%%%%%%%%%%

\begin{document}
\thispagestyle{empty}
\pagestyle{empty}
\begin{minipage}[h]{0.14\textwidth}
	\includegraphics[scale=0.24]{../../ufgd.png}
\end{minipage}
\begin{minipage}[h]{\textwidth}
\begin{tabular}{c}
{{\bf UNIVERSIDADE FEDERAL DA GRANDE DOURADOS}}\\
{{\bf \'{A}lgebra Linear e Geometria Anal\'{\i}tica --- Lista 5}}\\
{{\bf Prof.\ Adriano Barbosa}}\\
\end{tabular}
\vspace{-0.45cm}
%
\end{minipage}

%------------------------

%%%%%%%%%%%%%%%%%%%%%%%%%%%%%%%%   formulario  inicio  %%%%%%%%%%%%%%%%%%%%%%%%%%%%%%%%
\begin{enumerate}
	\vspace{0.5cm}
	\item Classifique as permuta\c{c}\~oes de $\{1, 2, 3, 4, 5\}$ abaixo em par ou
		\'{\i}mpar:

		\begin{enumerate*}
			\item $(4\ 1\ 3\ 5\ 2)$
			\hspace{0.5cm}
			\hspace{0.5cm}
			\item $(5\ 3\ 4\ 2\ 1)$
			\hspace{0.5cm}
			\hspace{0.5cm}
			\item $(1\ 4\ 2\ 3\ 5)$
		\end{enumerate*}

	\vspace{0.5cm}
	\item Encontre os valores de $\lambda$ para os quais $\det(A)=0$

		\begin{enumerate*}
			\item
				$\begin{bmatrix}
					\lambda-2 & 1 \\
					-5 & \lambda+4
				\end{bmatrix}$
			\hspace{0.5cm}
			\hspace{0.5cm}
			\item
				$\begin{bmatrix}
					\lambda-4 & 0 & 0 \\
					0 & \lambda & 2 \\
					0 & 3 & \lambda-1
				\end{bmatrix}$
		\end{enumerate*}

	\vspace{0.5cm}
	\item Mostre que o valor do determinante
		\[\begin{vmatrix}
			\sen(\theta) & \cos(\theta) & 0 \\
			-\cos(\theta) & \sen(\theta) & 0 \\
			\sen(\theta)-\cos(\theta) & \sen(\theta)+\cos(\theta) & 1
		\end{vmatrix}\]

	\vspace{0.5cm}
	\item Calcule o determinante das matrizes abaixo reduzindo \`a forma
		escalonada por linhas.

		\begin{enumerate*}
			\item
				$\begin{bmatrix}
					3 & 6 & -9 \\
					0 & 0 & -2 \\
					-2 & 1 & 5
				\end{bmatrix}$
			\hspace{0.5cm}
			\hspace{0.5cm}
			\item
				$\begin{bmatrix}
					0 & 3 & 1 \\
					1 & 1 & 2 \\
					3 & 2 & 4
				\end{bmatrix}$
		\end{enumerate*}

	\vspace{0.5cm}
	\item Sabendo que $\begin{vmatrix}
			a & b & c \\
			d & e & f \\
			g & h & i
		\end{vmatrix}=-6$,
		encontre

		\begin{enumerate*}
			\item
				$\begin{vmatrix}
					d & e & f \\
					g & h & i \\
					a & b & c
				\end{vmatrix}$
			\hspace{0.5cm}
			\hspace{0.5cm}
			\item
				$\begin{vmatrix}
					a+g & b+h & c+i \\
					d & e & f \\
					g & h & i
				\end{vmatrix}$
		\end{enumerate*}

	\vspace{0.5cm}
	\item Determine quais das matrizes abaixo s\~ao invert\'{\i}veis

		\begin{enumerate*}
			\item
				$\begin{bmatrix}
					1 & 0 & -1 \\
					9 & -1 & 4 \\
					8 & 9 & -1
				\end{bmatrix}$
			\hspace{0.5cm}
			\hspace{0.5cm}
			\item
				$\begin{bmatrix}
					\sqrt{2} & -\sqrt{7} & 0 \\
					3\sqrt{2} & -3\sqrt{7} & 0 \\
					5 & -9 & 0
				\end{bmatrix}$
		\end{enumerate*}

	\vspace{0.5cm}
	\item Sem calcular diretamente, mostre que $x=0$ e $x=2$ satisfazem
		\[\begin{vmatrix}
			x^2 & x & 2 \\
			2 & 1 & 1 \\
			0 & 0 & -5
		\end{vmatrix}=0\]

	\item Seja $A = \begin{bmatrix}
				4 & -1 & 1 & 6 \\
				0 & 0 & -3 & 3 \\
				4 & 1 & 0 & 14 \\
				4 & 1 & 3 & 2
			\end{bmatrix}$,
		encontre

		\vspace{0.3cm}
		\begin{enumerate*}
			\item $M_{13}$ e $C_{13}$
			\hspace{0.5cm}
			\hspace{0.5cm}
			\item $M_{22}$ e $C_{22}$
		\end{enumerate*}

	\vspace{0.5cm}
	\item Calcule o determinante das matrizes abaixo usando o m\'etodo dos
		co-fatores.

		\begin{enumerate*}
			\item
				$\begin{bmatrix}
					-3 & 0 & 7 \\
					2 & 5 & 1 \\
					-1 & 0 & 5
				\end{bmatrix}$
			\hspace{0.5cm}
			\hspace{0.5cm}
			\item
				$\begin{bmatrix}
					1 & k & k^2 \\
					1 & k & k^2 \\
					1 & k & k^2
				\end{bmatrix}$
			\hspace{0.5cm}
			\hspace{0.5cm}
			\item
				$\begin{bmatrix}
					4 & 0 & 0 & 1 & 0 \\
					3 & 3 & 3 & -1 & 0 \\
					1 & 2 & 4 & 2 & 3 \\
					9 & 4 & 6 & 2 & 3 \\
					2 & 2 & 4 & 2 & 3
				\end{bmatrix}$
		\end{enumerate*}

	\vspace{0.5cm}
	\item Use a adjunta de $A$ para encontrar $A^{-1}$.
		\[A = \begin{bmatrix}
				2 & 0 & 3 \\
				0 & 3 & 2 \\
				-2 & 0 & -4
			\end{bmatrix}\]
\end{enumerate}

\end{document}
