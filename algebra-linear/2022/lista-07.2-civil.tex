\documentclass[a4paper,5pt]{amsbook}
%%%%%%%%%%%%%%%%%%%%%%%%%%%%%%%%%%%%%%%%%%%%%%%%%%%%%%%%%%%%%%%%%%%%%

\usepackage{booktabs}
\usepackage{graphicx}
\usepackage{multicol}
\usepackage{textcomp}
\usepackage{systeme}
\usepackage{amssymb}
\usepackage[]{amsmath}
\usepackage{subcaption}
\usepackage[inline]{enumitem}

%%%%%%%%%%%%%%%%%%%%%%%%%%%%%%%%%%%%%%%%%%%%%%%%%%%%%%%%%%%%%%

\newcommand{\sen}{\,\mbox{sen}\,}
\newcommand{\tg}{\,\mbox{tg}\,}
\newcommand{\cosec}{\,\mbox{cosec}\,}
\newcommand{\cotg}{\,\mbox{cotg}\,}
\newcommand{\tr}{\,\mbox{tr}\,}
\newcommand{\ds}{\displaystyle}

%%%%%%%%%%%%%%%%%%%%%%%%%%%%%%%%%%%%%%%%%%%%%%%%%%%%%%%%%%%%%%%%%%%%%%%%

\setlength{\textwidth}{16cm} \setlength{\topmargin}{-2.5cm}
\setlength{\textheight}{30cm}
\setlength{\leftmargin}{1.2cm} \setlength{\rightmargin}{1.2cm}
\setlength{\oddsidemargin}{0cm}\setlength{\evensidemargin}{0cm}

%%%%%%%%%%%%%%%%%%%%%%%%%%%%%%%%%%%%%%%%%%%%%%%%%%%%%%%%%%%%%%%%%%%%%%%%

% \renewcommand{\baselinestretch}{1.6}
% \renewcommand{\thefootnote}{\fnsymbol{footnote}}
% \renewcommand{\theequation}{\thesection.\arabic{equation}}
% \setlength{\voffset}{-50pt}
% \numberwithin{equation}{chapter}

%%%%%%%%%%%%%%%%%%%%%%%%%%%%%%%%%%%%%%%%%%%%%%%%%%%%%%%%%%%%%%%%%%%%%%%

\begin{document}
\thispagestyle{empty}
\pagestyle{empty}
\begin{minipage}[h]{0.14\textwidth}
	\includegraphics[scale=0.24]{../../ufgd.png}
\end{minipage}
\begin{minipage}[h]{\textwidth}
\begin{tabular}{c}
{{\bf UNIVERSIDADE FEDERAL DA GRANDE DOURADOS}}\\
{{\bf \'{A}lgebra Linear e Geometria Anal\'{\i}tica --- Lista 7.2}}\\
{{\bf Prof.\ Adriano Barbosa}}\\
\end{tabular}
\vspace{-0.45cm}
%
\end{minipage}

%------------------------

%%%%%%%%%%%%%%%%%%%%%%%%%%%%%%%%   formulario  inicio  %%%%%%%%%%%%%%%%%%%%%%%%%%%%%%%%
\begin{enumerate}
	\vspace{0.3cm}
	\item Determine a equa\c{c}\~ao param\'etrica da reta $r$ definida pelos pontos $A
		= (2, -3, 4)$ e $B = (1, -1, 2)$ e verifique se os pontos $C =
		(\frac{5}{2}, -4, 5)$ e $D = (-1, 3, 4)$ pertencem a $r$.
	
	\vspace{0.3cm}
	\item Escreva a equa\c{c}\~ao param\'etrica da reta que passa por $A = (1, 2, 3)$ e
		\'e paralela a reta $r:~(x, y, z)~=(1, 4, 3) + t(0, 0, 1)$
	
	\vspace{0.3cm}
	\item Verifique se os pontos $P_1 = (5, -5, 6)$ e $P_2 = (4, -1, 12)$
		pertencem a reta $\displaystyle r: -(x-3) = \frac{y+1}{2} =
		-\frac{z-2}{2}$
	
	\vspace{0.3cm}
	\item Determine o vetor diretor das retas abaixo:

	\begin{enumerate*}
		\item 
			$\left\{\begin{array}{l}
				y = -x \\
				z = 3 + x
			\end{array}\right.$
		\hspace{0.5cm}
		\hspace{0.5cm}
		\item 
			$\left\{\begin{array}{l}
				y = 2x \\
				z = 3
			\end{array}\right.$
		\hspace{0.5cm}
		\hspace{0.5cm}
		\item $y = 3x-7$
		\hspace{0.5cm}
		\hspace{0.5cm}
		\item $\ds\frac{y-2}{3} = x-2$
	\end{enumerate*}
	
	\vspace{0.3cm}
	\item Determine o \^angulo entre as retas
	\begin{enumerate}
		\item
			$r_1:\left\{\begin{array}{l}
				x = -2 -t \\
				y = t \\
				z = 3 - 2t
			\end{array}\right.$
		\ \ \ \ \ e\ \ \ \ \ 
			$r_2:\begin{array}{l}
				\ds\frac{x}{2} = y+6 = z-1
			\end{array}$
		\item
			$r_1:\left\{\begin{array}{l}
				x = 1 + \sqrt{2}t \\
				y = t \\
				z = 5 - 3t
			\end{array}\right.$
		\ \ \ \ \ e\ \ \ \ \ 
			$r_2:\left\{\begin{array}{l}
				x = 3 \\
				y = 2
			\end{array}\right.$
	\end{enumerate}
	
	\vspace{0.3cm}
	\item Determine o valor de $n$ para que o \^angulo entre as retas seja
		$\frac{\pi}{6}$:

	$r_1:\begin{array}{l}
		\ds\frac{x-2}{4} = \frac{y}{5} = \frac{z}{3}
	\end{array}$
	\ \ \ \ \ e\ \ \ \ \ 
	$r_2:\left\{\begin{array}{l}
		y = nx + 5 \\
		z = 2x - 2
	\end{array}\right.$
	
	\vspace{0.3cm}
	\item Dados $A = (3, 4, -2)$ e 
	$r:\left\{\begin{array}{l}
		x = 1 + t \\
		y = 2 - t \\
		z = 4 + 2t
	\end{array}\right.$. Determine a equa\c{c}\~ao param\'etrica da reta que passa por
$A$ e \'e perpendicular a $r$.
	
	\vspace{0.3cm}
	\item Encontre a reta que passa pelo ponto m\'edio do segmento de extremos $A
		= (5, -1, 4)$ e $B = (-1, -7, 1)$ e seja perpendicular a ele.
\end{enumerate}

\end{document}
