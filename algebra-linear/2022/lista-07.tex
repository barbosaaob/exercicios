\documentclass[a4paper,5pt]{amsbook}
%%%%%%%%%%%%%%%%%%%%%%%%%%%%%%%%%%%%%%%%%%%%%%%%%%%%%%%%%%%%%%%%%%%%%

\usepackage{booktabs}
\usepackage{graphicx}
\usepackage{multicol}
\usepackage{textcomp}
\usepackage{systeme}
\usepackage{amssymb}
\usepackage[]{amsmath}
\usepackage{subcaption}
\usepackage[inline]{enumitem}

%%%%%%%%%%%%%%%%%%%%%%%%%%%%%%%%%%%%%%%%%%%%%%%%%%%%%%%%%%%%%%

\newcommand{\sen}{\,\mbox{sen}\,}
\newcommand{\tg}{\,\mbox{tg}\,}
\newcommand{\cosec}{\,\mbox{cosec}\,}
\newcommand{\cotg}{\,\mbox{cotg}\,}
\newcommand{\tr}{\,\mbox{tr}\,}
\newcommand{\ds}{\displaystyle}

%%%%%%%%%%%%%%%%%%%%%%%%%%%%%%%%%%%%%%%%%%%%%%%%%%%%%%%%%%%%%%%%%%%%%%%%

\setlength{\textwidth}{16cm} %\setlength{\topmargin}{-1.3cm}
\setlength{\textheight}{30cm}
\setlength{\leftmargin}{1.2cm} \setlength{\rightmargin}{1.2cm}
\setlength{\oddsidemargin}{0cm}\setlength{\evensidemargin}{0cm}

%%%%%%%%%%%%%%%%%%%%%%%%%%%%%%%%%%%%%%%%%%%%%%%%%%%%%%%%%%%%%%%%%%%%%%%%

% \renewcommand{\baselinestretch}{1.6}
% \renewcommand{\thefootnote}{\fnsymbol{footnote}}
% \renewcommand{\theequation}{\thesection.\arabic{equation}}
% \setlength{\voffset}{-50pt}
% \numberwithin{equation}{chapter}

%%%%%%%%%%%%%%%%%%%%%%%%%%%%%%%%%%%%%%%%%%%%%%%%%%%%%%%%%%%%%%%%%%%%%%%

\begin{document}
\thispagestyle{empty}
\pagestyle{empty}
\begin{minipage}[h]{0.14\textwidth}
	\includegraphics[scale=0.24]{../../ufgd.png}
\end{minipage}
\begin{minipage}[h]{\textwidth}
\begin{tabular}{c}
{{\bf UNIVERSIDADE FEDERAL DA GRANDE DOURADOS}}\\
{{\bf \'{A}lgebra Linear e Geometria Anal\'{\i}tica --- Lista 7}}\\
{{\bf Prof.\ Adriano Barbosa}}\\
\end{tabular}
\vspace{-0.45cm}
%
\end{minipage}

%------------------------

\vspace{1cm}
%%%%%%%%%%%%%%%%%%%%%%%%%%%%%%%%   formulario  inicio  %%%%%%%%%%%%%%%%%%%%%%%%%%%%%%%%
\begin{enumerate}
	\vspace{0.5cm}
    \item Sejam $u=(-3,2,1,0)$, $v=(4,7,-3,2)$ e $w=(5,-2,8,1)$. Encontre

        \begin{enumerate*}
            \item $v-w$
			\hspace{0.5cm}
			\hspace{0.5cm}
            \item $2u+7v$
			\hspace{0.5cm}
			\hspace{0.5cm}
            \item $-u+(v-4w)$
			\hspace{0.5cm}
			\hspace{0.5cm}
            \item $(6v-w)-(4u+v)$
        \end{enumerate*}

	\vspace{0.5cm}
	\item Sejam $u = (4,1,2,3)$, $v = (0,3,8,-2)$ e $w = (3,1,2,2)$. Calcule:

		\begin{enumerate*}
			\item $\|u+v\|$
			\hspace{0.5cm}
			\hspace{0.5cm}
			\item $\left\|\ds\frac{1}{\|w\|}w\right\|$
			\hspace{0.5cm}
			\hspace{0.5cm}
			\item $\|-2u\|+2\|u\|$
			\hspace{0.5cm}
			\hspace{0.5cm}
			\item $\langle u, v \rangle$
		\end{enumerate*}

	\vspace{0.5cm}
    \item Mostre que se $v$ \'e um vetor n\~ao-nulo de $\mathbb{R}^n$, ent\~ao
        $w=\frac{1}{\|v\|}v$ tem norma 1.

	\vspace{0.5cm}
	\item Mostre que n\~ao existem escalares $a$, $b$ e $c$ tais que
		\[a(1,0,1,0)+b(1,0,-2,1)+c(2,0,1,2)=(1,-2,2,3)\]

    \vspace{0.5cm}
    \item Verifique que a f\'ormula $\langle Au,v\rangle=\langle u,A^Tv\rangle$ \'e v\'alida para 
        \begin{enumerate}
            \item $A=\left[\begin{array}{ccc}
                    2 & -1 \\
                    3 & 4
                \end{array}\right]$,
                $u=\left[\begin{array}{cc}
                    3 \\
                    1
                \end{array}\right]$,
                $v=\left[\begin{array}{cc}
                    -2 \\
                    6
                \end{array}\right]$
            \item $A=\left[\begin{array}{ccc}
                    -1 & 2 & 4 \\
                    3 & 1 & 0 \\
                    5 & -2 & 3
                \end{array}\right]$,
                $u=\left[\begin{array}{cc}
                    -1 \\
                    2 \\
                    5
                \end{array}\right]$,
                $v=\left[\begin{array}{cc}
                    0 \\
                    2 \\
                    -4
                \end{array}\right]$
        \end{enumerate}

    \vspace{0.5cm}
    \item Determine os valores de $k$ para que os vetores $u$ e $v$ sejam
        ortogonais
        \begin{enumerate}
            \item $u=(2,1,3)$, $v=(1,7,k)$
            \item $u=(k,k,1)$, $v=(k,5,6)$
        \end{enumerate}

	\vspace{0.5cm}
	\item Se $u$ e $v$ s\~ao vetores em $\mathbb{R}^n$, vale a desigualdade de
		Cauchy-Schwarz $|\langle u, v \rangle| \le \|u\| \|v\|$. Verifique que
		a desigualdade de Cauchy-Schwarz vale para os vetores abaixo:
		\begin{enumerate}
			\item $u = (3,2)$, $v = (4, -1)$
			\item $u = (-3,1,0)$, $v = (2,-1,3)$
			\item $u = (0,-2,2,1)$, $v = (-1,-1,1,1)$
		\end{enumerate}

	\vspace{0.5cm}
	\item Use a desigualdade de Cauchy-Schwarz para provar que
		\[{(a \cos\theta + b \sen\theta)}^2 \le a^2 + b^2\]

	\vspace{0.5cm}
	\item Se $u$ e $v$ s\~ao matrizes $n \times 1$ e $A$ \'e uma matriz $n \times
		n$, mostre que
		\[{\left(v^TA^TAu\right)}^2 \le \left(u^TA^TAu\right)\left(v^TA^TAv\right)\]

    \vspace{0.5cm}
    \item Sejam $a_1, a_2, \ldots, a_n\in\mathbb{R}$. Em $\mathbb{R}^2$, os vetores
        $v_1=(a_1,0)$ e $v_2=(0,a_2)$ determinam um ret\^angulo de \'area $A=a_1a_2$ e
        em $\mathbb{R}^3$, os vetores $v_1=(a_1,0,0)$, $v_2=(0,a_2,0)$ e
        $v_3=(0,0,a_3)$ determinam um paralelep\'{\i}pedo de volume $V=a_1a_2a_3$. A
        \'area $A$ e o volume $V$ s\~ao chamados, \`as vezes, de ``medida
        euclidiana'' do ret\^angulo e do paralelep\'{\i}pedo, respectivamente.
        \begin{enumerate}
            \item Como voc\^e definiria a medida euclidiana da ``caixa'' em
                $\mathbb{R}^n$ determinada pelos vetores
                $v_1=(a_1,0,0,\ldots,0)$, $v_2=(0,a_2,0,\ldots,0)$, $\ldots$,
                $v_n=(0,0,0,\ldots,a_n)$?
            \item Como voc\^e definiria o comprimento da diagonal da caixa do
                item acima?
        \end{enumerate}
\end{enumerate}

\end{document}
