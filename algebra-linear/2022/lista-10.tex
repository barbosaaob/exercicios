\documentclass[a4paper,5pt]{amsbook}
%%%%%%%%%%%%%%%%%%%%%%%%%%%%%%%%%%%%%%%%%%%%%%%%%%%%%%%%%%%%%%%%%%%%%

\usepackage{booktabs}
\usepackage{graphicx}
\usepackage{multicol}
\usepackage{textcomp}
\usepackage{systeme}
\usepackage{amssymb}
\usepackage[]{amsmath}
\usepackage{subcaption}
\usepackage[inline]{enumitem}
\usepackage{gensymb}

%%%%%%%%%%%%%%%%%%%%%%%%%%%%%%%%%%%%%%%%%%%%%%%%%%%%%%%%%%%%%%

\newcommand{\sen}{\,\mbox{sen}\,}
\newcommand{\tg}{\,\mbox{tg}\,}
\newcommand{\cosec}{\,\mbox{cosec}\,}
\newcommand{\cotg}{\,\mbox{cotg}\,}
\newcommand{\tr}{\,\mbox{tr}\,}
\newcommand{\ds}{\displaystyle}

%%%%%%%%%%%%%%%%%%%%%%%%%%%%%%%%%%%%%%%%%%%%%%%%%%%%%%%%%%%%%%%%%%%%%%%%

\setlength{\textwidth}{16cm} %\setlength{\topmargin}{-1.3cm}
\setlength{\textheight}{23cm}
\setlength{\leftmargin}{1.2cm} \setlength{\rightmargin}{1.2cm}
\setlength{\oddsidemargin}{0cm}\setlength{\evensidemargin}{0cm}

%%%%%%%%%%%%%%%%%%%%%%%%%%%%%%%%%%%%%%%%%%%%%%%%%%%%%%%%%%%%%%%%%%%%%%%%

% \renewcommand{\baselinestretch}{1.6}
% \renewcommand{\thefootnote}{\fnsymbol{footnote}}
% \renewcommand{\theequation}{\thesection.\arabic{equation}}
% \setlength{\voffset}{-50pt}
% \numberwithin{equation}{chapter}

%%%%%%%%%%%%%%%%%%%%%%%%%%%%%%%%%%%%%%%%%%%%%%%%%%%%%%%%%%%%%%%%%%%%%%%

\begin{document}
\thispagestyle{empty}
\pagestyle{empty}
\begin{minipage}[h]{0.14\textwidth}
	\includegraphics[scale=0.24]{../../ufgd.png}
\end{minipage}
\begin{minipage}[h]{\textwidth}
\begin{tabular}{c}
{{\bf UNIVERSIDADE FEDERAL DA GRANDE DOURADOS}}\\
{{\bf \'{A}lgebra Linear e Geometria Anal\'{\i}tica --- Lista 10}}\\
{{\bf Prof.\ Adriano Barbosa}}\\
\end{tabular}
\vspace{-0.45cm}
%
\end{minipage}

%------------------------

\vspace{1cm}
%%%%%%%%%%%%%%%%%%%%%%%%%%%%%%%%   formulario  inicio  %%%%%%%%%%%%%%%%%%%%%%%%%%%%%%%%
\begin{enumerate}
	\vspace{0.5cm}
    \item Seja $T:\mathbb{R}^2\rightarrow\mathbb{R}^2$ o operador linear dado por
        $T(x,y)=(2x-y,-8x+4y)$. Quais dos vetores est\~ao em $Im(T)$?
        \begin{enumerate}
            \item $(1,-4)$
            \item $(5,0)$
            \item $(-3,12)$
        \end{enumerate}

    \vspace{0.5cm}
    \item Considerando o operador do exerc\'{\i}cio acima, quais dos vetores est\~ao
        em $N(T)$?
        \begin{enumerate}
            \item $(5,10)$
            \item $(3,2)$
            \item $(1,1)$
        \end{enumerate}

	\vspace{0.5cm}
	\item Calcule o n\'ucleo e a imagem das transforma\c{c}\~oes lineares abaixo:
		\begin{enumerate}
			\item $T:\mathbb{R}^2 \rightarrow \mathbb{R}^2$,
				$T(x,y) = (2x-3y, 3x)$
			\item $T:\mathbb{R}^2 \rightarrow \mathbb{R}^4$,
				$T(x,y) = (x-y, x, y, y - x)$
			\item $T:\mathbb{R}^3 \rightarrow \mathbb{R}$,
				$T(x,y,z) = x - y + z$
		\end{enumerate}

    \vspace{0.5cm}
    \item Encontre bases para o n\'umcleo e imagem das transforma\c{c}\~oes lineares do
        exerc\'{\i}cio anterior.

	\vspace{0.5cm}
	\item Determine se as transforma\c{c}\~oes lineares do exerc\'{\i}cio anterior s\~ao
		injetivas e se s\~ao sobrejetivas.

	\vspace{0.5cm}
	\item O operador linear $T:\mathbb{R}^2 \rightarrow \mathbb{R}^2$, $T(x,y)
		= (2x+y, 3x+4y)$ \'e invert\'{\i}vel? Encontre sua inversa se poss\'{\i}vel.

\end{enumerate}

\end{document}
