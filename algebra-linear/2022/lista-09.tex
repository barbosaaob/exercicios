\documentclass[a4paper,5pt]{amsbook}
%%%%%%%%%%%%%%%%%%%%%%%%%%%%%%%%%%%%%%%%%%%%%%%%%%%%%%%%%%%%%%%%%%%%%

\usepackage{booktabs}
\usepackage{graphicx}
\usepackage{multicol}
\usepackage{textcomp}
\usepackage{systeme}
\usepackage{amssymb}
\usepackage[]{amsmath}
\usepackage{subcaption}
\usepackage[inline]{enumitem}
\usepackage{gensymb}

%%%%%%%%%%%%%%%%%%%%%%%%%%%%%%%%%%%%%%%%%%%%%%%%%%%%%%%%%%%%%%

\newcommand{\sen}{\,\mbox{sen}\,}
\newcommand{\tg}{\,\mbox{tg}\,}
\newcommand{\cosec}{\,\mbox{cosec}\,}
\newcommand{\cotg}{\,\mbox{cotg}\,}
\newcommand{\tr}{\,\mbox{tr}\,}
\newcommand{\ds}{\displaystyle}

%%%%%%%%%%%%%%%%%%%%%%%%%%%%%%%%%%%%%%%%%%%%%%%%%%%%%%%%%%%%%%%%%%%%%%%%

\setlength{\textwidth}{16cm} \setlength{\topmargin}{-1cm}
\setlength{\textheight}{25cm}
\setlength{\leftmargin}{1.2cm} \setlength{\rightmargin}{1.2cm}
\setlength{\oddsidemargin}{0cm}\setlength{\evensidemargin}{0cm}

%%%%%%%%%%%%%%%%%%%%%%%%%%%%%%%%%%%%%%%%%%%%%%%%%%%%%%%%%%%%%%%%%%%%%%%%

% \renewcommand{\baselinestretch}{1.6}
% \renewcommand{\thefootnote}{\fnsymbol{footnote}}
% \renewcommand{\theequation}{\thesection.\arabic{equation}}
% \setlength{\voffset}{-50pt}
% \numberwithin{equation}{chapter}

%%%%%%%%%%%%%%%%%%%%%%%%%%%%%%%%%%%%%%%%%%%%%%%%%%%%%%%%%%%%%%%%%%%%%%%

\begin{document}
\thispagestyle{empty}
\pagestyle{empty}
\begin{minipage}[h]{0.14\textwidth}
	\includegraphics[scale=0.24]{../../ufgd.png}
\end{minipage}
\begin{minipage}[h]{\textwidth}
\begin{tabular}{c}
{{\bf UNIVERSIDADE FEDERAL DA GRANDE DOURADOS}}\\
{{\bf \'{A}lgebra Linear e Geometria Anal\'{\i}tica --- Lista 9}}\\
{{\bf Prof.\ Adriano Barbosa}}\\
\end{tabular}
\vspace{-0.45cm}
%
\end{minipage}

%------------------------

\vspace{1cm}
%%%%%%%%%%%%%%%%%%%%%%%%%%%%%%%%   formulario  inicio  %%%%%%%%%%%%%%%%%%%%%%%%%%%%%%%%
\begin{enumerate}
	\vspace{0.5cm}
	\item Encontre o dom\'{\i}nio e o contra-dom\'{\i}nio das transforma\c{c}\~oes abaixo e
		determine se s\~ao lineares:
		\begin{enumerate}
			\item $T(x,y,z) = (3x-2y+4z, 5x-8y+z)$
			\item $T(x,y) = (2xy-y, x+3xy, x+y)$
			\item $T(x,y,z,t) = (x^2-3y+z-2t, 3x-4y-z^2+t)$
		\end{enumerate}

	\vspace{0.5cm}
	\item Encontre a matriz can\^onica das transforma\c{c}\~oes lineares abaixo
		\begin{enumerate}
			\item $T: \mathbb{R}^4 \rightarrow \mathbb{R}2$, $T(x,y,z,t) =
				2x-3y+t, 3x+5y-t)$
			\item $T: \mathbb{R}^4 \rightarrow \mathbb{R}^4$, $T(x,y,z,t) = (x,
				x+y, x+y+z, x+y+z+t)$
			\item $T: \mathbb{R}^2 \rightarrow \mathbb{R}^3$, $T(x,y) = (-x+y,
				3x-2y, 5x-7y)$
		\end{enumerate}

	\vspace{0.5cm}
	\item Encontre a transforma\c{c}\~ao linear cuja matriz can\^onica \'e dada abaixo:

		\begin{enumerate*}
			\item $\begin{bmatrix}
					1 & 2 \\
					3 & 4
				\end{bmatrix}$
			\hspace{0.5cm}
			\hspace{0.5cm}
			\item $\begin{bmatrix}
					-1 & 2 & 0 \\
					3 & 1 & 5
				\end{bmatrix}$
			\hspace{0.5cm}
			\hspace{0.5cm}
			\item $\begin{bmatrix}
					-1 & 1 \\
					2 & 4 \\
					7 & 8
				\end{bmatrix}$
		\end{enumerate*}

	\vspace{0.5cm}
	\item Use a matriz can\^onica $[T]$ para obter $T(v)$ e em seguida confira
		o resultado calculando $T(v)$ diretamente:
		\begin{enumerate}
			\item $T: \mathbb{R}^2 \rightarrow \mathbb{R}^3$, $T(x,y) = (-x+y,
				y, x-y)$ avaliada em $(1,2)$
			\item $T: \mathbb{R}^3 \rightarrow \mathbb{R}^3$, $T(x,y,z) =
				(-x+2y, y-3z, x-y-z)$ avaliada em $(2,3,0)$
		\end{enumerate}

	\vspace{0.5cm}
	\item Encontre a matriz can\^onica para cada composi\c{c}\~ao abaixo:
		\begin{enumerate}
			\item Uma rota\c{c}\~ao de $90\degree$ seguida de uma reflex\~ao em torno
				do eixo $y$.
			\item Uma reflex\~ao em torno do eixo $x$ seguida de uma escala de
				raz\~ao $k=3$.
			\item Uma rota\c{c}\~ao de $60\degree$, seguida de uma proje\c{c}\~ao ortogonal
				sobre o eixo $x$, seguida de uma reflex\~ao em torno do eixo $y$.
			\item Uma rota\c{c}\~ao de $15\degree$, seguida de uma rota\c{c}\~ao de
				$105\degree$, seguida de uma rota\c{c}\~ao de $60\degree$.
		\end{enumerate}

	\vspace{0.5cm}
	\item Determine se $T_1\circ T_2 = T_2 \circ T_1$:
		\begin{enumerate}
			\item $T_1:\mathbb{R}^2 \rightarrow \mathbb{R}^2$ \'e a proje\c{c}\~ao
				ortogonal sobre o eixo $x$ e $T_2: \mathbb{R}^2 \rightarrow
				\mathbb{R}^2$ \'e a proje\c{c}\~ao ortogonal sobre o eixo $y$.
			\item $T_1:\mathbb{R}^2 \rightarrow \mathbb{R}^2$ \'e a rota\c{c}\~ao por
				um \^angulo $\theta_1$ e $T_2: \mathbb{R}^2 \rightarrow
				\mathbb{R}^2$ \'e a rota\c{c}\~ao por um \^angulo $\theta_2$.
			\item $T_1:\mathbb{R}^2 \rightarrow \mathbb{R}^2$ \'e a rota\c{c}\~ao por
				um \^angulo $\theta$ e $T_2: \mathbb{R}^2 \rightarrow
				\mathbb{R}^2$ \'e a proje\c{c}\~ao ortogonal sobre o eixo $x$.
		\end{enumerate}

	\vspace{0.5cm}
	\item Definimos as proje\c{c}\~oes ortogonais de $\mathbb{R}^3$ sobre os eixos
		$x$, $y$ e $x$, respectivamente, por
		\[T_x(x,y,z) = (x,0,0),\ T_y(x,y,z) = (0,y,0)\ e\ T_z(x,y,z) =
			(0,0,z).\]
	Mostre que as proje\c{c}\~oes acima s\~ao transforma\c{c}\~oes lineares.

	\vspace{0.5cm}
	\item Mostre que a reflex\~ao de vetores de $\mathbb{R}^2$ em torno da
		reta $y=x$ \'e uma transforma\c{c}\~ao linear e encontre sua matriz can\^onica.

	\vspace{0.5cm}
	\item Mostre que a proje\c{c}\~ao ortogonal de vetores de $\mathbb{R}^2$ sobre a
		reta $y=x$ \'e uma transforma\c{c}\~ao linear e encontre sua matriz can\^onica.

	\vspace{0.5cm}
	\item Mostre que os vetores $T(v)$ e $v - T(v)$ s\~ao ortogonais, onde
		$T:\mathbb{R}^2 \rightarrow \mathbb{R}^2$ \'e uma proje\c{c}\~ao ortogonal
		sobre os eixos coordenados ou sobre a reta $y=x$.
\end{enumerate}

\end{document}
