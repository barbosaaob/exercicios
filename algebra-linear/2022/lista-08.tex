\documentclass[a4paper,5pt]{amsbook}
%%%%%%%%%%%%%%%%%%%%%%%%%%%%%%%%%%%%%%%%%%%%%%%%%%%%%%%%%%%%%%%%%%%%%

\usepackage{booktabs}
\usepackage{graphicx}
\usepackage{multicol}
\usepackage{textcomp}
\usepackage{systeme}
\usepackage{amssymb}
\usepackage[]{amsmath}
\usepackage{subcaption}
\usepackage[inline]{enumitem}

%%%%%%%%%%%%%%%%%%%%%%%%%%%%%%%%%%%%%%%%%%%%%%%%%%%%%%%%%%%%%%

\newcommand{\sen}{\,\mbox{sen}\,}
\newcommand{\tg}{\,\mbox{tg}\,}
\newcommand{\cosec}{\,\mbox{cosec}\,}
\newcommand{\cotg}{\,\mbox{cotg}\,}
\newcommand{\tr}{\,\mbox{tr}\,}
\newcommand{\ds}{\displaystyle}

%%%%%%%%%%%%%%%%%%%%%%%%%%%%%%%%%%%%%%%%%%%%%%%%%%%%%%%%%%%%%%%%%%%%%%%%

\setlength{\textwidth}{16cm} %\setlength{\topmargin}{-1.3cm}
\setlength{\textheight}{30cm}
\setlength{\leftmargin}{1.2cm} \setlength{\rightmargin}{1.2cm}
\setlength{\oddsidemargin}{0cm}\setlength{\evensidemargin}{0cm}

%%%%%%%%%%%%%%%%%%%%%%%%%%%%%%%%%%%%%%%%%%%%%%%%%%%%%%%%%%%%%%%%%%%%%%%%

% \renewcommand{\baselinestretch}{1.6}
% \renewcommand{\thefootnote}{\fnsymbol{footnote}}
% \renewcommand{\theequation}{\thesection.\arabic{equation}}
% \setlength{\voffset}{-50pt}
% \numberwithin{equation}{chapter}

%%%%%%%%%%%%%%%%%%%%%%%%%%%%%%%%%%%%%%%%%%%%%%%%%%%%%%%%%%%%%%%%%%%%%%%

\begin{document}
\thispagestyle{empty}
\pagestyle{empty}
\begin{minipage}[h]{0.14\textwidth}
	\includegraphics[scale=0.24]{../../ufgd.png}
\end{minipage}
\begin{minipage}[h]{\textwidth}
\begin{tabular}{c}
{{\bf UNIVERSIDADE FEDERAL DA GRANDE DOURADOS}}\\
{{\bf \'{A}lgebra Linear e Geometria Anal\'{\i}tica --- Lista 8}}\\
{{\bf Prof.\ Adriano Barbosa}}\\
\end{tabular}
\vspace{-0.45cm}
%
\end{minipage}

%------------------------

\vspace{1cm}
%%%%%%%%%%%%%%%%%%%%%%%%%%%%%%%%   formulario  inicio  %%%%%%%%%%%%%%%%%%%%%%%%%%%%%%%%
\begin{enumerate}
	\vspace{0.5cm}
    \item Determine se os conjuntos de vetores s\~ao linearmente independentes ou
        linearmente dependentes em $\mathbb{R}^2$.
        \begin{enumerate}
            \item $(2,1)$, $(3,0)$
            \item $(4,1)$, $(-7,-8)$
            \item $(0,0)$, $(1,3)$
            \item $(3,9)$, $(-4,-12)$
        \end{enumerate}
        
    \vspace{0.5cm}
    \item Quais dos conjuntos de vetores do exerc\'{\i}cio anterior s\~ao base de
        $\mathbb{R}^2$?
        
    \vspace{0.5cm}
    \item Quais dos seguintes conjuntos de vetores s\~ao base de $\mathbb{R}^3$?
        \begin{enumerate}
            \item $(1,0,0)$, $(2,2,0)$, $(3,3,0)$
            \item $(2,-3,1)$, $(4,1,1)$, $(0,-7,1)$
            \item $(3,1,-4)$, $(2,5,6)$, $(1,4,8)$
            \item $(1,6,4)$, $(2,4,-1)$, $(-1,2,5)$
        \end{enumerate}

    \vspace{0.5cm}
    \item Explique por que os seguintes conjuntos de vetores n\~ao s\~ao base dos
        espa\c{c}os vetoriais indicados.
        \begin{enumerate}
            \item $(1,2)$, $(0,3)$, $(2,7)$ de $\mathbb{R}^2$
            \item $(-1,3,2)$, $(6,1,1)$ de $\mathbb{R}^3$
        \end{enumerate}

    \vspace{0.5cm}
    \item Encontre as coordenadas de $w$ em rela\c{c}\~ao a base $S=\{u_1,u_2\}$ de
        $\mathbb{R}^2$.
        \begin{enumerate}
            \item $u_1=(1,0)$, $u_2=(0,1)$; $w=(3,-7)$
            \item $u_1=(2,-4)$, $u_2=(3,8)$; $w=(1,1)$
        \end{enumerate}

    \vspace{0.5cm}
    \item Encontre as coordenadas de $w$ em rela\c{c}\~ao a base $S=\{u_1,u_2,u_3\}$
        de $\mathbb{R}^3$.
        \begin{enumerate}
            \item $u_1=(1,0,0)$, $u_2=(2,2,0)$, $u_3=(3,3,3)$; $w=(2,-1,3)$
            \item $u_1=(1,2,3)$, $u_2=(-4,5,6)$, $u_3=(7,-8,9)$; $w=(5,-12,3)$
        \end{enumerate}

    \vspace{0.5cm}
    \item Determine bases dos seguintes subespa\c{c}os de $\mathbb{R}^3$.
        \begin{enumerate}
            \item o plano $3x-2y+5z=0$
            \item o plano $x-y=0$
            \item a reta $x=2t$, $y=-t$, $z=4t$
        \end{enumerate}

    \vspace{0.5cm}
    \item Determine a dimens\~ao dos seguintes subespa\c{c}os de $\mathbb{R}^4$
        \begin{enumerate}
            \item conjuntos dos vetores da forma $(a,b,c,0)$
            \item conjuntos dos vetores da forma $(a,b,a-b,a+b)$
            \item conjuntos dos vetores da forma $(a,a,a,a)$
        \end{enumerate}
\end{enumerate}

\end{document}
