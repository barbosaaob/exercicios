\documentclass[a4paper,5pt]{amsbook}
%%%%%%%%%%%%%%%%%%%%%%%%%%%%%%%%%%%%%%%%%%%%%%%%%%%%%%%%%%%%%%%%%%%%%

\usepackage{booktabs}
\usepackage{graphicx}
\usepackage{multicol}
\usepackage{textcomp}
\usepackage{systeme}
\usepackage{amssymb}
\usepackage[]{amsmath}
\usepackage{subcaption}
\usepackage[inline]{enumitem}
\usepackage{gensymb}
\usepackage[utf8]{inputenc}

%%%%%%%%%%%%%%%%%%%%%%%%%%%%%%%%%%%%%%%%%%%%%%%%%%%%%%%%%%%%%%

\renewcommand{\sin}{\,\mbox{sen}\,}
\newcommand{\sen}{\,\mbox{sen}\,}
\newcommand{\tg}{\,\mbox{tg}\,}
\newcommand{\cosec}{\,\mbox{cosec}\,}
\newcommand{\cotg}{\,\mbox{cotg}\,}
\newcommand{\tr}{\,\mbox{tr}\,}
\newcommand{\ds}{\displaystyle}

%%%%%%%%%%%%%%%%%%%%%%%%%%%%%%%%%%%%%%%%%%%%%%%%%%%%%%%%%%%%%%%%%%%%%%%%

\setlength{\textwidth}{16cm} \setlength{\topmargin}{-1.3cm}
\setlength{\textheight}{30cm}
\setlength{\leftmargin}{1.2cm} \setlength{\rightmargin}{1.2cm}
\setlength{\oddsidemargin}{0cm}\setlength{\evensidemargin}{0cm}

%%%%%%%%%%%%%%%%%%%%%%%%%%%%%%%%%%%%%%%%%%%%%%%%%%%%%%%%%%%%%%%%%%%%%%%%

% \renewcommand{\baselinestretch}{1.6}
% \renewcommand{\thefootnote}{\fnsymbol{footnote}}
% \renewcommand{\theequation}{\thesection.\arabic{equation}}
% \setlength{\voffset}{-50pt}
% \numberwithin{equation}{chapter}

%%%%%%%%%%%%%%%%%%%%%%%%%%%%%%%%%%%%%%%%%%%%%%%%%%%%%%%%%%%%%%%%%%%%%%%

\begin{document}
\thispagestyle{empty}
\pagestyle{empty}
\begin{minipage}[h]{0.14\textwidth}
    \includegraphics[scale=0.24]{../../ufgd.png}
\end{minipage}
\begin{minipage}[h]{\textwidth}
    \begin{tabular}{c}
        {{\bf UNIVERSIDADE FEDERAL DA GRANDE DOURADOS}}\\
        {{\bf C\'{a}lculo Diferencial e Integral II --- Lista 10}}\\
        {{\bf Prof.\ Adriano Barbosa}}\\
    \end{tabular}
    \vspace{-0.45cm}
    %
\end{minipage}

%------------------------

\vspace{1cm}
%%%%%%%%%%%%%%%%%%%%%%%%%%%%%%%%   formulario  inicio  %%%%%%%%%%%%%%%%%%%%%%%%%%%%%%%%
\begin{enumerate}
    \setlength\itemsep{0.5cm}
    \item Escreva as cinco primeiras parcelas das s\'eries e calcule, se
    poss\'{\i}vel, sua soma:
    	\begin{enumerate}
            \setlength\itemsep{0.3cm}
    		\item $\ds\sum_{n=1}^\infty\ds\frac{1}{n(n+1)}$
    
                [Dica: escreva como soma de fra\c{c}\~oes parciais e em
                seguida escreva as somas parciais.]
    		\item $\ds\sum_{k=0}^\infty \ds\frac{1}{(4k+1)(4k+5)}$
    
                [Dica: utilize a mesma estrat\'egia do item acima.]
    		\item $\ds\sum_{n=1}^\infty\ds\frac{n^2}{3n^2+2}$
    		\item $\ds\sum_{k=0}^\infty e^{-k}$
    		%\item $\ds1+\frac{1}{\sqrt[3]{2}}+\frac{1}{\sqrt[3]{3}}+\cdots+\frac{1}{\sqrt[3]{n}}+\cdots$
    	\end{enumerate}
    
    \item Determine se as s\'eries geom\'etricas s\~ao convergentes ou divergentes.
    Calcule a soma das s\'eries convergentes.
    	\begin{enumerate}
            \setlength\itemsep{0.3cm}
    		\item $\ds4+3+\frac{9}{4}+\frac{27}{16}+\cdots$
    		\item $\ds2+0,5+0,125+0,03125+\cdots$
    	\end{enumerate}
    
    \item Escreva $0,88888\ldots$ como uma fra\c{c}\~ao.
    
    \item Calcule a soma das s\'eries
    	\begin{enumerate}
            \setlength\itemsep{0.3cm}
    		\item $\ds\left(\frac{1}{2}+\frac{1}{4}\right)+\left(\frac{1}{2^2}+\frac{1}{4^2}\right)+\left(\frac{1}{2^3}+\frac{1}{4^3}\right)+\cdots$
    		\item $\ds\sum_{k=1}^\infty \left(\frac{1}{5^k} - \frac{1}{k(k+1)}\right)$
    	\end{enumerate}
\end{enumerate}
\end{document}
