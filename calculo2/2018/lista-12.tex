\documentclass[a4paper,5pt]{amsbook}
%%%%%%%%%%%%%%%%%%%%%%%%%%%%%%%%%%%%%%%%%%%%%%%%%%%%%%%%%%%%%%%%%%%%%

\usepackage{booktabs}
\usepackage{graphicx}
\usepackage{multicol}
\usepackage{textcomp}
\usepackage{systeme}
\usepackage{amssymb}
\usepackage[]{amsmath}
\usepackage{subcaption}
\usepackage[inline]{enumitem}
\usepackage{gensymb}
\usepackage[utf8]{inputenc}

%%%%%%%%%%%%%%%%%%%%%%%%%%%%%%%%%%%%%%%%%%%%%%%%%%%%%%%%%%%%%%

\renewcommand{\sin}{\,\mbox{sen}\,}
\newcommand{\sen}{\,\mbox{sen}\,}
\newcommand{\tg}{\,\mbox{tg}\,}
\newcommand{\cosec}{\,\mbox{cosec}\,}
\newcommand{\cotg}{\,\mbox{cotg}\,}
\newcommand{\tr}{\,\mbox{tr}\,}
\newcommand{\ds}{\displaystyle}

%%%%%%%%%%%%%%%%%%%%%%%%%%%%%%%%%%%%%%%%%%%%%%%%%%%%%%%%%%%%%%%%%%%%%%%%

\setlength{\textwidth}{16cm} \setlength{\topmargin}{-1.3cm}
\setlength{\textheight}{30cm}
\setlength{\leftmargin}{1.2cm} \setlength{\rightmargin}{1.2cm}
\setlength{\oddsidemargin}{0cm}\setlength{\evensidemargin}{0cm}

%%%%%%%%%%%%%%%%%%%%%%%%%%%%%%%%%%%%%%%%%%%%%%%%%%%%%%%%%%%%%%%%%%%%%%%%

% \renewcommand{\baselinestretch}{1.6}
% \renewcommand{\thefootnote}{\fnsymbol{footnote}}
% \renewcommand{\theequation}{\thesection.\arabic{equation}}
% \setlength{\voffset}{-50pt}
% \numberwithin{equation}{chapter}

%%%%%%%%%%%%%%%%%%%%%%%%%%%%%%%%%%%%%%%%%%%%%%%%%%%%%%%%%%%%%%%%%%%%%%%

\begin{document}
\thispagestyle{empty}
\pagestyle{empty}
\begin{minipage}[h]{0.14\textwidth}
    \includegraphics[scale=0.24]{../../ufgd.png}
\end{minipage}
\begin{minipage}[h]{\textwidth}
    \begin{tabular}{c}
        {{\bf UNIVERSIDADE FEDERAL DA GRANDE DOURADOS}}\\
        {{\bf C\'{a}lculo Diferencial e Integral II --- Lista 12}}\\
        {{\bf Prof.\ Adriano Barbosa}}\\
    \end{tabular}
    \vspace{-0.45cm}
    %
\end{minipage}

%------------------------

\vspace{1cm}
%%%%%%%%%%%%%%%%%%%%%%%%%%%%%%%%   formulario  inicio  %%%%%%%%%%%%%%%%%%%%%%%%%%%%%%%%
\begin{enumerate}
    \setlength\itemsep{0.5cm}
    \item Determine para quais valores de $x\in\mathbb{R}$ as s\'eries s\~ao
    convergentes
    	\begin{enumerate}
    		\item $\ds\sum_{n=1}^\infty (-1)^n nx^n$
    		\item $\ds\sum_{n=1}^\infty \frac{x^n}{n3^n}$
    		\item $\ds\sum_{n=1}^\infty \frac{3^n(x+4)^n}{\sqrt{n}}$
    		\item $\ds\sum_{n=0}^\infty \frac{n(x+2)^n}{3^{n+1}}$
    		\item $\ds\sum_{n=1}^\infty \frac{(x-2)^n}{n^n}$
    		\item $\ds\sum_{n=1}^\infty n!(2x-1)^n$
    	\end{enumerate}
    
    \item
    	\begin{enumerate}
            \item Escreva as fun\c{c}\~oes $\sin x$ e $\cos x$ como s\'erie de
            Maclaurin e encontre seu raio e intervalo de converg\^encia.
            \item Utilize o item $(a)$ e a s\'erie de Maclaurin da fun\c{c}\~ao $e^x$
            para verificar a F\'ormula de Euler: $e^{ix} = \cos x + i\ \sin x$, onde
            $i$ \'e a unidade imagin\'aria.
    	\end{enumerate}
    
    \item Encontre a s\'erie de Taylor das fun\c{c}\~oes abaixo centradas no valor dado
    	\begin{enumerate}
    		\item $f(x) = x^4 - 3x^2 + 1$, $a = 1$
    		\item $f(x) = \ln x$, $a = 2$
            \item $f(x) = \ds\frac{1}{x}$, $a=-3$
    	\end{enumerate}
\end{enumerate}
\end{document}
