\documentclass[a4paper,5pt]{amsbook}
%%%%%%%%%%%%%%%%%%%%%%%%%%%%%%%%%%%%%%%%%%%%%%%%%%%%%%%%%%%%%%%%%%%%%

\usepackage{booktabs}
\usepackage{graphicx}
\usepackage{multicol}
\usepackage{textcomp}
\usepackage{systeme}
\usepackage{amssymb}
\usepackage[]{amsmath}
\usepackage{subcaption}
\usepackage[inline]{enumitem}
\usepackage{gensymb}
\usepackage[utf8]{inputenc}

%%%%%%%%%%%%%%%%%%%%%%%%%%%%%%%%%%%%%%%%%%%%%%%%%%%%%%%%%%%%%%

\newcommand{\sen}{\,\mbox{sen}\,}
\newcommand{\tg}{\,\mbox{tg}\,}
\newcommand{\cosec}{\,\mbox{cosec}\,}
\newcommand{\cotg}{\,\mbox{cotg}\,}
\newcommand{\tr}{\,\mbox{tr}\,}
\newcommand{\ds}{\displaystyle}

%%%%%%%%%%%%%%%%%%%%%%%%%%%%%%%%%%%%%%%%%%%%%%%%%%%%%%%%%%%%%%%%%%%%%%%%

\setlength{\textwidth}{16cm} \setlength{\topmargin}{-1.3cm}
\setlength{\textheight}{30cm}
\setlength{\leftmargin}{1.2cm} \setlength{\rightmargin}{1.2cm}
\setlength{\oddsidemargin}{0cm}\setlength{\evensidemargin}{0cm}

%%%%%%%%%%%%%%%%%%%%%%%%%%%%%%%%%%%%%%%%%%%%%%%%%%%%%%%%%%%%%%%%%%%%%%%%

% \renewcommand{\baselinestretch}{1.6}
% \renewcommand{\thefootnote}{\fnsymbol{footnote}}
% \renewcommand{\theequation}{\thesection.\arabic{equation}}
% \setlength{\voffset}{-50pt}
% \numberwithin{equation}{chapter}

%%%%%%%%%%%%%%%%%%%%%%%%%%%%%%%%%%%%%%%%%%%%%%%%%%%%%%%%%%%%%%%%%%%%%%%

\begin{document}
\thispagestyle{empty}
\pagestyle{empty}
\begin{minipage}[h]{0.14\textwidth}
	\includegraphics[scale=0.24]{../../ufgd.png}
\end{minipage}
\begin{minipage}[h]{\textwidth}
\begin{tabular}{c}
{{\bf UNIVERSIDADE FEDERAL DA GRANDE DOURADOS}}\\
{{\bf C\'{a}lculo Diferencial e Integral II --- Lista 8}}\\
{{\bf Prof.\ Adriano Barbosa}}\\
\end{tabular}
\vspace{-0.45cm}
%
\end{minipage}

%------------------------

\vspace{1cm}
%%%%%%%%%%%%%%%%%%%%%%%%%%%%%%%%   formulario  inicio  %%%%%%%%%%%%%%%%%%%%%%%%%%%%%%%%
\begin{enumerate}
	\vspace{0.5cm}
    \item Mostre que $y=\frac{2}{3}e^x + e^{-2x}$ \'e uma solu\c{c}\~ao da equa\c{c}\~ao
        diferencial $y'+2y=2e^x$.

	\vspace{0.5cm}
    \item
        \begin{enumerate}
            \item Para quais valores de $r$ a fun\c{c}\~ao $y=e^{rx}$ satisfaz a
                equa\c{c}\~ao diferencial $2y''+y'-y=0$?
            \vspace{0.3cm}
            \item Se $r_1$ e $r_2$ s\~ao os valores de $r$ encontrados no item
                (a), mostre que $y=ae^{r_1 x} + be^{r_2 x}$ tamb\'em \'e uma
                solu\c{c}\~ao da EDO quaisquer que sejam $a$, $b\in\mathbb{R}$.
        \end{enumerate}

	\vspace{0.5cm}
    \item Uma popula\c{c}\~ao \'e modelada pela equa\c{c}\~ao diferencial
        \[\frac{dP}{dt}=1,2P\left(1-\frac{P}{4200}\right)\]
        \begin{enumerate}
            \item Para quais valores de $P$ a popula\c{c}\~ao cresce?
            \item Para quais valores de $P$ a popula\c{c}\~ao decresce?
        \end{enumerate}

	\vspace{0.5cm}
    \item Resolva as equa\c{c}\~oes diferenciais abaixo:
        \begin{enumerate}
            \vspace{0.3cm}
            \item $\ds\frac{dp}{dt}=t^2p-p+t^2-1$
            \vspace{0.3cm}
            \item $\ds(y+\sen y)y'=x+x^3$
            \vspace{0.3cm}
            \item $\ds\frac{dy}{dt}=\frac{t}{ye^{y+t^2}}$
        \end{enumerate}

	\vspace{0.5cm}
    \item Resolva os problemas de valor inicial abaixo:
        \begin{enumerate}
            \vspace{0.3cm}
            \item $\ds\frac{dy}{dx}=\frac{\ln{x}}{xy}$, $y(1)=2$
            \vspace{0.3cm}
            \item $\ds\frac{dy}{dx}=\frac{x}{y}$, $y(0)=-3$
            \vspace{0.3cm}
            \item $\ds\frac{dP}{dt}=\sqrt{Pt}$, $P(1)=2$
            \vspace{0.3cm}
            \item $\ds x\ln{x}=y(1+\sqrt{3+y^2})y'$, $y(1)=1$
        \end{enumerate}

	\vspace{0.5cm}
    \item Resolva a equa\c{c}\~ao diferencial $y'=x+y$ utilizando a mudan\c{c}a de vari\'aveis $u=x+y$.

	\vspace{0.5cm}
    \item Use a mudan\c{c}a de vari\'aveis $v=y/x$ para resolver a EDO $xy'=y+xe^{y/x}$.
\end{enumerate}
\end{document}
