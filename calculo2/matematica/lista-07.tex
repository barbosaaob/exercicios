\documentclass[a4paper,5pt]{amsbook}
%%%%%%%%%%%%%%%%%%%%%%%%%%%%%%%%%%%%%%%%%%%%%%%%%%%%%%%%%%%%%%%%%%%%%

\usepackage{booktabs}
\usepackage{graphicx}
\usepackage{multicol}
\usepackage{textcomp}
\usepackage{systeme}
\usepackage{amssymb}
\usepackage[]{amsmath}
\usepackage{subcaption}
\usepackage[inline]{enumitem}
\usepackage{gensymb}
\usepackage[utf8]{inputenc}

%%%%%%%%%%%%%%%%%%%%%%%%%%%%%%%%%%%%%%%%%%%%%%%%%%%%%%%%%%%%%%

\newcommand{\sen}{\,\mbox{sen}\,}
\newcommand{\tg}{\,\mbox{tg}\,}
\newcommand{\cosec}{\,\mbox{cosec}\,}
\newcommand{\cotg}{\,\mbox{cotg}\,}
\newcommand{\tr}{\,\mbox{tr}\,}
\newcommand{\ds}{\displaystyle}

%%%%%%%%%%%%%%%%%%%%%%%%%%%%%%%%%%%%%%%%%%%%%%%%%%%%%%%%%%%%%%%%%%%%%%%%

\setlength{\textwidth}{16cm} \setlength{\topmargin}{-1.3cm}
\setlength{\textheight}{30cm}
\setlength{\leftmargin}{1.2cm} \setlength{\rightmargin}{1.2cm}
\setlength{\oddsidemargin}{0cm}\setlength{\evensidemargin}{0cm}

%%%%%%%%%%%%%%%%%%%%%%%%%%%%%%%%%%%%%%%%%%%%%%%%%%%%%%%%%%%%%%%%%%%%%%%%

% \renewcommand{\baselinestretch}{1.6}
% \renewcommand{\thefootnote}{\fnsymbol{footnote}}
% \renewcommand{\theequation}{\thesection.\arabic{equation}}
% \setlength{\voffset}{-50pt}
% \numberwithin{equation}{chapter}

%%%%%%%%%%%%%%%%%%%%%%%%%%%%%%%%%%%%%%%%%%%%%%%%%%%%%%%%%%%%%%%%%%%%%%%

\begin{document}
\thispagestyle{empty}
\pagestyle{empty}
\begin{minipage}[h]{0.14\textwidth}
	\includegraphics[scale=0.24]{../../ufgd.png}
\end{minipage}
\begin{minipage}[h]{\textwidth}
\begin{tabular}{c}
{{\bf UNIVERSIDADE FEDERAL DA GRANDE DOURADOS}}\\
{{\bf C\'{a}lculo 2 --- Lista 7}}\\
{{\bf Prof.\ Adriano Barbosa}}\\
\end{tabular}
\vspace{-0.45cm}
%
\end{minipage}

%------------------------

\vspace{1cm}
%%%%%%%%%%%%%%%%%%%%%%%%%%%%%%%%   formulario  inicio  %%%%%%%%%%%%%%%%%%%%%%%%%%%%%%%%
\begin{enumerate}
	\vspace{0.5cm}
    \item Calcule as integrais abaixo utilizando a mudan\c{c}a de vari\'aveis indicada:
        \begin{enumerate}
            \item $\ds\int\frac{dx}{x^2\sqrt{4-x^2}}$, $x=2\sen{\theta}$
            \vspace{0.2cm}
            \item $\ds\int\frac{\sqrt{x^2-4}}{x}\ dx$, $x=2\sec{\theta}$
        \end{enumerate}

	\vspace{0.5cm}
    \item Calcule as integrais:
        \begin{enumerate}
            \item $\ds\int\frac{dx}{\sqrt{x^2+16}}$
            \vspace{0.2cm}
            \item $\ds\int\sqrt{1-4x^2}\ dx$
            \vspace{0.2cm}
            \item $\ds\int\frac{\sqrt{x^2-9}}{x^3}\ dx$
            \vspace{0.2cm}
            \item $\ds\int\frac{x}{\sqrt{x^2-7}}\ dx$
            \vspace{0.2cm}
            \item $\ds\int\frac{\sqrt{1+x^2}}{x}\ dx$
            \vspace{0.2cm}
            \item $\ds\int_{\sqrt{2}}^{2}\frac{1}{t^3\sqrt{t^2-1}}\ dt$
            \vspace{0.2cm}
            \item $\ds\int_0^{0,6}\frac{x^2}{\sqrt{9-25x^2}}\ dx$
            \vspace{0.2cm}
            \item $\ds\int\frac{x}{\sqrt{x^2+x+1}}\ dx$
            \vspace{0.2cm}
            \item $\ds\int\sqrt{5+4x-x^2}\ dx$
            \vspace{0.2cm}
            \item $\ds\int x\sqrt{1-x^4}\ dx$
        \end{enumerate}

    \vspace{0.5cm}
    \item A par\'abola $y=\frac{1}{2}x^2$ divide o disco $x^2+y^2\le 8$ em duas
        partes. Calcule a \'area de ambas as partes.

    \vspace{0.5cm}
    \item Calcule a \'area da regi\~ao limitada pela hip\'erbole $9x^2-4y^2=36$ e
        pela reta $x=3$.
\end{enumerate}
\end{document}
