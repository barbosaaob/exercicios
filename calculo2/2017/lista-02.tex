\documentclass[a4paper,5pt]{amsbook}
%%%%%%%%%%%%%%%%%%%%%%%%%%%%%%%%%%%%%%%%%%%%%%%%%%%%%%%%%%%%%%%%%%%%%

\usepackage{booktabs}
\usepackage{graphicx}
\usepackage{multicol}
\usepackage{textcomp}
\usepackage{systeme}
\usepackage{amssymb}
\usepackage[]{amsmath}
\usepackage{subcaption}
\usepackage[inline]{enumitem}
\usepackage{gensymb}

%%%%%%%%%%%%%%%%%%%%%%%%%%%%%%%%%%%%%%%%%%%%%%%%%%%%%%%%%%%%%%

\newcommand{\sen}{\,\mbox{sen}\,}
\newcommand{\tg}{\,\mbox{tg}\,}
\newcommand{\cosec}{\,\mbox{cosec}\,}
\newcommand{\cotg}{\,\mbox{cotg}\,}
\newcommand{\tr}{\,\mbox{tr}\,}
\newcommand{\ds}{\displaystyle}

%%%%%%%%%%%%%%%%%%%%%%%%%%%%%%%%%%%%%%%%%%%%%%%%%%%%%%%%%%%%%%%%%%%%%%%%

\setlength{\textwidth}{16cm} \setlength{\topmargin}{-1.3cm}
\setlength{\textheight}{30cm}
\setlength{\leftmargin}{1.2cm} \setlength{\rightmargin}{1.2cm}
\setlength{\oddsidemargin}{0cm}\setlength{\evensidemargin}{0cm}

%%%%%%%%%%%%%%%%%%%%%%%%%%%%%%%%%%%%%%%%%%%%%%%%%%%%%%%%%%%%%%%%%%%%%%%%

% \renewcommand{\baselinestretch}{1.6}
% \renewcommand{\thefootnote}{\fnsymbol{footnote}}
% \renewcommand{\theequation}{\thesection.\arabic{equation}}
% \setlength{\voffset}{-50pt}
% \numberwithin{equation}{chapter}

%%%%%%%%%%%%%%%%%%%%%%%%%%%%%%%%%%%%%%%%%%%%%%%%%%%%%%%%%%%%%%%%%%%%%%%

\begin{document}
\thispagestyle{empty}
\pagestyle{empty}
\begin{minipage}[h]{0.14\textwidth}
	\includegraphics[scale=0.24]{../../ufgd.png}
\end{minipage}
\begin{minipage}[h]{\textwidth}
\begin{tabular}{c}
{{\bf UNIVERSIDADE FEDERAL DA GRANDE DOURADOS}}\\
{{\bf C\'{a}lculo Diferencial e Integral II --- Lista 2}}\\
{{\bf Prof.\ Adriano Barbosa}}\\
\end{tabular}
\vspace{-0.45cm}
%
\end{minipage}

%------------------------

\vspace{1cm}
%%%%%%%%%%%%%%%%%%%%%%%%%%%%%%%%   formulario  inicio  %%%%%%%%%%%%%%%%%%%%%%%%%%%%%%%%
\begin{enumerate}
	\vspace{0.5cm}
	\item Resolva as integrais utilizando as escolhas de $u$ e $v$ dadas:
		\begin{enumerate}
			\item $\ds\int x^2\ln{x}\ dx;\ u=\ln{x}, dv=x^2\ dx$
			\item $\ds\int \theta \cos{\theta}\ d\theta;\ u=\theta,
				dv=\cos{\theta}\ d\theta$
		\end{enumerate}

	\vspace{0.5cm}
	\item Resolva as integrais indefinidas:
		\begin{enumerate}
			\item $\ds\int x\cos(5x)\ dx$
			\item $\ds\int (x^2+2x)\cos{x}\ dx$
			\item $\ds\int \ln(\sqrt[3]{x})\ dx$
			\item $\ds\int {(\ln{x})}^2\ dx$
			\item $\ds\int \frac{xe^{2x}}{{(1+2x)}^2}\ dx$
			\item $\ds\int z^3e^z\ dz$
		\end{enumerate}
	
	\vspace{0.5cm}
	\item Resolva as integrais definidas:
		\begin{enumerate}
			\item $\ds\int_1^3 r^3\ln{r}\ dr$
			\item $\ds\int_0^1 \frac{y}{e^{2y}}\ dy$
			\item $\ds\int_1^2x^4{(\ln{x})}^2\ dx$
			\item $\ds\int_0^{1/2} x \cos(\pi x)\ dx$
			\item $\ds\int_0^t e^s \sen(t-s)\ ds$
		\end{enumerate}

	\vspace{0.5cm}
	\item Efetue uma substitui\c{c}\~ao e em seguida use a integra\c{c}\~ao por partes para
		resolver as integrais abaixo:
		\begin{enumerate}
			\item $\ds\int \cos(\sqrt{x})\ dx$
			\item $\ds\int x \ln(1+x)\ dx$
		\end{enumerate}

\end{enumerate}
\end{document}
