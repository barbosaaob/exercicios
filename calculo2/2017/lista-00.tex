\documentclass[a4paper,5pt]{amsbook}
%%%%%%%%%%%%%%%%%%%%%%%%%%%%%%%%%%%%%%%%%%%%%%%%%%%%%%%%%%%%%%%%%%%%%

\usepackage{booktabs}
\usepackage{graphicx}
\usepackage{multicol}
\usepackage{textcomp}
\usepackage{systeme}
\usepackage{amssymb}
\usepackage[]{amsmath}
\usepackage{subcaption}
\usepackage[inline]{enumitem}

%%%%%%%%%%%%%%%%%%%%%%%%%%%%%%%%%%%%%%%%%%%%%%%%%%%%%%%%%%%%%%

\newcommand{\sen}{\,\mbox{sen}\,}
\newcommand{\tg}{\,\mbox{tg}\,}
\newcommand{\cosec}{\,\mbox{cosec}\,}
\newcommand{\cotg}{\,\mbox{cotg}\,}
\newcommand{\tr}{\,\mbox{tr}\,}
\newcommand{\ds}{\displaystyle}

%%%%%%%%%%%%%%%%%%%%%%%%%%%%%%%%%%%%%%%%%%%%%%%%%%%%%%%%%%%%%%%%%%%%%%%%

\setlength{\textwidth}{16cm} %\setlength{\topmargin}{-1.3cm}
\setlength{\textheight}{30cm}
\setlength{\leftmargin}{1.2cm} \setlength{\rightmargin}{1.2cm}
\setlength{\oddsidemargin}{0cm}\setlength{\evensidemargin}{0cm}

%%%%%%%%%%%%%%%%%%%%%%%%%%%%%%%%%%%%%%%%%%%%%%%%%%%%%%%%%%%%%%%%%%%%%%%%

% \renewcommand{\baselinestretch}{1.6}
% \renewcommand{\thefootnote}{\fnsymbol{footnote}}
% \renewcommand{\theequation}{\thesection.\arabic{equation}}
% \setlength{\voffset}{-50pt}
% \numberwithin{equation}{chapter}

%%%%%%%%%%%%%%%%%%%%%%%%%%%%%%%%%%%%%%%%%%%%%%%%%%%%%%%%%%%%%%%%%%%%%%%

\begin{document}
\thispagestyle{empty}
\pagestyle{empty}
\begin{minipage}[h]{0.14\textwidth}
	\includegraphics[scale=0.24]{../../ufgd.png}
\end{minipage}
\begin{minipage}[h]{\textwidth}
\begin{tabular}{c}
{{\bf UNIVERSIDADE FEDERAL DA GRANDE DOURADOS}}\\
{{\bf C\'alculo Diferencial e Integral II --- Lista 0}}\\
{{\bf Prof.\ Adriano Barbosa}}\\
\end{tabular}
\vspace{-0.45cm}
%
\end{minipage}

%------------------------

\vspace{1cm}
%%%%%%%%%%%%%%%%%%%%%%%%%%%%%%%%   formulario  inicio  %%%%%%%%%%%%%%%%%%%%%%%%%%%%%%%%
\begin{enumerate}
	\vspace{0.5cm}
	\item Encontre a antiderivada mais geral para cada fun\c{c}\~ao abaixo:

		\hspace{-0.75cm}
		\begin{enumerate*}
			\item $f(x) = x-3$
			\hspace{0.2cm}
			\hspace{0.2cm}
			\item $f(x) = \ds\frac{1}{2}+\frac{3}{4}x^2-\frac{4}{5}x^3$
			\hspace{0.2cm}
			\hspace{0.2cm}
			\item $f(x) = (x+1)(2x-1)$
			\hspace{0.2cm}
			\hspace{0.2cm}
			\item $f(x) = \ds7x^\frac{2}{5} + 8x^{-\frac{4}{5}}$
			\\	
			\item $f(x) = \sqrt{2}$
			\hspace{0.2cm}
			\hspace{0.2cm}
			\item $f(x) = \ds\frac{10}{x^9}$
			\hspace{0.2cm}
			\hspace{0.2cm}
			\item $f(x) = \ds\frac{1+t+t^2}{\sqrt{t}}$
		\end{enumerate*}

	\vspace{0.5cm}
	\item Encontre a fun\c{c}\~ao $f$:

		\hspace{-0.8cm}
		\begin{enumerate*}
			\item $f''(x) = 20x^3-12x^2+6x$
			\hspace{0.2cm}
			\hspace{0.2cm}
			\item $f''(x) = x^6 - 3x^4 + x + 1$
			\hspace{0.2cm}
			\hspace{0.2cm}
			\item $f''(x) = \ds\frac{2}{3}x^\frac{2}{3}$
			\hspace{0.2cm}
			\hspace{0.2cm}
			\item $f'''(x) = \cos(t)$
			\hspace{0.2cm}
			\hspace{0.2cm}
			\item $f''(x) = 6x+\sen(x)$
			\hspace{0.2cm}
			\hspace{0.2cm}
			\item $f'''(x) = x-\sqrt{x}$
		\end{enumerate*}

	\vspace{0.5cm}
	\item Calcule as integrais definidas:
	
		\hspace{-0.7cm}
		\begin{enumerate*}
			\item $\ds\int_{-1}^2 x^3-2x\ dx$
			\hspace{0.2cm}
			\hspace{0.2cm}
			\item $\ds\int_1^4 5-2t+3t^2\ dt$
			\hspace{0.2cm}
			\hspace{0.2cm}
			\item $\ds\int_1^9 \sqrt{x}\ dx$
			\hspace{0.2cm}
			\hspace{0.2cm}
			\item $\ds\int_\frac{\pi}{6}^\pi \sen(\theta)\ d\theta$
			\hspace{0.2cm}
			\hspace{0.2cm}
			\item $\ds\int_0^1 (u+2)(u-3)\ du$
			\hspace{0.2cm}
			\hspace{0.2cm}
			\item $\ds\int_1^9 \frac{x-1}{\sqrt{x}}\ dx$
			\hspace{0.2cm}
			\hspace{0.2cm}
			\item $\ds\int_0^\frac{\pi}{4} \sec^2(t)\ dt$
			\hspace{0.2cm}
			\hspace{0.2cm}
			\item $\ds\int_1^2 (1+2y)^2\ dy$
		\end{enumerate*}

	\vspace{0.5cm}
	\item O que est\'a errado na solu\c{c}\~ao abaixo?
		\[\ds\int_{-2}^{1} x^{-4}\ dx = \left.\frac{x^{-3}}{-3}\right|_{-2}^1 = -\frac{3}{8}\]
\end{enumerate}

\end{document}
