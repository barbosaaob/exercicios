\documentclass[a4paper,5pt]{amsbook}
%%%%%%%%%%%%%%%%%%%%%%%%%%%%%%%%%%%%%%%%%%%%%%%%%%%%%%%%%%%%%%%%%%%%%

\usepackage{booktabs}
\usepackage{graphicx}
\usepackage{multicol}
\usepackage{textcomp}
\usepackage{systeme}
\usepackage{amssymb}
\usepackage{amsmath}
\usepackage{subcaption}
\usepackage[inline]{enumitem}
\usepackage[portuguese]{babel}

%%%%%%%%%%%%%%%%%%%%%%%%%%%%%%%%%%%%%%%%%%%%%%%%%%%%%%%%%%%%%%

\newcommand{\sen}{\,\mbox{sen}\,}
\newcommand{\tg}{\,\mbox{tg}\,}
\newcommand{\cosec}{\,\mbox{cosec}\,}
\newcommand{\cotg}{\,\mbox{cotg}\,}
\newcommand{\tr}{\,\mbox{tr}\,}
\newcommand{\ds}{\displaystyle}

%%%%%%%%%%%%%%%%%%%%%%%%%%%%%%%%%%%%%%%%%%%%%%%%%%%%%%%%%%%%%%%%%%%%%%%%

\setlength{\textwidth}{16cm} %\setlength{\topmargin}{-1.3cm}
\setlength{\textheight}{26cm}
\setlength{\leftmargin}{1.2cm} \setlength{\rightmargin}{1.2cm}
\setlength{\oddsidemargin}{0cm}\setlength{\evensidemargin}{0cm}
\setlength{\topmargin}{-1cm}

%%%%%%%%%%%%%%%%%%%%%%%%%%%%%%%%%%%%%%%%%%%%%%%%%%%%%%%%%%%%%%%%%%%%%%%%

% \renewcommand{\baselinestretch}{1.6}
% \renewcommand{\thefootnote}{\fnsymbol{footnote}}
% \renewcommand{\theequation}{\thesection.\arabic{equation}}
% \setlength{\voffset}{-50pt}
% \numberwithin{equation}{chapter}

%%%%%%%%%%%%%%%%%%%%%%%%%%%%%%%%%%%%%%%%%%%%%%%%%%%%%%%%%%%%%%%%%%%%%%%

\begin{document}
\thispagestyle{empty}
\pagestyle{empty}
\begin{minipage}[h]{0.14\textwidth}
	\includegraphics[scale=0.24]{../../ufgd.png}
\end{minipage}
\begin{minipage}[h]{\textwidth}
\begin{tabular}{c}
{{\bf UNIVERSIDADE FEDERAL DA GRANDE DOURADOS}}\\
{{\bf Geometria --- Lista 4}}\\
{{\bf Prof.\ Adriano Barbosa}}\\
\end{tabular}
\vspace{-0.45cm}
%
\end{minipage}

%------------------------

\vspace{1cm}
%%%%%%%%%%%%%%%%%%%%%%%%%%%%%%%%   formulario  inicio  %%%%%%%%%%%%%%%%%%%%%%%%%%%%%%%%
\begin{enumerate}
    \item Um cubo de 20cm de altura, apoiado em um piso horizontal e com parte
        superior aberta, cont\'em \'agua at\'e a altura de 15cm. Colocando uma
        pir\^amide regular s\'olida de base quadrada e altura 30cm com a base
        apoiada no fundo do cubo, o n\'{\i}vel da \'agua atinge a altura m\'axima do
        cubo sem derramar.
        \begin{enumerate}
            \vspace{0.3cm}
            \item Qual o volume do tronco da pir\^amide submersa?
            \vspace{0.3cm}
            \item Qual o volume da pir\^amide?
        \end{enumerate}

    \vspace{0.5cm}
    \item Dados tr\^es pontos $A$, $B$ e $C$ n\~ao colineares, fa\c{c}a o que se pede
        tendo em vista que este \'e um problema de Geometria Plana. Considere
        conhecidas as constru\c{c}\~oes, com r\'egua e compasso, da mediatriz de um
        segmento e da paralela a um segmento passando por um ponto dado.
        \begin{enumerate}
            \vspace{0.3cm}
            \item Descreve os passos de constru\c{c}\~ao necess\'arios para obter,
                utilizando r\'egua e compasso, duas retas distintas $r$ e $s$ que
                cont\^em $C$ e tais que $r$ e $s$ equidistam de $A$ e $B$.
            \vspace{0.3cm}
            \item Justifique a constru\c{c}\~ao do item anterior.
        \end{enumerate}

    \vspace{0.5cm}
    \item Duas esferas de raios $r$ e $R$, com $r<R$, s\~ao tangentes interiores,
        isto \'e, possuem apenas um ponto em comum e o centro da esfera de raio
        $r$ est\'a no interior da esfera de raio $R$.
        \begin{enumerate}
            \vspace{0.3cm}
            \item Prove que o ponto de interse\c{c}\~ao das duas esferas \'e colinear
                aos centros destas esferas.
            \vspace{0.3cm}
            \item Sabe-se que o centro da esfera menor \'e o ponto m\'edio de uma
                das arestas de um tetraedro regular inscrito na esfera maior.
                Calcule $r$ em fun\c{c}\~ao de $R$.

                Dica: $R =
                \ds\frac{a\sqrt{6}}{4}$, onde $a$ \'e a aresta do tetraedro.
        \end{enumerate}

    \vspace{0.5cm}
    \item Dados dois segmentos de comprimentos $s$ e $q$, com $s>2q$, indique a
        constru\c{c}\~ao, com r\'egua e compasso, de segmentos cujos comprimentos sejam
        iguais \`as ra\'{\i}zes da equa\c{c}\~ao do segundo grau $x^2-sx+q^2 = 0$.

    \vspace{0.5cm}
    \item A altura $CH$ e a mediana $BK$ s\~ao tra\c{c}adas em um tri\^angulo
        acut\^angulo $ABC$. Sabendo que $BK \equiv CH$ e $K\hat{B}C = H\hat{C}B$,
        prove que o tri\^angulo $ABC$ \'e equil\'atero.
\end{enumerate}

\end{document}
