\documentclass[a4paper,5pt]{amsbook}
%%%%%%%%%%%%%%%%%%%%%%%%%%%%%%%%%%%%%%%%%%%%%%%%%%%%%%%%%%%%%%%%%%%%%

%\usepackage{booktabs}
\usepackage{graphicx}
%\usepackage{multicol}
%\usepackage{textcomp}
%\usepackage{systeme}
%\usepackage{amssymb}
%\usepackage[]{amsmath}
%\usepackage{subcaption}
\usepackage[inline]{enumitem}
\usepackage{gensymb}

%%%%%%%%%%%%%%%%%%%%%%%%%%%%%%%%%%%%%%%%%%%%%%%%%%%%%%%%%%%%%%

\newcommand{\sen}{\,\mbox{sen}\,}
\newcommand{\tg}{\,\mbox{tg}\,}
\newcommand{\cosec}{\,\mbox{cosec}\,}
\newcommand{\cotg}{\,\mbox{cotg}\,}
\newcommand{\tr}{\,\mbox{tr}\,}
\newcommand{\ds}{\displaystyle}
\newcommand{\ra}{\rightarrow}
\newcommand{\lra}{\leftrightarrow}
\newcommand{\Ra}{\Rightarrow}
\newcommand{\LRa}{\Leftrightarrow}
\renewcommand{\lnot}{\sim}
\newcommand{\larg}{\vdash}

%%%%%%%%%%%%%%%%%%%%%%%%%%%%%%%%%%%%%%%%%%%%%%%%%%%%%%%%%%%%%%%%%%%%%%%%

\setlength{\textwidth}{16cm} \setlength{\topmargin}{-1.7cm}
\setlength{\textheight}{25cm}
\setlength{\leftmargin}{1.2cm} \setlength{\rightmargin}{1.2cm}
\setlength{\oddsidemargin}{0cm}\setlength{\evensidemargin}{0cm}

%%%%%%%%%%%%%%%%%%%%%%%%%%%%%%%%%%%%%%%%%%%%%%%%%%%%%%%%%%%%%%%%%%%%%%%%

% \renewcommand{\baselinestretch}{1.6}
% \renewcommand{\thefootnote}{\fnsymbol{footnote}}
% \renewcommand{\theequation}{\thesection.\arabic{equation}}
% \setlength{\voffset}{-50pt}
% \numberwithin{equation}{chapter}

%%%%%%%%%%%%%%%%%%%%%%%%%%%%%%%%%%%%%%%%%%%%%%%%%%%%%%%%%%%%%%%%%%%%%%%

\begin{document}
\thispagestyle{empty}
\pagestyle{empty}
\begin{minipage}[h]{0.14\textwidth}
	\includegraphics[scale=0.24]{../ufgd.png}
\end{minipage}
\begin{minipage}[h]{\textwidth}
\begin{tabular}{c}
{{\bf UNIVERSIDADE FEDERAL DA GRANDE DOURADOS}}\\
{{\bf \'Algebra Elementar --- Lista 7}}\\
{{\bf Prof.\ Adriano Barbosa}}\\
\end{tabular}
\vspace{-0.45cm}
%
\end{minipage}

%------------------------

\vspace{1cm}
%%%%%%%%%%%%%%%%%%%%%%%%%   formulario  inicio  %%%%%%%%%%%%%%%%%%%%%%%%%%%
\begin{enumerate}
    \vspace{0.5cm}
    \item Prove que:
        \begin{enumerate}
            \item O n\'umero natural $n$ \'e par se, e somente se, o n\'umero natural
            $n+1$ \'e \'{\i}mpar.
            \item Para todo inteiro $m$, o inteiro $m^3$ \'e \'{\i}mpar se, e somente
            se, $m$ \'e inteiro \'{\i}mpar.
        \end{enumerate}

    \vspace{0.5cm}
    \item Demonstre as afirma\c{c}\~oes abaixo ou d\^e um contraexemplo:
        \begin{enumerate}
            \item Todo tri\^angulo is\'osceles \'e um tri\^angulos equil\'atero.
            \item Todo tri\^angulo equil\'atero \'e um tri\^angulos is\'osceles.
            \item Existe um tri\^angulo is\'osceles que n\~ao \'e um tri\^angulo reto.
            \item Nenhum tri\^angulo reto \'e um tri\^angulo is\'osceles.
            \item Nenhum tri\^angulo equil\'atero \'e um tri\^angulo reto.
            \item Sejam $a,b,c\in\mathbb{N}$. Se $a$ \'e \'{\i}mpar e $a+b=c$, ent\~ao
            $b$ \'e par e $c$ \'e \'{\i}mpar.
            \item Sejam $a,b\in\mathbb{N}$. Se $a$ divide $b$ e $b$ divide
            $a$, ent\~ao $a=b$.
            \item Sejam $a,b\in\mathbb{Z}$. Se $a$ divide $b$ e $b$ divide
            $a$, ent\~ao $a=b$.
        \end{enumerate}

    \vspace{0.5cm}
    \item Mostre que: [Dica: releia os teoremas mostraos em aula e os
    exerc\'{\i}cios da lista 6]
        \begin{enumerate}
            \item Todo inteiro par \'e soma de dois inteiros \'{\i}mpares.
            \item A soma de quatro inteiros consecutivos \'e par.
            \item Para todo inteiro $n$, $n^2+n$ \'e par.
            \item Para todo par de inteiros \'{\i}mparres $m$ e $n$, tem-se que $2$
            divide $m^2+3n^2$.
        \end{enumerate}
\end{enumerate}

\end{document}
