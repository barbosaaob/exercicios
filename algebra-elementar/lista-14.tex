\documentclass[a4paper,5pt]{amsbook}
%%%%%%%%%%%%%%%%%%%%%%%%%%%%%%%%%%%%%%%%%%%%%%%%%%%%%%%%%%%%%%%%%%%%%

%\usepackage{booktabs}
\usepackage{graphicx}
%\usepackage{multicol}
%\usepackage{textcomp}
%\usepackage{systeme}
\usepackage{amssymb}
%\usepackage[]{amsmath}
%\usepackage{subcaption}
\usepackage[inline]{enumitem}
\usepackage{gensymb}
\usepackage{tasks}

%%%%%%%%%%%%%%%%%%%%%%%%%%%%%%%%%%%%%%%%%%%%%%%%%%%%%%%%%%%%%%

\newcommand{\sen}{\,\mbox{sen}\,}
\newcommand{\tg}{\,\mbox{tg}\,}
\newcommand{\cosec}{\,\mbox{cosec}\,}
\newcommand{\cotg}{\,\mbox{cotg}\,}
\newcommand{\tr}{\,\mbox{tr}\,}
\newcommand{\ds}{\displaystyle}
\newcommand{\ra}{\rightarrow}
\newcommand{\lra}{\leftrightarrow}
\newcommand{\Ra}{\Rightarrow}
\newcommand{\LRa}{\Leftrightarrow}
\renewcommand{\lnot}{\sim}
\newcommand{\larg}{\vdash}

%%%%%%%%%%%%%%%%%%%%%%%%%%%%%%%%%%%%%%%%%%%%%%%%%%%%%%%%%%%%%%%%%%%%%%%%

\setlength{\textwidth}{16cm} \setlength{\topmargin}{-1.7cm}
\setlength{\textheight}{25cm}
\setlength{\leftmargin}{1.2cm} \setlength{\rightmargin}{1.2cm}
\setlength{\oddsidemargin}{0cm}\setlength{\evensidemargin}{0cm}

%%%%%%%%%%%%%%%%%%%%%%%%%%%%%%%%%%%%%%%%%%%%%%%%%%%%%%%%%%%%%%%%%%%%%%%%

% \renewcommand{\baselinestretch}{1.6}
% \renewcommand{\thefootnote}{\fnsymbol{footnote}}
% \renewcommand{\theequation}{\thesection.\arabic{equation}}
% \setlength{\voffset}{-50pt}
% \numberwithin{equation}{chapter}

%%%%%%%%%%%%%%%%%%%%%%%%%%%%%%%%%%%%%%%%%%%%%%%%%%%%%%%%%%%%%%%%%%%%%%%

\begin{document}
\thispagestyle{empty}
\pagestyle{empty}
\begin{minipage}[h]{0.14\textwidth}
	\includegraphics[scale=0.24]{../ufgd.png}
\end{minipage}
\begin{minipage}[h]{\textwidth}
\begin{tabular}{c}
{{\bf UNIVERSIDADE FEDERAL DA GRANDE DOURADOS}}\\
{{\bf \'Algebra Elementar --- Lista 14}}\\
{{\bf Prof.\ Adriano Barbosa}}\\
\end{tabular}
\vspace{-0.45cm}
%
\end{minipage}

%------------------------

\vspace{1cm}
%%%%%%%%%%%%%%%%%%%%%%%%%   formulario  inicio  %%%%%%%%%%%%%%%%%%%%%%%%%%%
\begin{enumerate}
    \setlength\itemsep{0.5cm}
    \item Dados os conjuntos $A=\{1,2\}$ e $B=\{4,5\}$, quais s\~ao todas as
    fun\c{c}\~oes que t\^em $A$ como dom\'{\i}nio e $B$ como contradom\'{\i}nio?

    \item Seja $X$ o conjunto dos tri\^angulos no plano. Se, para cada $x\in X$,
    fizermos corresponder o n\'umero real $f(x)=$per\'{\i}metro de $x$, podemos
    afirmar que $f:X\ra \mathbb{R}$ \'e uma fun\c{c}\~ao?

    \item Sejam $X$ o conjunto dos n\'umeros reais positivos e $Y$ o conjunto dos
    tri\^angulos do plano. Podemos definir, para cada $x\in X$, a fun\c{c}\~ao $f:X\ra
    Y$, $f(x)=y$, onde $y\in Y$ \'e um tri\^angulo cuja \'area \'e $x$?

    \item Mostre que:
        \begin{enumerate}
            \setlength\itemsep{0.2cm}
            \item Se $x>1$, ent\~ao $x^2>x$.
            \item Se $0<x<1$, ent\~ao $x^2<1$.
            \item Se $xy>0$, ent\~ao $x$ e $y$ s\~ao ambos positivos ou $x$ e $y$
            s\~ao ambos negativos.
        \end{enumerate}

    \item Determine, se poss\'{\i}vel, o menor e o maior elemento de cada um dos
    conjuntos:
        \begin{enumerate}
            \setlength\itemsep{0.2cm}
            \item $\{e,\sqrt{2}\}$
            \item $[-7, \infty)$
            \item $(-3,5)$
            \item $(-4,4)\cup\{5\}$
            \item $[-4,0)\cup(0,3)$
        \end{enumerate}
\end{enumerate}

\end{document}
