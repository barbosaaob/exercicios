\documentclass[a4paper,5pt]{amsbook}
%%%%%%%%%%%%%%%%%%%%%%%%%%%%%%%%%%%%%%%%%%%%%%%%%%%%%%%%%%%%%%%%%%%%%

%\usepackage{booktabs}
\usepackage{graphicx}
%\usepackage{multicol}
%\usepackage{textcomp}
%\usepackage{systeme}
\usepackage{amssymb}
%\usepackage[]{amsmath}
%\usepackage{subcaption}
\usepackage[inline]{enumitem}
\usepackage{gensymb}
\usepackage{tasks}

%%%%%%%%%%%%%%%%%%%%%%%%%%%%%%%%%%%%%%%%%%%%%%%%%%%%%%%%%%%%%%

\newcommand{\sen}{\,\mbox{sen}\,}
\newcommand{\tg}{\,\mbox{tg}\,}
\newcommand{\cosec}{\,\mbox{cosec}\,}
\newcommand{\cotg}{\,\mbox{cotg}\,}
\newcommand{\tr}{\,\mbox{tr}\,}
\newcommand{\ds}{\displaystyle}
\newcommand{\ra}{\rightarrow}
\newcommand{\lra}{\leftrightarrow}
\newcommand{\Ra}{\Rightarrow}
\newcommand{\LRa}{\Leftrightarrow}
\renewcommand{\lnot}{\sim}
\newcommand{\larg}{\vdash}

%%%%%%%%%%%%%%%%%%%%%%%%%%%%%%%%%%%%%%%%%%%%%%%%%%%%%%%%%%%%%%%%%%%%%%%%

\setlength{\textwidth}{16cm} \setlength{\topmargin}{-1.7cm}
\setlength{\textheight}{25cm}
\setlength{\leftmargin}{1.2cm} \setlength{\rightmargin}{1.2cm}
\setlength{\oddsidemargin}{0cm}\setlength{\evensidemargin}{0cm}

%%%%%%%%%%%%%%%%%%%%%%%%%%%%%%%%%%%%%%%%%%%%%%%%%%%%%%%%%%%%%%%%%%%%%%%%

% \renewcommand{\baselinestretch}{1.6}
% \renewcommand{\thefootnote}{\fnsymbol{footnote}}
% \renewcommand{\theequation}{\thesection.\arabic{equation}}
% \setlength{\voffset}{-50pt}
% \numberwithin{equation}{chapter}

%%%%%%%%%%%%%%%%%%%%%%%%%%%%%%%%%%%%%%%%%%%%%%%%%%%%%%%%%%%%%%%%%%%%%%%

\begin{document}
\thispagestyle{empty}
\pagestyle{empty}
\begin{minipage}[h]{0.14\textwidth}
	\includegraphics[scale=0.24]{../ufgd.png}
\end{minipage}
\begin{minipage}[h]{\textwidth}
\begin{tabular}{c}
{{\bf UNIVERSIDADE FEDERAL DA GRANDE DOURADOS}}\\
{{\bf \'Algebra Elementar --- Lista 8}}\\
{{\bf Prof.\ Adriano Barbosa}}\\
\end{tabular}
\vspace{-0.45cm}
%
\end{minipage}

%------------------------

\vspace{1cm}
%%%%%%%%%%%%%%%%%%%%%%%%%   formulario  inicio  %%%%%%%%%%%%%%%%%%%%%%%%%%%
\begin{enumerate}
    \vspace{0.5cm}
    \item Dados $A=\{1,2\}$ e $B=\{1,2,3\}$. Determine se as afirma\c{c}\~oes abaixo
        s\~ao verdadeiras ou falsas:
        \begin{tasks}[style=enumerate, counter-format={(tsk[a])}, label-offset={0.25cm}](4)
            \task $1\in A$
            \task $1\subset A$
            \task $\{1\}\in A$
            \task $\{1\}\subset A$
            \task $\varnothing \in A$
            \task $\varnothing \subset A$
            \task $A\subset B$
            \task $B\subset A$
            \task $A=B$
            \task $\varnothing\in\{\varnothing, A\}$
            \task $\varnothing\subset\{\varnothing, A\}$
            \task $\{\varnothing\}\in\{\varnothing, A\}$
            \task $\{\varnothing\}\subset\{\varnothing, A\}$
        \end{tasks}

    \vspace{0.5cm}
    \item Sejam $A=\{x\in\mathbb{R}\ |\ x^2-1=0\}$ e $B=\{n\in\mathbb{Z}\ |\
        |n|=1\}$. Mostre que $A=B$.

    \vspace{0.5cm}
    \item Escreva o conjunto das partes dos conjuntos abaixo.
        \begin{enumerate}
            \setlength\itemsep{0.2cm}
            \item \{Ana, Jo\~ao\}
            \item \{\$,\#,\&\}
            \item $\{a, \varnothing, \{a\}\}$
        \end{enumerate}

    \vspace{0.5cm}
    \item D\^e um exemplo de conjuntos $A$, $B$ e $C$ tais que:
        \begin{enumerate}
            \setlength\itemsep{0.2cm}
            \item $A\subset B$, $B\not\subset C$ e $A\subset C$
            \item $A\subset B$, $B\not\subset C$ e $A\not\subset C$
            \item $A\not\subset B$, $B\not\subset C$ e $A\subset C$
            \item $A\in B$, $B\not\in C$ e $A\not\in C$
            \item $A\in B$, $A\subset C$ e $B\not\subset C$
        \end{enumerate}
\end{enumerate}

\end{document}
