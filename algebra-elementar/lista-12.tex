\documentclass[a4paper,5pt]{amsbook}
%%%%%%%%%%%%%%%%%%%%%%%%%%%%%%%%%%%%%%%%%%%%%%%%%%%%%%%%%%%%%%%%%%%%%

%\usepackage{booktabs}
\usepackage{graphicx}
%\usepackage{multicol}
%\usepackage{textcomp}
%\usepackage{systeme}
\usepackage{amssymb}
%\usepackage[]{amsmath}
%\usepackage{subcaption}
\usepackage[inline]{enumitem}
\usepackage{gensymb}
\usepackage{tasks}

%%%%%%%%%%%%%%%%%%%%%%%%%%%%%%%%%%%%%%%%%%%%%%%%%%%%%%%%%%%%%%

\newcommand{\sen}{\,\mbox{sen}\,}
\newcommand{\tg}{\,\mbox{tg}\,}
\newcommand{\cosec}{\,\mbox{cosec}\,}
\newcommand{\cotg}{\,\mbox{cotg}\,}
\newcommand{\tr}{\,\mbox{tr}\,}
\newcommand{\ds}{\displaystyle}
\newcommand{\ra}{\rightarrow}
\newcommand{\lra}{\leftrightarrow}
\newcommand{\Ra}{\Rightarrow}
\newcommand{\LRa}{\Leftrightarrow}
\renewcommand{\lnot}{\sim}
\newcommand{\larg}{\vdash}

%%%%%%%%%%%%%%%%%%%%%%%%%%%%%%%%%%%%%%%%%%%%%%%%%%%%%%%%%%%%%%%%%%%%%%%%

\setlength{\textwidth}{16cm} \setlength{\topmargin}{-1.7cm}
\setlength{\textheight}{25cm}
\setlength{\leftmargin}{1.2cm} \setlength{\rightmargin}{1.2cm}
\setlength{\oddsidemargin}{0cm}\setlength{\evensidemargin}{0cm}

%%%%%%%%%%%%%%%%%%%%%%%%%%%%%%%%%%%%%%%%%%%%%%%%%%%%%%%%%%%%%%%%%%%%%%%%

% \renewcommand{\baselinestretch}{1.6}
% \renewcommand{\thefootnote}{\fnsymbol{footnote}}
% \renewcommand{\theequation}{\thesection.\arabic{equation}}
% \setlength{\voffset}{-50pt}
% \numberwithin{equation}{chapter}

%%%%%%%%%%%%%%%%%%%%%%%%%%%%%%%%%%%%%%%%%%%%%%%%%%%%%%%%%%%%%%%%%%%%%%%

\begin{document}
\thispagestyle{empty}
\pagestyle{empty}
\begin{minipage}[h]{0.14\textwidth}
	\includegraphics[scale=0.24]{../ufgd.png}
\end{minipage}
\begin{minipage}[h]{\textwidth}
\begin{tabular}{c}
{{\bf UNIVERSIDADE FEDERAL DA GRANDE DOURADOS}}\\
{{\bf \'Algebra Elementar --- Lista 12}}\\
{{\bf Prof.\ Adriano Barbosa}}\\
\end{tabular}
\vspace{-0.45cm}
%
\end{minipage}

%------------------------

\vspace{1cm}
%%%%%%%%%%%%%%%%%%%%%%%%%   formulario  inicio  %%%%%%%%%%%%%%%%%%%%%%%%%%%
\begin{enumerate}
    \setlength\itemsep{0.5cm}
    \item Escreva a representa\c{c}\~ao decimal dos n\'umeros reais abaixo:

        \noindent{}
        \begin{enumerate*}
            \item $\ds\frac{21}{12}$
            \hspace{0.3cm}
            \hspace{0.3cm}
            \item $\ds\frac{1}{3}$
            \hspace{0.3cm}
            \hspace{0.3cm}
            \item $\sqrt{3}$
            \hspace{0.3cm}
            \hspace{0.3cm}
            \item $\ds\frac{\sqrt{2}}{2}$
            \hspace{0.3cm}
            \hspace{0.3cm}
            \item $\ds\frac{2}{\sqrt{2}}$
        \end{enumerate*}

    \item Determine se os n\'umeros abaixo s\~ao racionais ou irracionais.
        \begin{enumerate}
            \setlength\itemsep{0.2cm}
            \item $\ds\frac{\sqrt{3}}{2}$
            \item $\ds\frac{\sqrt{4}}{2}$
            \item $\ds\frac{3\pi}{4\pi}$
            \item $\ds\frac{-5\sqrt{7}}{4\sqrt{7}}$
            \item $\ds\sqrt{\frac{4}{25}}$
            \item $\sqrt{24}\sqrt{6}$
            \item $(2-\sqrt{3})-(4-\sqrt{3})$
            \item $(-4+3\sqrt{3})(-4-3\sqrt{3})$
            \item $\ds\frac{2}{\sqrt{3}}\div\frac{\sqrt{3}}{4}$
        \end{enumerate}

    \item Sejam $A=\{x\ |\ 1\le x < 5\}$, $B=(3,8]$ e $C=(5,8)$.  Determine os
    conjuntos:
        \begin{enumerate}
            \setlength\itemsep{0.2cm}
            \item $A\cup B$
            \item $A\cap B$
            \item $A\cup C$
            \item $A-B$
            \item $B-A$
            \item $A-C$
        \end{enumerate}
\end{enumerate}

\end{document}
