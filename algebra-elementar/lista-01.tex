\documentclass[a4paper,5pt]{amsbook}
%%%%%%%%%%%%%%%%%%%%%%%%%%%%%%%%%%%%%%%%%%%%%%%%%%%%%%%%%%%%%%%%%%%%%

%\usepackage{booktabs}
\usepackage{graphicx}
%\usepackage{multicol}
%\usepackage{textcomp}
%\usepackage{systeme}
%\usepackage{amssymb}
%\usepackage[]{amsmath}
%\usepackage{subcaption}
\usepackage[inline]{enumitem}
\usepackage{gensymb}

%%%%%%%%%%%%%%%%%%%%%%%%%%%%%%%%%%%%%%%%%%%%%%%%%%%%%%%%%%%%%%

\newcommand{\sen}{\,\mbox{sen}\,}
\newcommand{\tg}{\,\mbox{tg}\,}
\newcommand{\cosec}{\,\mbox{cosec}\,}
\newcommand{\cotg}{\,\mbox{cotg}\,}
\newcommand{\tr}{\,\mbox{tr}\,}
\newcommand{\ds}{\displaystyle}
\newcommand{\ra}{\rightarrow}
\renewcommand{\lnot}{\sim}

%%%%%%%%%%%%%%%%%%%%%%%%%%%%%%%%%%%%%%%%%%%%%%%%%%%%%%%%%%%%%%%%%%%%%%%%

\setlength{\textwidth}{16cm} \setlength{\topmargin}{-1.7cm}
\setlength{\textheight}{25cm}
\setlength{\leftmargin}{1.2cm} \setlength{\rightmargin}{1.2cm}
\setlength{\oddsidemargin}{0cm}\setlength{\evensidemargin}{0cm}

%%%%%%%%%%%%%%%%%%%%%%%%%%%%%%%%%%%%%%%%%%%%%%%%%%%%%%%%%%%%%%%%%%%%%%%%

% \renewcommand{\baselinestretch}{1.6}
% \renewcommand{\thefootnote}{\fnsymbol{footnote}}
% \renewcommand{\theequation}{\thesection.\arabic{equation}}
% \setlength{\voffset}{-50pt}
% \numberwithin{equation}{chapter}

%%%%%%%%%%%%%%%%%%%%%%%%%%%%%%%%%%%%%%%%%%%%%%%%%%%%%%%%%%%%%%%%%%%%%%%

\begin{document}
\thispagestyle{empty}
\pagestyle{empty}
\begin{minipage}[h]{0.14\textwidth}
	\includegraphics[scale=0.24]{../ufgd.png}
\end{minipage}
\begin{minipage}[h]{\textwidth}
\begin{tabular}{c}
{{\bf UNIVERSIDADE FEDERAL DA GRANDE DOURADOS}}\\
{{\bf \'Algebra Elementar --- Lista 1}}\\
{{\bf Prof.\ Adriano Barbosa}}\\
\end{tabular}
\vspace{-0.45cm}
%
\end{minipage}

%------------------------

\vspace{1cm}
%%%%%%%%%%%%%%%%%%%%%%%%%   formulario  inicio  %%%%%%%%%%%%%%%%%%%%%%%%%%%
\begin{enumerate}
    \item Determine o valor l\'ogico das proposi\c{c}\~oes abaixo:
        \begin{enumerate}
            \item $0,131313\ldots$ \'e uma d\'{\i}zima peri\'odica simples.
            \item O hexaedro regular tem 8 arestas.
            \item Todo n\'umero divis\'{\i}vel por 5 termina com algar\'{\i}smo 5.
            \item O produto de dois n\'umeros \'{\i}mpares \'e um n\'umero \'{\i}mpar.
            \item O n\'umero 125 \'e um cubo perfeito.
            \item $0, 4, -4$ s\~ao ra\'{\i}zes da equa\c{c}\~ao $x^3-16x=0$.
            \item $\sen{\left(\frac{\pi}{2}+x\right)} = \sen{\left(\frac{\pi}{2}-x\right)}$.
        \end{enumerate}

    \vspace{0.5cm}
    \item Sejam as proposi\c{c}\~oes $p:$ Jorge \'e rico e $q:$ Carlos \'e feliz.
    Traduzir para a linguagem corrente as seguintes proposi\c{c}\~oes:
        \begin{enumerate}
            \item $\lnot p$
            \item $p\land q$
            \item $p\lor q$
            \item $(\lnot p) \land q$
            \item $p\rightarrow q$
            \item $p\leftrightarrow \lnot q$
        \end{enumerate}

    \vspace{0.5cm}
    \item Sejam as proposi\c{c}\~oes $p:$ Ana \'e rica e $q:$ Ana \'e feliz. Traduza para
    a linguagem simb\'olica as seguintes proposi\c{c}\~oes.
        \begin{enumerate}
            \item Ana \'e pobre, mas feliz.
            \item Ana \'e rica ou feliz.
            \item Ana \'e pobre e infeliz.
            \item Ana \'e pobre ou rica, mas \'e infeliz.
        \end{enumerate}

    \vspace{0.5cm}
    \item Determine o valor l\'ogico das proposi\c{c}\~oes.
        \begin{enumerate}
            \item $3+2=7$ e $5+5=10$
            \item Roma \'e a capital da Fran\c{c}a ou $\tg(45\degree)=1$.
            \item Se $3+2=6$ ent\~ao $4+4=9$.
            \item $3+4=7$ se, e somente se, $5^3=125$.
            \item $0>1 \land \sqrt{3}$ \'e irracional
            \item $3\neq 3 \lor 5\neq 5$
            \item $\pi > 4 \rightarrow 3> \sqrt{5}$
            \item $\sqrt{-1}=-1 \leftrightarrow \sqrt{-2}=-2$
        \end{enumerate}

    \vspace{0.5cm}
    \item Determine $V(p)$ sabendo que:
        \begin{enumerate}
            \item $V(q)=F$ e $V(p\land q)=F$
            \item $V(q)=F$ e $V(p\lor q)=F$
            \item $V(q)=F$ e $V(p\rightarrow q)=F$
            \item $V(q)=V$ e $V(p\leftrightarrow q)=F$
        \end{enumerate}

    \vspace{0.5cm}
    \item Determine $V(p)$ e $V(q)$ sabendo que:
        \begin{enumerate}
            \item $V(p\rightarrow q)=V$ e $V(p\land q)=F$
            \item $V(p\rightarrow q)=V$ e $V(p\lor q)=F$
            \item $V(p\leftrightarrow q)=V$ e $V(p\land q)=V$
            \item $V(p\leftrightarrow q)=F$ e $V((\lnot p)\lor q)=V$
        \end{enumerate}

\end{enumerate}

\end{document}
