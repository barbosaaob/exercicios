\documentclass[a4paper,5pt]{amsbook}
%%%%%%%%%%%%%%%%%%%%%%%%%%%%%%%%%%%%%%%%%%%%%%%%%%%%%%%%%%%%%%%%%%%%%

%\usepackage{booktabs}
\usepackage{graphicx}
%\usepackage{multicol}
%\usepackage{textcomp}
%\usepackage{systeme}
\usepackage{amssymb}
%\usepackage[]{amsmath}
%\usepackage{subcaption}
\usepackage[inline]{enumitem}
\usepackage{gensymb}
\usepackage{tasks}

%%%%%%%%%%%%%%%%%%%%%%%%%%%%%%%%%%%%%%%%%%%%%%%%%%%%%%%%%%%%%%

\newcommand{\sen}{\,\mbox{sen}\,}
\newcommand{\tg}{\,\mbox{tg}\,}
\newcommand{\cosec}{\,\mbox{cosec}\,}
\newcommand{\cotg}{\,\mbox{cotg}\,}
\newcommand{\tr}{\,\mbox{tr}\,}
\newcommand{\ds}{\displaystyle}
\newcommand{\ra}{\rightarrow}
\newcommand{\lra}{\leftrightarrow}
\newcommand{\Ra}{\Rightarrow}
\newcommand{\LRa}{\Leftrightarrow}
\renewcommand{\lnot}{\sim}
\newcommand{\larg}{\vdash}

%%%%%%%%%%%%%%%%%%%%%%%%%%%%%%%%%%%%%%%%%%%%%%%%%%%%%%%%%%%%%%%%%%%%%%%%

\setlength{\textwidth}{16cm} \setlength{\topmargin}{-1.7cm}
\setlength{\textheight}{25cm}
\setlength{\leftmargin}{1.2cm} \setlength{\rightmargin}{1.2cm}
\setlength{\oddsidemargin}{0cm}\setlength{\evensidemargin}{0cm}

%%%%%%%%%%%%%%%%%%%%%%%%%%%%%%%%%%%%%%%%%%%%%%%%%%%%%%%%%%%%%%%%%%%%%%%%

% \renewcommand{\baselinestretch}{1.6}
% \renewcommand{\thefootnote}{\fnsymbol{footnote}}
% \renewcommand{\theequation}{\thesection.\arabic{equation}}
% \setlength{\voffset}{-50pt}
% \numberwithin{equation}{chapter}

%%%%%%%%%%%%%%%%%%%%%%%%%%%%%%%%%%%%%%%%%%%%%%%%%%%%%%%%%%%%%%%%%%%%%%%

\begin{document}
\thispagestyle{empty}
\pagestyle{empty}
\begin{minipage}[h]{0.14\textwidth}
	\includegraphics[scale=0.24]{../ufgd.png}
\end{minipage}
\begin{minipage}[h]{\textwidth}
\begin{tabular}{c}
{{\bf UNIVERSIDADE FEDERAL DA GRANDE DOURADOS}}\\
{{\bf \'Algebra Elementar --- Lista 11}}\\
{{\bf Prof.\ Adriano Barbosa}}\\
\end{tabular}
\vspace{-0.45cm}
%
\end{minipage}

%------------------------

\vspace{1cm}
%%%%%%%%%%%%%%%%%%%%%%%%%   formulario  inicio  %%%%%%%%%%%%%%%%%%%%%%%%%%%
\begin{enumerate}
    \vspace{0.5cm}
    \item Para o conjunto dos n\'umeros inteiros, d\^e exemplos onde:
        \begin{enumerate}
            \item A comutatividade de subtra\c{c}\~ao $m-n=n-m$ \'e falsa.
            \item A associatividade da subtra\c{c}\~ao $m-(n-p)=(m-n)-p$ \'e falsa.
            \item A opera\c{c}\~ao de divis\~ao n\~ao est\'a bem definida.
            \item A comutatividade da divis\~ao $m\div n=n\div m$ \'e falsa.
            \item A associatividade da divis\~ao $m\div (n\div p) = (m\div n)\div
            p$ \'e falsa.
        \end{enumerate}

    \vspace{0.5cm}
    \item Use a defini\c{c}\~ao de igualdade de n\'umeros racionais para verificar se
    os pares abaixo s\~ao iguais:
        \begin{enumerate}
            \item $\ds\frac{14}{8}$, $\ds\frac{21}{12}$
            \vspace{0.3cm}
            \item $\ds\frac{6}{14}$, $\ds\frac{15}{35}$
            \vspace{0.3cm}
            \item $\ds\frac{6}{15}$, $\ds\frac{14}{35}$
        \end{enumerate}

    \vspace{0.5cm}
    \item Demonstre a propriedade de distributividade da multiplica\c{c}\~ao sobre a
    adi\c{c}\~ao para n\'umeros racionais:
        \[\ds\frac{a}{b}\cdot\left(\frac{c}{d}+\frac{e}{f}\right) =
        \frac{a}{b}\cdot\frac{c}{d} + \frac{a}{b}\cdot\frac{e}{f}, \forall\
        \frac{a}{b},\frac{c}{d},\frac{e}{f}\in\mathbb{Q}\]
\end{enumerate}

\end{document}
