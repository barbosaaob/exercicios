\documentclass[a4paper,5pt]{amsbook}
%%%%%%%%%%%%%%%%%%%%%%%%%%%%%%%%%%%%%%%%%%%%%%%%%%%%%%%%%%%%%%%%%%%%%

%\usepackage{booktabs}
\usepackage{graphicx}
%\usepackage{multicol}
%\usepackage{textcomp}
%\usepackage{systeme}
%\usepackage{amssymb}
%\usepackage[]{amsmath}
%\usepackage{subcaption}
\usepackage[inline]{enumitem}
\usepackage{gensymb}

%%%%%%%%%%%%%%%%%%%%%%%%%%%%%%%%%%%%%%%%%%%%%%%%%%%%%%%%%%%%%%

\newcommand{\sen}{\,\mbox{sen}\,}
\newcommand{\tg}{\,\mbox{tg}\,}
\newcommand{\cosec}{\,\mbox{cosec}\,}
\newcommand{\cotg}{\,\mbox{cotg}\,}
\newcommand{\tr}{\,\mbox{tr}\,}
\newcommand{\ds}{\displaystyle}
\newcommand{\ra}{\rightarrow}
\newcommand{\lra}{\leftrightarrow}
\newcommand{\Ra}{\Rightarrow}
\newcommand{\LRa}{\Leftrightarrow}
\renewcommand{\lnot}{\sim}

%%%%%%%%%%%%%%%%%%%%%%%%%%%%%%%%%%%%%%%%%%%%%%%%%%%%%%%%%%%%%%%%%%%%%%%%

\setlength{\textwidth}{16cm} \setlength{\topmargin}{-1.7cm}
\setlength{\textheight}{25cm}
\setlength{\leftmargin}{1.2cm} \setlength{\rightmargin}{1.2cm}
\setlength{\oddsidemargin}{0cm}\setlength{\evensidemargin}{0cm}

%%%%%%%%%%%%%%%%%%%%%%%%%%%%%%%%%%%%%%%%%%%%%%%%%%%%%%%%%%%%%%%%%%%%%%%%

% \renewcommand{\baselinestretch}{1.6}
% \renewcommand{\thefootnote}{\fnsymbol{footnote}}
% \renewcommand{\theequation}{\thesection.\arabic{equation}}
% \setlength{\voffset}{-50pt}
% \numberwithin{equation}{chapter}

%%%%%%%%%%%%%%%%%%%%%%%%%%%%%%%%%%%%%%%%%%%%%%%%%%%%%%%%%%%%%%%%%%%%%%%

\begin{document}
\thispagestyle{empty}
\pagestyle{empty}
\begin{minipage}[h]{0.14\textwidth}
	\includegraphics[scale=0.24]{../ufgd.png}
\end{minipage}
\begin{minipage}[h]{\textwidth}
\begin{tabular}{c}
{{\bf UNIVERSIDADE FEDERAL DA GRANDE DOURADOS}}\\
{{\bf \'Algebra Elementar --- Lista 3}}\\
{{\bf Prof.\ Adriano Barbosa}}\\
\end{tabular}
\vspace{-0.45cm}
%
\end{minipage}

%------------------------

\vspace{1cm}
%%%%%%%%%%%%%%%%%%%%%%%%%   formulario  inicio  %%%%%%%%%%%%%%%%%%%%%%%%%%%
\begin{enumerate}
    \vspace{0.5cm}
    \item Verifique a validade das Leis de De Morgan:
        \begin{enumerate}
            \item $\lnot (p\land q) \LRa \ \lnot p\ \lor \lnot q$
            \item $\lnot (p\lor q) \LRa \ \lnot p\ \land \lnot q$
            \item $p\land q \LRa \ \lnot( \lnot p\ \lor \lnot q)$
            \item $p\lor q \LRa \ \lnot(\lnot p\ \land \lnot q)$
        \end{enumerate} 

    \vspace{0.5cm}
    \item Verifique as equival\^encias: $\lnot (p\lra q) \LRa p \lra \lnot q \LRa
    \ \lnot p \lra q$.

    \vspace{0.5cm}
    \item D\^e a nega\c{c}\~ao das proprosi\c{c}\~oes abaixo:
        \begin{enumerate}
            \item Rosas s\~ao vermelhas e violetas s\~ao azuis.
            \item \'E falso que n\~ao est\'a frio ou que est\'a chovendo.
            \item N\~ao \'e verdade que Jorge estuda f\'{\i}sica, mas n\~ao qu\'{\i}mica.
        \end{enumerate}
\end{enumerate}

\end{document}
