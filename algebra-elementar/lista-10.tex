\documentclass[a4paper,5pt]{amsbook}
%%%%%%%%%%%%%%%%%%%%%%%%%%%%%%%%%%%%%%%%%%%%%%%%%%%%%%%%%%%%%%%%%%%%%

%\usepackage{booktabs}
\usepackage{graphicx}
%\usepackage{multicol}
%\usepackage{textcomp}
%\usepackage{systeme}
\usepackage{amssymb}
%\usepackage[]{amsmath}
%\usepackage{subcaption}
\usepackage[inline]{enumitem}
\usepackage{gensymb}
\usepackage{tasks}

%%%%%%%%%%%%%%%%%%%%%%%%%%%%%%%%%%%%%%%%%%%%%%%%%%%%%%%%%%%%%%

\newcommand{\sen}{\,\mbox{sen}\,}
\newcommand{\tg}{\,\mbox{tg}\,}
\newcommand{\cosec}{\,\mbox{cosec}\,}
\newcommand{\cotg}{\,\mbox{cotg}\,}
\newcommand{\tr}{\,\mbox{tr}\,}
\newcommand{\ds}{\displaystyle}
\newcommand{\ra}{\rightarrow}
\newcommand{\lra}{\leftrightarrow}
\newcommand{\Ra}{\Rightarrow}
\newcommand{\LRa}{\Leftrightarrow}
\renewcommand{\lnot}{\sim}
\newcommand{\larg}{\vdash}

%%%%%%%%%%%%%%%%%%%%%%%%%%%%%%%%%%%%%%%%%%%%%%%%%%%%%%%%%%%%%%%%%%%%%%%%

\setlength{\textwidth}{16cm} \setlength{\topmargin}{-1.7cm}
\setlength{\textheight}{25cm}
\setlength{\leftmargin}{1.2cm} \setlength{\rightmargin}{1.2cm}
\setlength{\oddsidemargin}{0cm}\setlength{\evensidemargin}{0cm}

%%%%%%%%%%%%%%%%%%%%%%%%%%%%%%%%%%%%%%%%%%%%%%%%%%%%%%%%%%%%%%%%%%%%%%%%

% \renewcommand{\baselinestretch}{1.6}
% \renewcommand{\thefootnote}{\fnsymbol{footnote}}
% \renewcommand{\theequation}{\thesection.\arabic{equation}}
% \setlength{\voffset}{-50pt}
% \numberwithin{equation}{chapter}

%%%%%%%%%%%%%%%%%%%%%%%%%%%%%%%%%%%%%%%%%%%%%%%%%%%%%%%%%%%%%%%%%%%%%%%

\begin{document}
\thispagestyle{empty}
\pagestyle{empty}
\begin{minipage}[h]{0.14\textwidth}
	\includegraphics[scale=0.24]{../ufgd.png}
\end{minipage}
\begin{minipage}[h]{\textwidth}
\begin{tabular}{c}
{{\bf UNIVERSIDADE FEDERAL DA GRANDE DOURADOS}}\\
{{\bf \'Algebra Elementar --- Lista 10}}\\
{{\bf Prof.\ Adriano Barbosa}}\\
\end{tabular}
\vspace{-0.45cm}
%
\end{minipage}

%------------------------

\vspace{1cm}
%%%%%%%%%%%%%%%%%%%%%%%%%   formulario  inicio  %%%%%%%%%%%%%%%%%%%%%%%%%%%
\begin{enumerate}
    \vspace{0.5cm}
    \item Sejam $U=\{3,5,\pi,e,i,\mu\}$, $A=\{3,\pi,i\}$, $B=\{5,\pi,e\}$ e
    $C=\{i,\mu\}$. Determine os conjuntos abaixo:
        \begin{enumerate}
            \item $A-B$
            \item $(A-B)-C$
            \item $A-(B-C)$
            \item $A-C$
            \item $A-(A\cap C)$
        \end{enumerate}

    \vspace{0.5cm}
    \item Use o princ\'{\i}pio de indu\c{c}\~ao para mostrar a validade das f\'ormulas abaixo:
        \begin{enumerate}
            \item $1+3+5+\cdots+(2n-1) = n^2$
            \item $1^2+2^2+3^2+\cdots+n^2 = \displaystyle\frac{n(n+1)(2n+1)}{6}$
            \item $1\cdot2+2\cdot3+4\cdot4+\cdots+n(n+1) = \displaystyle\frac{n(n+1)(n+2)}{3}$
        \end{enumerate}
\end{enumerate}

\end{document}
