\documentclass[a4paper,5pt]{amsbook}
%%%%%%%%%%%%%%%%%%%%%%%%%%%%%%%%%%%%%%%%%%%%%%%%%%%%%%%%%%%%%%%%%%%%%

%\usepackage{booktabs}
\usepackage{graphicx}
%\usepackage{multicol}
%\usepackage{textcomp}
%\usepackage{systeme}
\usepackage{amssymb}
%\usepackage[]{amsmath}
%\usepackage{subcaption}
\usepackage[inline]{enumitem}
\usepackage{gensymb}
\usepackage{tasks}

%%%%%%%%%%%%%%%%%%%%%%%%%%%%%%%%%%%%%%%%%%%%%%%%%%%%%%%%%%%%%%

\newcommand{\sen}{\,\mbox{sen}\,}
\newcommand{\tg}{\,\mbox{tg}\,}
\newcommand{\cosec}{\,\mbox{cosec}\,}
\newcommand{\cotg}{\,\mbox{cotg}\,}
\newcommand{\tr}{\,\mbox{tr}\,}
\newcommand{\ds}{\displaystyle}
\newcommand{\ra}{\rightarrow}
\newcommand{\lra}{\leftrightarrow}
\newcommand{\Ra}{\Rightarrow}
\newcommand{\LRa}{\Leftrightarrow}
\renewcommand{\lnot}{\sim}
\newcommand{\larg}{\vdash}

%%%%%%%%%%%%%%%%%%%%%%%%%%%%%%%%%%%%%%%%%%%%%%%%%%%%%%%%%%%%%%%%%%%%%%%%

\setlength{\textwidth}{16cm} \setlength{\topmargin}{-1.7cm}
\setlength{\textheight}{25cm}
\setlength{\leftmargin}{1.2cm} \setlength{\rightmargin}{1.2cm}
\setlength{\oddsidemargin}{0cm}\setlength{\evensidemargin}{0cm}

%%%%%%%%%%%%%%%%%%%%%%%%%%%%%%%%%%%%%%%%%%%%%%%%%%%%%%%%%%%%%%%%%%%%%%%%

% \renewcommand{\baselinestretch}{1.6}
% \renewcommand{\thefootnote}{\fnsymbol{footnote}}
% \renewcommand{\theequation}{\thesection.\arabic{equation}}
% \setlength{\voffset}{-50pt}
% \numberwithin{equation}{chapter}

%%%%%%%%%%%%%%%%%%%%%%%%%%%%%%%%%%%%%%%%%%%%%%%%%%%%%%%%%%%%%%%%%%%%%%%

\begin{document}
\thispagestyle{empty}
\pagestyle{empty}
\begin{minipage}[h]{0.14\textwidth}
	\includegraphics[scale=0.24]{../ufgd.png}
\end{minipage}
\begin{minipage}[h]{\textwidth}
\begin{tabular}{c}
{{\bf UNIVERSIDADE FEDERAL DA GRANDE DOURADOS}}\\
{{\bf \'Algebra Elementar --- Lista 13}}\\
{{\bf Prof.\ Adriano Barbosa}}\\
\end{tabular}
\vspace{-0.45cm}
%
\end{minipage}

%------------------------

\vspace{1cm}
%%%%%%%%%%%%%%%%%%%%%%%%%   formulario  inicio  %%%%%%%%%%%%%%%%%%%%%%%%%%%
\begin{enumerate}
    \setlength\itemsep{0.5cm}
    \item Dados $A=\{1,a\#\}$ e $B=\{\$,@,b\}$, determine todos os pares ordenados:
        \begin{enumerate}
            \setlength\itemsep{0.2cm}
            \item $(x,y)$ tais que $x\in A$ e $y\in B$
            \item $(x,y)$ tais que $x\in B$ e $y\in A$
            \item $(x,y)$ tais que $x\in A$ e $y\in A$
            \item $(x,y)$ tais que $x\in B$ e $y\in B$
        \end{enumerate}

    \item Dados $A=\{x\in\mathbb{R}\ |\ 1\le x \le 3\}$, $B=\{x\in\mathbb{R}\
    |\ -2\le x \le 2\}$ e $C=\{x\in\mathbb{R}\ |\ -4 < x \le 1\}$, represente
    graficamente os produtos cartesianos:
        \begin{enumerate}
            \setlength\itemsep{0.2cm}
            \item $A\times B$
            \item $B\times A$
            \item $A\times C$
            \item $B^2=B\times B$
            \item $A^2=A\times A$
        \end{enumerate}

    \item Determine os elementos de $R$ para cada rela\c{c}\~ao abaixo:
        \begin{enumerate}
            \setlength\itemsep{0.2cm}
            \item $A=\{2,3,4\}$, $B=\{2,6,12,17\}$, $R=\{(x,y)\in A\times B\ |\
            x \mbox{ divide } y\}$
            \item $A=\{1,2,3,4\}$, $R=\{(x,y)\in A\times A\ |\ y-x \mbox{ \'e um
            natural par}\}$
            \item $R=\{(x,y)\in\mathbb{R}^2\ |\ x^2+y^2=1\}$
        \end{enumerate}
\end{enumerate}

\end{document}
