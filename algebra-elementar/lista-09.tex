\documentclass[a4paper,5pt]{amsbook}
%%%%%%%%%%%%%%%%%%%%%%%%%%%%%%%%%%%%%%%%%%%%%%%%%%%%%%%%%%%%%%%%%%%%%

%\usepackage{booktabs}
\usepackage{graphicx}
%\usepackage{multicol}
%\usepackage{textcomp}
%\usepackage{systeme}
\usepackage{amssymb}
%\usepackage[]{amsmath}
%\usepackage{subcaption}
\usepackage[inline]{enumitem}
\usepackage{gensymb}
\usepackage{tasks}

%%%%%%%%%%%%%%%%%%%%%%%%%%%%%%%%%%%%%%%%%%%%%%%%%%%%%%%%%%%%%%

\newcommand{\sen}{\,\mbox{sen}\,}
\newcommand{\tg}{\,\mbox{tg}\,}
\newcommand{\cosec}{\,\mbox{cosec}\,}
\newcommand{\cotg}{\,\mbox{cotg}\,}
\newcommand{\tr}{\,\mbox{tr}\,}
\newcommand{\ds}{\displaystyle}
\newcommand{\ra}{\rightarrow}
\newcommand{\lra}{\leftrightarrow}
\newcommand{\Ra}{\Rightarrow}
\newcommand{\LRa}{\Leftrightarrow}
\renewcommand{\lnot}{\sim}
\newcommand{\larg}{\vdash}

%%%%%%%%%%%%%%%%%%%%%%%%%%%%%%%%%%%%%%%%%%%%%%%%%%%%%%%%%%%%%%%%%%%%%%%%

\setlength{\textwidth}{16cm} \setlength{\topmargin}{-1.7cm}
\setlength{\textheight}{25cm}
\setlength{\leftmargin}{1.2cm} \setlength{\rightmargin}{1.2cm}
\setlength{\oddsidemargin}{0cm}\setlength{\evensidemargin}{0cm}

%%%%%%%%%%%%%%%%%%%%%%%%%%%%%%%%%%%%%%%%%%%%%%%%%%%%%%%%%%%%%%%%%%%%%%%%

% \renewcommand{\baselinestretch}{1.6}
% \renewcommand{\thefootnote}{\fnsymbol{footnote}}
% \renewcommand{\theequation}{\thesection.\arabic{equation}}
% \setlength{\voffset}{-50pt}
% \numberwithin{equation}{chapter}

%%%%%%%%%%%%%%%%%%%%%%%%%%%%%%%%%%%%%%%%%%%%%%%%%%%%%%%%%%%%%%%%%%%%%%%

\begin{document}
\thispagestyle{empty}
\pagestyle{empty}
\begin{minipage}[h]{0.14\textwidth}
	\includegraphics[scale=0.24]{../ufgd.png}
\end{minipage}
\begin{minipage}[h]{\textwidth}
\begin{tabular}{c}
{{\bf UNIVERSIDADE FEDERAL DA GRANDE DOURADOS}}\\
{{\bf \'Algebra Elementar --- Lista 9}}\\
{{\bf Prof.\ Adriano Barbosa}}\\
\end{tabular}
\vspace{-0.45cm}
%
\end{minipage}

%------------------------

\vspace{1cm}
%%%%%%%%%%%%%%%%%%%%%%%%%   formulario  inicio  %%%%%%%%%%%%%%%%%%%%%%%%%%%
\begin{enumerate}
    \vspace{0.5cm}
    \item Observe o c\'alculo abaixo:
        \begin{eqnarray*}
                        & \sqrt{6-x}+x=0 \\
            \Rightarrow & \sqrt{6-x} = -x \\
            \Rightarrow & {(\sqrt{6-x})}^2 = (-x)^2 \\
            \Rightarrow & 6-x = x^2 \\
            \Rightarrow & x^2 + x - 6 = 0 \\
            \Rightarrow & x = 2 \mbox{ ou } x = -3 \\
            \Rightarrow & x \in \{-3, 2\}
        \end{eqnarray*}
    Podemos concluir que o conjunto solu\c{c}\~ao da equa\c{c}\~ao $\sqrt{6-x}+x=0$ \'e o
    conjunto $\{-3,2\}$? Justifique sua resposta.

    \vspace{0.5cm}
    \item Sejam $U = \{a,b,c,d,e,f\}$, $A = \{a,b,c\}$, $B = \{c,d,e\}$, $C =
    \{e,f\}$. Calcule:
        \begin{enumerate}
            \item $A^C$
            \item $B^C$
            \item $A\cup B$
            \item $A^C \cup B^C$
            \item $A \cap B$
            \item $A^C \cap B^C$
            \item $(A\cup B)^C$
            \item $(A\cap B)^C$
            \item $(A\cap B)\cup C$
            \item $A\cap (B\cup C)$
            \item $(A\cap C)\cup (A\cap C)$
            \item $(A\cup C)\cap (A\cup C)$
        \end{enumerate}

     \vspace{0.5cm}
     \item Mostre que:
        \begin{enumerate}
            %\item $A\subset A\cup B$
            %\item $A\cap B \subset A$
            \item $A\subset B \LRa A\cup B = B$
            \item $A\subset B \LRa A\cap B = A$
            \item $A\subset B \LRa B^C\subset A^C$
            \item $(A\cup B)^C = A^C \cap B^C$
            \item $(A\cap B)^C = A^C \cup B^C$
        \end{enumerate}
\end{enumerate}

\end{document}
