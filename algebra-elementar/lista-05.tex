\documentclass[a4paper,5pt]{amsbook}
%%%%%%%%%%%%%%%%%%%%%%%%%%%%%%%%%%%%%%%%%%%%%%%%%%%%%%%%%%%%%%%%%%%%%

%\usepackage{booktabs}
\usepackage{graphicx}
%\usepackage{multicol}
%\usepackage{textcomp}
%\usepackage{systeme}
%\usepackage{amssymb}
%\usepackage[]{amsmath}
%\usepackage{subcaption}
\usepackage[inline]{enumitem}
\usepackage{gensymb}

%%%%%%%%%%%%%%%%%%%%%%%%%%%%%%%%%%%%%%%%%%%%%%%%%%%%%%%%%%%%%%

\newcommand{\sen}{\,\mbox{sen}\,}
\newcommand{\tg}{\,\mbox{tg}\,}
\newcommand{\cosec}{\,\mbox{cosec}\,}
\newcommand{\cotg}{\,\mbox{cotg}\,}
\newcommand{\tr}{\,\mbox{tr}\,}
\newcommand{\ds}{\displaystyle}
\newcommand{\ra}{\rightarrow}
\newcommand{\lra}{\leftrightarrow}
\newcommand{\Ra}{\Rightarrow}
\newcommand{\LRa}{\Leftrightarrow}
\renewcommand{\lnot}{\sim}
\newcommand{\larg}{\vdash}

%%%%%%%%%%%%%%%%%%%%%%%%%%%%%%%%%%%%%%%%%%%%%%%%%%%%%%%%%%%%%%%%%%%%%%%%

\setlength{\textwidth}{16cm} \setlength{\topmargin}{-1.7cm}
\setlength{\textheight}{25cm}
\setlength{\leftmargin}{1.2cm} \setlength{\rightmargin}{1.2cm}
\setlength{\oddsidemargin}{0cm}\setlength{\evensidemargin}{0cm}

%%%%%%%%%%%%%%%%%%%%%%%%%%%%%%%%%%%%%%%%%%%%%%%%%%%%%%%%%%%%%%%%%%%%%%%%

% \renewcommand{\baselinestretch}{1.6}
% \renewcommand{\thefootnote}{\fnsymbol{footnote}}
% \renewcommand{\theequation}{\thesection.\arabic{equation}}
% \setlength{\voffset}{-50pt}
% \numberwithin{equation}{chapter}

%%%%%%%%%%%%%%%%%%%%%%%%%%%%%%%%%%%%%%%%%%%%%%%%%%%%%%%%%%%%%%%%%%%%%%%

\begin{document}
\thispagestyle{empty}
\pagestyle{empty}
\begin{minipage}[h]{0.14\textwidth}
	\includegraphics[scale=0.24]{../ufgd.png}
\end{minipage}
\begin{minipage}[h]{\textwidth}
\begin{tabular}{c}
{{\bf UNIVERSIDADE FEDERAL DA GRANDE DOURADOS}}\\
{{\bf \'Algebra Elementar --- Lista 5}}\\
{{\bf Prof.\ Adriano Barbosa}}\\
\end{tabular}
\vspace{-0.45cm}
%
\end{minipage}

%------------------------

\vspace{1cm}
%%%%%%%%%%%%%%%%%%%%%%%%%   formulario  inicio  %%%%%%%%%%%%%%%%%%%%%%%%%%%
\begin{enumerate}
    \vspace{0.5cm}
    \item Construa a tabela verdade e determine se os argumentos abaixo s\~ao v\'alidos:
        \begin{enumerate}
            \item $P \land Q, P \ra \lnot Q \larg P \land \lnot Q$
            \item $P, \lnot P \ra Q \larg \lnot Q$
            %\item $P \land Q, R \ra P \larg Q \lor R$
            %\item $\lnot P, Q \ra P \land R \larg \lnot Q$
            \item $P \land \lnot Q, \lnot Q \larg P$
            \item $P \ra Q, Q \ra R, R \larg Q$
        \end{enumerate}
    
    \vspace{0.5cm}
    \item Use o m\'etodo dedutivo para provar os argumentos abaixo.
        \begin{enumerate}
            \item $P \larg Q \ra P$
            \item $A \land B \larg A \lor B$
            \item $J \ra K, J \ra L \larg J \ra K \land L$
            \item $C \lor (D \land E), \lnot E \larg C$
        \end{enumerate}
    
    \vspace{0.5cm}
    \item Deduza uma conclus\~ao v\'alida para os argumentos abaixo.
        \begin{enumerate}
            \item Se Ana receber o e-mail (E), ela comparecer\'a (C), pois est\'a
            interessada (I).  Mesmo n\~ao comparecendo, Ana est\'a interessada.
            Portanto, (?).
            \item Se a reserva de ouro permanecer fixa (F) e o gasto de ouro
            aumentar (A), ent\~ao o pre\c{c}o do ouro ir\'a subir (S). O pre\c{c}o do ouro n\~ao
            sobe. Portando, (?).
            \item Se Jo\~ao vai ao parque (P), ent\~ao Jo\~ao veste jeans (V). Se Jo\~ao
            veste jeans, ent\~ao Jo\~ao n\~ao vai ao jantar (J) nem a balada (B). Jo\~ao
            vai a balada.  Se Jo\~ao n\~ao vai ao jantar, ent\~ao ele tem o ingresso (I),
            mas ele n\~ao tem o ingresso. Portando, (?).
        \end{enumerate}
\end{enumerate}

\end{document}
