\documentclass[a4paper,5pt]{amsbook}
%%%%%%%%%%%%%%%%%%%%%%%%%%%%%%%%%%%%%%%%%%%%%%%%%%%%%%%%%%%%%%%%%%%%%

%\usepackage{booktabs}
\usepackage{graphicx}
%\usepackage{multicol}
%\usepackage{textcomp}
%\usepackage{systeme}
%\usepackage{amssymb}
%\usepackage[]{amsmath}
%\usepackage{subcaption}
\usepackage[inline]{enumitem}
\usepackage{gensymb}

%%%%%%%%%%%%%%%%%%%%%%%%%%%%%%%%%%%%%%%%%%%%%%%%%%%%%%%%%%%%%%

\newcommand{\sen}{\,\mbox{sen}\,}
\newcommand{\tg}{\,\mbox{tg}\,}
\newcommand{\cosec}{\,\mbox{cosec}\,}
\newcommand{\cotg}{\,\mbox{cotg}\,}
\newcommand{\tr}{\,\mbox{tr}\,}
\newcommand{\ds}{\displaystyle}
\newcommand{\ra}{\rightarrow}
\newcommand{\lra}{\leftrightarrow}
\newcommand{\Ra}{\Rightarrow}
\newcommand{\LRa}{\Leftrightarrow}
\renewcommand{\lnot}{\sim}
\newcommand{\larg}{\vdash}

%%%%%%%%%%%%%%%%%%%%%%%%%%%%%%%%%%%%%%%%%%%%%%%%%%%%%%%%%%%%%%%%%%%%%%%%

\setlength{\textwidth}{16cm} \setlength{\topmargin}{-1.7cm}
\setlength{\textheight}{25cm}
\setlength{\leftmargin}{1.2cm} \setlength{\rightmargin}{1.2cm}
\setlength{\oddsidemargin}{0cm}\setlength{\evensidemargin}{0cm}

%%%%%%%%%%%%%%%%%%%%%%%%%%%%%%%%%%%%%%%%%%%%%%%%%%%%%%%%%%%%%%%%%%%%%%%%

% \renewcommand{\baselinestretch}{1.6}
% \renewcommand{\thefootnote}{\fnsymbol{footnote}}
% \renewcommand{\theequation}{\thesection.\arabic{equation}}
% \setlength{\voffset}{-50pt}
% \numberwithin{equation}{chapter}

%%%%%%%%%%%%%%%%%%%%%%%%%%%%%%%%%%%%%%%%%%%%%%%%%%%%%%%%%%%%%%%%%%%%%%%

\begin{document}
\thispagestyle{empty}
\pagestyle{empty}
\begin{minipage}[h]{0.14\textwidth}
	\includegraphics[scale=0.24]{../ufgd.png}
\end{minipage}
\begin{minipage}[h]{\textwidth}
\begin{tabular}{c}
{{\bf UNIVERSIDADE FEDERAL DA GRANDE DOURADOS}}\\
{{\bf \'Algebra Elementar --- Lista 6}}\\
{{\bf Prof.\ Adriano Barbosa}}\\
\end{tabular}
\vspace{-0.45cm}
%
\end{minipage}

%------------------------

\vspace{1cm}
%%%%%%%%%%%%%%%%%%%%%%%%%   formulario  inicio  %%%%%%%%%%%%%%%%%%%%%%%%%%%
\begin{enumerate}
    \vspace{0.5cm}
    \item Determine o conjunto verdade das seguintes senten\c{c}as abertas:
        \begin{enumerate}
            \item $x\in\mathbb{N}$, $x$ \'e um n\'umero primo par.
            \item $x\in\mathbb{Z}$, $x$ \'e um m\'utiplo de $4$.
            \item $x\in\mathbb{Z}$, $x<5$.
            \item $x\in\mathbb{Z}$, $x^2+2x-1=0$.
            \item $x\in\mathbb{R}$, $x^2+2x-1=0$.
        \end{enumerate}

%    \vspace{0.5cm}
%    \item Seja $T$ o conjunto de todos os tri\^angulos. Dadas as senten\c{c}as
%    abertas: $E(x):$ $x$ \'e um tri\^angulo equil\'atero, $I(x):$ $x$ \'e um tri\^angulo
%    is\'osceles e $R(x):$ $x$ \'e um tri\^angulo reto. Escreva simbolicamente as
%    senten\c{c}as abaixo usando um quantificador e determine seu valor l\'ogico.
%        \begin{enumerate}
%            \item Todos os tri\^angulos is\'osceles s\~ao tri\^angulos equil\'ateros.
%            \item Todos os tri\^angulos equil\'ateros s\~ao tri\^angulos is\'osceles.
%            \item Algum tri\^angulo is\'osceles n\~ao \'e um tri\^angulo reto.
%            \item Nenhum tri\^angulo reto \'e um tri\^angulo is\'osceles.
%            \item Nenhum tri\^angulo equil\'atero \'e um tri\^angulo reto.
%        \end{enumerate}

    \vspace{0.5cm}
    \item Determine se as senten\c{c}as abaixo s\~ao verdadeiras ou falsas.
    Justifique sua resposta ou d\^e um contraexemplo.
        \begin{enumerate}
            \item Existem n\'umeros naturais $m$ e $n$ tais que $m$ \'e maior que $n$.
            \item $m$ \'e maior que $n$ para todos os n\'umeros naturais $m$ e $n$.
            \item Para todo inteiro $x$ existe um inteiro $y$ tal que $x=2y$.
            \item Existe u \'unico n\'umero racional $x$ tal que $x+y=0$, para
            todos os n\'umeros racionais $y$.
            \item Para todo n\'umero real positivo $x$ existe um n\'umero natural
            $n$ tal que $\ds\frac{1}{n}<x$.
        \end{enumerate}

    \vspace{0.5cm}
    \item Apresnete uma prova direta para as proposi\c{c}\~oes abaixo:
        \begin{enumerate}
            \item Se $m$ \'e um inteiro par, ent\~ao $m+1$ \'e um inteiro \'{\i}mpar.
            \item Se $m$ e $n$ s\~ao ambos inteiros pares, ent\~ao $m+n$ \'e um
            inteiro par.
        \end{enumerate}

    \vspace{0.5cm}
    \item Apresnete uma prova pela forma contrapositiva para as proposi\c{c}\~oes abaixo:
        \begin{enumerate}
            \item Se $m$ \'e um inteiro \'{\i}mpar, ent\~ao $m+1$ \'e um inteiro par.
            \item Se $m+n$ \'e um inteiro par, ent\~ao $m$ e $n$ s\~ao ambos
            inteiros \'{\i}mpas ou ambos inteiros pares.
        \end{enumerate}

    \vspace{0.5cm}
    \item Apresnete uma por contradi\c{c}\~ao para as proposi\c{c}\~oes abaixo:
        \begin{enumerate}
            \item Se $mn$ \'e um inteiro \'{\i}mpar, ent\~ao $m$ e $n$ s\~ao ambos
            inteiros \'{\i}mpares.
            \item Sejam $m$ e $n$ n\'umeros inteiros. Se $m-n$ \'e um
            inteiro \'{\i}mpar, ent\~ao $m+n$ \'e um inteiro \'{\i}mpar.
        \end{enumerate}
\end{enumerate}

\end{document}
