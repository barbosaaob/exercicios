\documentclass[a4paper,5pt]{amsbook}
%%%%%%%%%%%%%%%%%%%%%%%%%%%%%%%%%%%%%%%%%%%%%%%%%%%%%%%%%%%%%%%%%%%%%

%\usepackage{booktabs}
\usepackage{graphicx}
%\usepackage{multicol}
%\usepackage{textcomp}
%\usepackage{systeme}
%\usepackage{amssymb}
%\usepackage[]{amsmath}
%\usepackage{subcaption}
\usepackage[inline]{enumitem}
\usepackage{gensymb}

%%%%%%%%%%%%%%%%%%%%%%%%%%%%%%%%%%%%%%%%%%%%%%%%%%%%%%%%%%%%%%

\newcommand{\sen}{\,\mbox{sen}\,}
\newcommand{\tg}{\,\mbox{tg}\,}
\newcommand{\cosec}{\,\mbox{cosec}\,}
\newcommand{\cotg}{\,\mbox{cotg}\,}
\newcommand{\tr}{\,\mbox{tr}\,}
\newcommand{\ds}{\displaystyle}
\newcommand{\ra}{\rightarrow}
\newcommand{\lra}{\leftrightarrow}
\newcommand{\Ra}{\Rightarrow}
\newcommand{\LRa}{\Leftrightarrow}
\renewcommand{\lnot}{\sim}

%%%%%%%%%%%%%%%%%%%%%%%%%%%%%%%%%%%%%%%%%%%%%%%%%%%%%%%%%%%%%%%%%%%%%%%%

\setlength{\textwidth}{16cm} \setlength{\topmargin}{-1.7cm}
\setlength{\textheight}{25cm}
\setlength{\leftmargin}{1.2cm} \setlength{\rightmargin}{1.2cm}
\setlength{\oddsidemargin}{0cm}\setlength{\evensidemargin}{0cm}

%%%%%%%%%%%%%%%%%%%%%%%%%%%%%%%%%%%%%%%%%%%%%%%%%%%%%%%%%%%%%%%%%%%%%%%%

% \renewcommand{\baselinestretch}{1.6}
% \renewcommand{\thefootnote}{\fnsymbol{footnote}}
% \renewcommand{\theequation}{\thesection.\arabic{equation}}
% \setlength{\voffset}{-50pt}
% \numberwithin{equation}{chapter}

%%%%%%%%%%%%%%%%%%%%%%%%%%%%%%%%%%%%%%%%%%%%%%%%%%%%%%%%%%%%%%%%%%%%%%%

\begin{document}
\thispagestyle{empty}
\pagestyle{empty}
\begin{minipage}[h]{0.14\textwidth}
	\includegraphics[scale=0.24]{../ufgd.png}
\end{minipage}
\begin{minipage}[h]{\textwidth}
\begin{tabular}{c}
{{\bf UNIVERSIDADE FEDERAL DA GRANDE DOURADOS}}\\
{{\bf \'Algebra Elementar --- Lista 2}}\\
{{\bf Prof.\ Adriano Barbosa}}\\
\end{tabular}
\vspace{-0.45cm}
%
\end{minipage}

%------------------------

\vspace{1cm}
%%%%%%%%%%%%%%%%%%%%%%%%%   formulario  inicio  %%%%%%%%%%%%%%%%%%%%%%%%%%%
\begin{enumerate}
    \vspace{0.5cm}
    \item Construa as tabelas verdade das proposi\c{c}\~oes abaixo:
        \begin{enumerate}
            \item $\lnot(p\lor \lnot q)$
            \item $p\land q \ra p\lor q$
            \item $(p\lra \lnot q)\lra q \ra p$
            \item $(p\lra \lnot q) \ra \lnot p \land q$
        \end{enumerate}

    \vspace{0.5cm}
    \item Determine se as proposi\c{c}\~oes s\~ao tautologias ou contradi\c{c}\~oes:
        \begin{enumerate}
            \item $(p\ra p) \lor (p\ra \lnot p)$
            \item $(p\ra q) \land \lnot q \ra \lnot p$
            \item $p\ra (p\ra q \land \lnot q)$
            \item $\lnot(p \lra q) \land p \ra q$
        \end{enumerate}

    \vspace{0.5cm}
    \item Verifique as validade das implica\c{c}\~oes abaixo:
        \begin{enumerate}
            \item Regra do silogismo disjuntivo: $(p\lor q)\land \lnot p \Ra q$
            \item Regra modus ponens: $(p\ra q)\land p \Ra q$
            \item Regra modus tollens: $(p\ra q) \lor \lnot q \Ra \lnot p$
            \item Regra do silogismos hipot\'etico: $(p\ra q)\land(q\ra r)\Ra p\ra r$
        \end{enumerate}

    \vspace{0.5cm}
    \item Verifique a validade das equival\^encias:
        \begin{enumerate}
            \item Regra de absor\c{c}\~ao: $p\ra p\land q \LRa p\ra q$
            \item $p\land(p\lor q)\LRa p$
            \item $p\lra p\land q \LRa p\ra q$
            \item $(p\ra q)\land(p\ra r)\LRa p\ra q\land r$
            \item Regra de exporta\c{c}\~ao-importa\c{c}\~ao: $p\land q \ra r\LRa p\ra(q\ra r)$
        \end{enumerate}

    \vspace{0.5cm}
    \item Escreva a rec\'{\i}proca, contr\'aria e contrapositiva das proposi\c{c}\~oes abaixo:
        \begin{enumerate}
            \item Se Jo\~ao \'e rico, ent\~ao tem um carro caro.
            \item Se chover, ent\~ao Ana n\~ao vai a praia.
            \item Se eu n\~ao for ao cinema, ent\~ao irei ao teatro.
        \end{enumerate}
\end{enumerate}

\end{document}
