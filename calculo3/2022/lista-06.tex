\documentclass[a4paper,5pt]{amsbook}
%%%%%%%%%%%%%%%%%%%%%%%%%%%%%%%%%%%%%%%%%%%%%%%%%%%%%%%%%%%%%%%%%%%%%

\usepackage{booktabs}
\usepackage{graphicx}
\usepackage{multicol}
\usepackage{textcomp}
\usepackage{systeme}
\usepackage{amssymb}
\usepackage[]{amsmath}
\usepackage{subcaption}
\usepackage[inline]{enumitem}
\usepackage{gensymb}

%%%%%%%%%%%%%%%%%%%%%%%%%%%%%%%%%%%%%%%%%%%%%%%%%%%%%%%%%%%%%%

\newcommand{\sen}{\,\mbox{sen}}
\newcommand{\tg}{\,\mbox{tg}\,}
\newcommand{\cosec}{\,\mbox{cosec}\,}
\newcommand{\cotg}{\,\mbox{cotg}\,}
\newcommand{\tr}{\,\mbox{tr}\,}
\newcommand{\ds}{\displaystyle}
\newcommand{\ra}{\rightarrow}

%%%%%%%%%%%%%%%%%%%%%%%%%%%%%%%%%%%%%%%%%%%%%%%%%%%%%%%%%%%%%%%%%%%%%%%%

\setlength{\textwidth}{16cm} %\setlength{\topmargin}{-1.3cm}
\setlength{\textheight}{25cm}
\setlength{\leftmargin}{1.2cm} \setlength{\rightmargin}{1.2cm}
\setlength{\oddsidemargin}{0cm}\setlength{\evensidemargin}{0cm}

%%%%%%%%%%%%%%%%%%%%%%%%%%%%%%%%%%%%%%%%%%%%%%%%%%%%%%%%%%%%%%%%%%%%%%%%

% \renewcommand{\baselinestretch}{1.6}
% \renewcommand{\thefootnote}{\fnsymbol{footnote}}
% \renewcommand{\theequation}{\thesection.\arabic{equation}}
% \setlength{\voffset}{-50pt}
% \numberwithin{equation}{chapter}

%%%%%%%%%%%%%%%%%%%%%%%%%%%%%%%%%%%%%%%%%%%%%%%%%%%%%%%%%%%%%%%%%%%%%%%

\begin{document}
\thispagestyle{empty}
\pagestyle{empty}
\begin{minipage}[h]{0.14\textwidth}
	\includegraphics[scale=0.24]{../../ufgd.png}
\end{minipage}
\begin{minipage}[h]{\textwidth}
\begin{tabular}{c}
{{\bf UNIVERSIDADE FEDERAL DA GRANDE DOURADOS}}\\
{{\bf C\'alculo de V\'arias Vari\'aveis --- Lista 6}}\\
{{\bf Prof.\ Adriano Barbosa}}\\
\end{tabular}
\vspace{-0.45cm}
%
\end{minipage}

%------------------------

\vspace{1cm}
%%%%%%%%%%%%%%%%%%%%%%%%%%%%%%%%   formulario  inicio  %%%%%%%%%%%%%%%%%%%%%%%%%%%%%%%%
\begin{enumerate}
    \setlength\itemsep{0.5cm}
    \item Estime o volume do s\'olido que est\'a definido abaixo da superf\'{\i}cie
    $z=xy$ e acima do ret\^angulo $R=\{(x,y)\in\mathbb{R}^2\ |\ 0 \le x \le 6, 0
    \le y \le 4\}$.  Use a soma de Riemann com $m=3, n=2$ e os pontos do canto
    superior direito.

    \item Calcule as integrais interpretando-as como volume de um s\'olido.
        \begin{enumerate}
            \setlength\itemsep{0.2cm}
            \item $\ds\iint_R 3\ dA$, $R=\{(x,y)\ |\ -2 \le x \le 2, 1 \le y \le
            6\}$
            \item $\ds\iint_R 4-2y\ dA$, $R=[0,1]\times[0,1]$
        \end{enumerate}

    \item Calcule as integrais iteradas.
        \begin{enumerate}
            \setlength\itemsep{0.2cm}
            \item $\ds\int_1^4\int_0^2 6x^2y-2x\ dydx$
            \item $\ds\int_0^2\int_0^4 y^3e^{2x}\ dydx$
            \item $\ds\int_{-3}^3\int_0^{\pi/2} y+y^2\cos{x}\ dxdy$
        \end{enumerate}

    \item Calcule as integrais duplas.
        \begin{enumerate}
            \setlength\itemsep{0.2cm}
            \item $\ds\iint_R \sen(x-y)\ dA$, $R=\{(x,y)\ |\ 0 \le x \le \pi/2,
            0 \le y \le \pi/2\}$
            \item $\ds\iint_R \frac{xy^2}{x^2+1}\ dA$, $R=\{(x,y)\ |\ 0 \le x
            \le 1, -3 \le y \le 3\}$
            \item $\ds\iint_R x\sen(x+y)\ dA$, $R=[0,\pi/6]\times[0,\pi/3]$
        \end{enumerate}
\end{enumerate}

\end{document}
