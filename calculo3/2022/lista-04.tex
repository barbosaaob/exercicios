\documentclass[a4paper,5pt]{amsbook}
%%%%%%%%%%%%%%%%%%%%%%%%%%%%%%%%%%%%%%%%%%%%%%%%%%%%%%%%%%%%%%%%%%%%%

\usepackage{booktabs}
\usepackage{graphicx}
\usepackage{multicol}
\usepackage{textcomp}
\usepackage{systeme}
\usepackage{amssymb}
\usepackage[]{amsmath}
\usepackage{subcaption}
\usepackage[inline]{enumitem}
\usepackage{gensymb}

%%%%%%%%%%%%%%%%%%%%%%%%%%%%%%%%%%%%%%%%%%%%%%%%%%%%%%%%%%%%%%

\newcommand{\sen}{\,\mbox{sen}}
\newcommand{\tg}{\,\mbox{tg}\,}
\newcommand{\cosec}{\,\mbox{cosec}\,}
\newcommand{\cotg}{\,\mbox{cotg}\,}
\newcommand{\tr}{\,\mbox{tr}\,}
\newcommand{\ds}{\displaystyle}
\newcommand{\ra}{\rightarrow}

%%%%%%%%%%%%%%%%%%%%%%%%%%%%%%%%%%%%%%%%%%%%%%%%%%%%%%%%%%%%%%%%%%%%%%%%

\setlength{\textwidth}{16cm} %\setlength{\topmargin}{-1.3cm}
\setlength{\textheight}{25cm}
\setlength{\leftmargin}{1.2cm} \setlength{\rightmargin}{1.2cm}
\setlength{\oddsidemargin}{0cm}\setlength{\evensidemargin}{0cm}

%%%%%%%%%%%%%%%%%%%%%%%%%%%%%%%%%%%%%%%%%%%%%%%%%%%%%%%%%%%%%%%%%%%%%%%%

% \renewcommand{\baselinestretch}{1.6}
% \renewcommand{\thefootnote}{\fnsymbol{footnote}}
% \renewcommand{\theequation}{\thesection.\arabic{equation}}
% \setlength{\voffset}{-50pt}
% \numberwithin{equation}{chapter}

%%%%%%%%%%%%%%%%%%%%%%%%%%%%%%%%%%%%%%%%%%%%%%%%%%%%%%%%%%%%%%%%%%%%%%%

\begin{document}
\thispagestyle{empty}
\pagestyle{empty}
\begin{minipage}[h]{0.14\textwidth}
	\includegraphics[scale=0.24]{../../ufgd.png}
\end{minipage}
\begin{minipage}[h]{\textwidth}
\begin{tabular}{c}
{{\bf UNIVERSIDADE FEDERAL DA GRANDE DOURADOS}}\\
{{\bf C\'alculo de V\'arias Vari\'aveis --- Lista 4}}\\
{{\bf Prof.\ Adriano Barbosa}}\\
\end{tabular}
\vspace{-0.45cm}
%
\end{minipage}

%------------------------

\vspace{1cm}
%%%%%%%%%%%%%%%%%%%%%%%%%%%%%%%%   formulario  inicio  %%%%%%%%%%%%%%%%%%%%%%%%%%%%%%%%
\begin{enumerate}
    \setlength\itemsep{0.5cm}
    \item Calcule a derivada direcional das fun\c{c}\~oes abaixo no ponto dado na dire\c{c}\~ao $u$.
        \begin{enumerate}
            \setlength\itemsep{0.2cm}
            \item $f(x,y) = e^x\sen{y}$, $(0,\pi/3)$, $u=(-6,8)$.
            \item $f(p,q) = p^4-p^2q^3$, $(2,1)$, $u=(1,3)$.
            \item $f(,x,y,z) = xe^y+ye^z+ze^x$, $(0,0,0)$, $u=(5,1,-2)$.
        \end{enumerate}

    \item Para cada item abaixo, calcule o gradiente de $f$, avalie o gradiente
    em $P$ e calcule a taxa de varia\c{c}\~ao de $f$ em $P$ na dire\c{c}\~ao $u$.
        \begin{enumerate}
            \setlength\itemsep{0.2cm}
            \item $f(x,y) = \sen{(2x+3y)}$, $P=(-6,4)$, $u=(\sqrt{3}/2, -1/2)$.
            \item $f(x,y,z) = x^2yz-xyz^3$, $P=(2,-1,1)$, $u=(0,4/5, -3/5)$.
        \end{enumerate}

    \item Calcule a taxa de varia\c{c}\~ao m\'axima de $f$ no ponto dado e determine a
    dire\c{c}\~ao em que ela ocorre.
        \begin{enumerate}
            \setlength\itemsep{0.2cm}
            \item $f(x,y) = 4y\sqrt{x}$, $(4,1)$.
            \item $f(x,y) = \sen{(xy)}$, $(1,0)$.
        \end{enumerate}

    \item A temperatura $T$ numa bola de metal \'e inversamente proporcional \`a
    dist\^ancoa do centroda bole, que tomamos como a origem. A temperatura no
    ponto $(1,2,2)$ \'e de $120\degree$C. Determine a taxa de varia\c{c}\~ao de $T$ em
    $(1,2,2)$ na dire\c{c}\~ao $(1,-1,1)$.
\end{enumerate}

\end{document}
