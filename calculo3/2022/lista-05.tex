\documentclass[a4paper,5pt]{amsbook}
%%%%%%%%%%%%%%%%%%%%%%%%%%%%%%%%%%%%%%%%%%%%%%%%%%%%%%%%%%%%%%%%%%%%%

\usepackage{booktabs}
\usepackage{graphicx}
\usepackage{multicol}
\usepackage{textcomp}
\usepackage{systeme}
\usepackage{amssymb}
\usepackage[]{amsmath}
\usepackage{subcaption}
\usepackage[inline]{enumitem}
\usepackage{gensymb}

%%%%%%%%%%%%%%%%%%%%%%%%%%%%%%%%%%%%%%%%%%%%%%%%%%%%%%%%%%%%%%

\newcommand{\sen}{\,\mbox{sen}}
\newcommand{\tg}{\,\mbox{tg}\,}
\newcommand{\cosec}{\,\mbox{cosec}\,}
\newcommand{\cotg}{\,\mbox{cotg}\,}
\newcommand{\tr}{\,\mbox{tr}\,}
\newcommand{\ds}{\displaystyle}
\newcommand{\ra}{\rightarrow}

%%%%%%%%%%%%%%%%%%%%%%%%%%%%%%%%%%%%%%%%%%%%%%%%%%%%%%%%%%%%%%%%%%%%%%%%

\setlength{\textwidth}{16cm} %\setlength{\topmargin}{-1.3cm}
\setlength{\textheight}{25cm}
\setlength{\leftmargin}{1.2cm} \setlength{\rightmargin}{1.2cm}
\setlength{\oddsidemargin}{0cm}\setlength{\evensidemargin}{0cm}

%%%%%%%%%%%%%%%%%%%%%%%%%%%%%%%%%%%%%%%%%%%%%%%%%%%%%%%%%%%%%%%%%%%%%%%%

% \renewcommand{\baselinestretch}{1.6}
% \renewcommand{\thefootnote}{\fnsymbol{footnote}}
% \renewcommand{\theequation}{\thesection.\arabic{equation}}
% \setlength{\voffset}{-50pt}
% \numberwithin{equation}{chapter}

%%%%%%%%%%%%%%%%%%%%%%%%%%%%%%%%%%%%%%%%%%%%%%%%%%%%%%%%%%%%%%%%%%%%%%%

\begin{document}
\thispagestyle{empty}
\pagestyle{empty}
\begin{minipage}[h]{0.14\textwidth}
	\includegraphics[scale=0.24]{../../ufgd.png}
\end{minipage}
\begin{minipage}[h]{\textwidth}
\begin{tabular}{c}
{{\bf UNIVERSIDADE FEDERAL DA GRANDE DOURADOS}}\\
{{\bf C\'alculo de V\'arias Vari\'aveis --- Lista 5}}\\
{{\bf Prof.\ Adriano Barbosa}}\\
\end{tabular}
\vspace{-0.45cm}
%
\end{minipage}

%------------------------

\vspace{1cm}
%%%%%%%%%%%%%%%%%%%%%%%%%%%%%%%%   formulario  inicio  %%%%%%%%%%%%%%%%%%%%%%%%%%%%%%%%
\begin{enumerate}
    \setlength\itemsep{0.5cm}
    \item Encontre os m\'aximos e m\'{\i}nimos locais e os pontos de sela das fun\c{c}\~oes.
        \begin{enumerate}
            \item $f(x,y) = x^2+xy+y^2+y$
            \item $f(x,y) = (x-y)(1-xy)$
            \item $f(x,y) = e^x \cos{y}$
            \item $f(x,y) = (x^2+y^2)e^{y^2-x^2}$
        \end{enumerate}

    \item Encontre a menor dist\^ancia entre o ponto $(2,0,-3)$ e o plano
    $x+y+z=1$.

    \item Encontre os tr\^es n\'umeros $a$, $b$ e $c$ tais que $a+b+c=100$ e o
    produto $abc$ seja o maior poss\'{\i}vel.

    \item Deseja-se poduzir uma caixa sem tampa com volume de 32000cm$^3$.
    Quais devem ser as dimen\c{c}\~oes da caixa de modo que a quantidade de papel\~ao
    utilizada seja a menor poss\'{\i}vel?

    \item Use o m\'etodo dos multiplicadores de Lagrange para determinar os
    valores m\'aximos e m\'{\i}nimos de cada fun\c{c}\~ao restrita a condi\c{c}\~ao dada.
        \begin{enumerate}
            \setlength\itemsep{0.5cm}
            \item $f(x,y) = x^2+y^2$ tal que $xy=1$
            \item $f(x,y) = y^2-x^2$ tal que $\ds\frac{1}{4}x^2+y^2=1$
            \item $f(x,y,z) = 2x+2y+z$ tal que $x^2+y^2+z^2=9$
            \item $f(x,y,z) = xyz$ tal que $x^2+2y^2+3z^2=6$
        \end{enumerate}

    \item Encontre os pontos do cone $z^2=x^2+y^2$ que est\~ao mais pr\'oximos do
    ponto $(4,2,0)$.

    \item Encontre as dimens\~oes de uma caixa retangular com volume 1000cm$^3$
    com menor \'area de superf\'{\i}cie poss\'{\i}vel.

    \item Encontre as dimens\~oes de uma caixa retangular com volume m\'aximo tal
    que a soma dos comprimentos de suas 12 arestas \'e uma constante $c$.
\end{enumerate}

\end{document}
