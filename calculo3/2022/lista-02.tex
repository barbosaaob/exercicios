\documentclass[a4paper,5pt]{amsbook}
%%%%%%%%%%%%%%%%%%%%%%%%%%%%%%%%%%%%%%%%%%%%%%%%%%%%%%%%%%%%%%%%%%%%%

\usepackage{booktabs}
\usepackage{graphicx}
\usepackage{multicol}
\usepackage{textcomp}
\usepackage{systeme}
\usepackage{amssymb}
\usepackage[]{amsmath}
\usepackage{subcaption}
\usepackage[inline]{enumitem}
\usepackage{gensymb}

%%%%%%%%%%%%%%%%%%%%%%%%%%%%%%%%%%%%%%%%%%%%%%%%%%%%%%%%%%%%%%

\newcommand{\sen}{\,\mbox{sen}}
\newcommand{\tg}{\,\mbox{tg}\,}
\newcommand{\cosec}{\,\mbox{cosec}\,}
\newcommand{\cotg}{\,\mbox{cotg}\,}
\newcommand{\tr}{\,\mbox{tr}\,}
\newcommand{\ds}{\displaystyle}
\newcommand{\ra}{\rightarrow}

%%%%%%%%%%%%%%%%%%%%%%%%%%%%%%%%%%%%%%%%%%%%%%%%%%%%%%%%%%%%%%%%%%%%%%%%

\setlength{\textwidth}{16cm} %\setlength{\topmargin}{-1.3cm}
\setlength{\textheight}{25cm}
\setlength{\leftmargin}{1.2cm} \setlength{\rightmargin}{1.2cm}
\setlength{\oddsidemargin}{0cm}\setlength{\evensidemargin}{0cm}

%%%%%%%%%%%%%%%%%%%%%%%%%%%%%%%%%%%%%%%%%%%%%%%%%%%%%%%%%%%%%%%%%%%%%%%%

% \renewcommand{\baselinestretch}{1.6}
% \renewcommand{\thefootnote}{\fnsymbol{footnote}}
% \renewcommand{\theequation}{\thesection.\arabic{equation}}
% \setlength{\voffset}{-50pt}
% \numberwithin{equation}{chapter}

%%%%%%%%%%%%%%%%%%%%%%%%%%%%%%%%%%%%%%%%%%%%%%%%%%%%%%%%%%%%%%%%%%%%%%%

\begin{document}
\thispagestyle{empty}
\pagestyle{empty}
\begin{minipage}[h]{0.14\textwidth}
	\includegraphics[scale=0.24]{../../ufgd.png}
\end{minipage}
\begin{minipage}[h]{\textwidth}
\begin{tabular}{c}
{{\bf UNIVERSIDADE FEDERAL DA GRANDE DOURADOS}}\\
{{\bf C\'alculo de V\'arias Vari\'aveis --- Lista 2}}\\
{{\bf Prof.\ Adriano Barbosa}}\\
\end{tabular}
\vspace{-0.45cm}
%
\end{minipage}

%------------------------

\vspace{1cm}
%%%%%%%%%%%%%%%%%%%%%%%%%%%%%%%%   formulario  inicio  %%%%%%%%%%%%%%%%%%%%%%%%%%%%%%%%
\begin{enumerate}
    \setlength\itemsep{0.5cm}
    \item Use uma tabela de valores de $f(x,y)$ para $(x,y)$ pr\'oximos da origem
    e estime o valor de $\ds\lim_{(x,y)\ra(0,0)} f(x,y)$ para as fun\c{c}\~oes
    abaixo:
        \begin{enumerate}
            \setlength\itemsep{0.2cm}
            \item $f(x,y) = \ds\frac{x^2y^3+x^3y^2-5}{2-xy}$
            \item $f(x,y) = \ds\frac{2xy}{x^2+2y^2}$
        \end{enumerate}

    \item Se $\ds\lim_{(x,y)\ra(3,1)} f(x,y) = 6$, o que podemos dizer sobre o
    valor de $f(3,1)$? E se $f$ for cont\'{\i}nua?

    \item Determine o conjunto de pontos onde as fun\c{c}\~oes s\~ao cont\'{\i}nuas:
        \begin{enumerate}
            \setlength\itemsep{0.2cm}
            \item $f(x,y) = \ds\frac{xy}{1+e^{x-y}}$
            \item $f(x,y) = \ds\frac{1+x^2+y^2}{1-x^2-y^2}$
            \item $f(x,y) = \ln{(x^2+y^2-4)}$
            \item $f(x,y) = \sqrt{y-x^2}\ln{z}$
            %\item $f(x,y) = \left\{\begin{array}{cl}
            %    \ds\frac{x^2y^3}{2x^2+y^2}, & \text{ se } (x,y)\neq(0,0) \\
            %    1, & \text{ se } (x,y)=(0,0)
            %    \end{array}\right.$
        \end{enumerate}

    \item Calcule as derivadas parciais das fun\c{c}\~oes:
        \begin{enumerate}
            \setlength\itemsep{0.2cm}
            \item $f(x,y) = y^5-3xy$
            \item $f(x,t) = e^{-t}\cos{(\pi x)}$
            \item $z = (2x+3y)^{10}$
            \item $f(x,y) = \ds\frac{x}{y}$
            \item $w = \ln{(x+2y+3z)}$
            \item $F(x,y) = \ds\int_y^x \cos{(e^t)}\ dt$
        \end{enumerate}

    \item Calcule as segundas derivadas parciais das fun\c{c}\~oes:
        \begin{enumerate}
            \setlength\itemsep{0.2cm}
            \item $f(x,y) = x^3y^5+2x^4y$
            \item $w = \sqrt{u^2+v^2}$
            \item $v = e^{xe^y}$
        \end{enumerate}

    \item Determine o sinal das derivadas parciais para a fun\c{c}\~ao cujo gr\'afico
    est\'a abaixo:
        \begin{enumerate}
            \setlength\itemsep{0.2cm}
            \item $f_x(1,2)$
            \item $f_x(-1,2)$
            \item $f_y(1,2)$
            \item $f_y(-1,2)$
        \end{enumerate}
        \begin{figure}[h]
            \centering
            \includegraphics[width=0.7\textwidth]{lista-02-fig1.png}
        \end{figure}
\end{enumerate}

\end{document}
