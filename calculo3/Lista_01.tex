\documentclass[a4paper,5pt]{amsbook}
%%%%%%%%%%%%%%%%%%%%%%%%%%%%%%%%%%%%%%%%%%%%%%%%%%%%%%%%%%%%%%%%%%%%%

\usepackage{graphics}
\usepackage[]{float}
\usepackage{amssymb}
\usepackage{amsfonts}
\usepackage[]{amsmath}
\usepackage[]{epsfig}
%\usepackage[brazil]{babel}
\usepackage[utf8]{inputenc}
\usepackage{verbatim}
%\usepackage[]{pstricks}
%\usepackage[notcite,notref]{showkeys}

%%%%%%%%%%%%%%%%%%%%%%%%%%%%%%%%%%%%%%%%%%%%%%%%%%%%%%%%%%%%%%%%%%%%%%

\newtheorem{theorem}{Teorema}[chapter]
\newtheorem{lemma}{Lema}[chapter]
\newtheorem{definition}{Defini\c{c}\~{a}o}[chapter]
\newtheorem{remark}{Observa\c{c}\~{a}o}[chapter]
\newtheorem{proposition}{Proposi\c{c}\~{a}o}[chapter]
\newtheorem{corollary}{Corolario}[chapter]
\newtheorem{example}{Exemplo}[chapter]

%%%%%%%%%%%%%%%%%%%%%%%%%%%%%%%%%%%%%%%%%%%%%%%%%%%%%%%%%%%%%%

\newcommand{\R}{\mathbb R}
\newcommand{\Q}{\mathbb Q}
\newcommand{\Z}{\mathbb Z}
\newcommand{\N}{\mathbb N}
\newcommand{\lan}{\langle}
\newcommand{\ran}{\rangle}
\newcommand{\sen}{\text{sen}}
\newcommand{\I}{\infty}
\newcommand{\vet}[1]{\ensuremath{\overrightarrow{#1}}}
%%%%%%%%%%%%%%%%%%%%%%%%%%%%%%%%%%%%%%%%%%%%%%%%%%%%%%%%%%%%%%%%%%%%%%%%
%---------------------------------------------------------
\setlength{\textwidth}{16cm} %\setlength{\topmargin}{-0.1cm}
\setlength{\leftmargin}{1.2cm} \setlength{\rightmargin}{1.2cm}
\setlength{\oddsidemargin}{0cm}\setlength{\evensidemargin}{0cm}
%---------------------------------------------------------
%%%%%%%%%%%%%%%%%%%%%%%%%%%%%%%%%%%%%%%%%%%%%%%%%%%%%%%%%%%%%%%%%%%%%%%%

%\setlength{\textwidth}{16cm}
%\addtolength{\oddsidemargin}{-1.7cm}
%\addtolength{\evensidemargin}{-1.7cm}
%\setlength{\textheight}{20cm}
%%%%%%%%%%%%%%%%%%%%%%%%%%%%%%%%%%%%%%%%%%%%%%%%%%%%%%%%%%%%%%%%%%%%%%%
%%
\renewcommand{\baselinestretch}{1.6}
\renewcommand{\thefootnote}{\fnsymbol{footnote}}
\renewcommand{\theequation}{\thesection.\arabic{equation}}
\setlength{\voffset}{-50pt}
\numberwithin{equation}{chapter}
%%
%%%%%%%%%%%%%%%%%%%%%%%%%%%%%%%%%%%%%%%%%%%%%%%%%%%%%%%%%%%%%%%%%%%%%%%


%%%%%%%%%%%%%%%%%%%%%%%%%%%%%%%%%%%%%%%%%%%%%%%%%%%%%%%%%%%%%%%%%%%%%%%


%%%%%%%%%%%%%%%%%%%%%%%%%%%%%%%%%%%%%%%%%%%%%%%%%%%%%%%%%%%%%%%%%%%%%%%
\begin{document}
\thispagestyle{empty}
\begin{minipage}[b]{0.45\linewidth}
\begin{tabular}{c}
\hline \hline
{{\bf UNIVERSIDADE FEDERAL DA GRANDE DOURADOS}}\\

{{\bf FACET}} \\

\hline
Lista 01\hspace{12cm}  12/09/2016  \\
\hline \hline
\end{tabular}
%
\end{minipage} 

\vspace{0.2cm}
%%%%%%%%%%%%%%%%%%%%%%%%%%%%%%%%   formulario  in\'icio  %%%%%%%%%%%%%%%%%%%%%%%%%%%%%%%%

\begin{enumerate}

 \item Calcule as integrais duplas:
 \begin{itemize}
 
 \item[a)]$\displaystyle\iint_R xe^{xy}dxdy$ onde $R=[1,3]\times[0,1].$\\
 \item[b)]$\displaystyle\iint_R y\ln{x}dxdy$ onde $R=[2,3]\times[1,2].$\\
 \item[c)]$\displaystyle\iint_R \dfrac{x}{1+xy} dxdy$ onde $R=[0,1]\times[0,1].$\\
 \item[d)]$\displaystyle\int_{0}^{2}\int_{0}^{\pi}r\sen^2{\theta}d\theta dr.$\\
 \item[e)]$\displaystyle\int_{0}^{\ln{2}}\int_{0}^{1}xye^{y^2x}dydx.$\\
 \item[f)]$\displaystyle\int_{\pi/2}^{\pi}\int_{1}^{2}x\cos{(xy)}dydx.$\\
 \item[g)]$\displaystyle\iint_R (2x+y)dxdy$ onde $R$ \'e a regi\~ao delimitada por $x=y^2-1$, $x=5$, $y=-1$ e $y=2$.\\
 \item[h)]$\displaystyle\iint_R xdxdy$ onde $R$ \'e a regi\~ao delimitada por $y=-x$, $y=4x$ e $y=\dfrac{3x}{2}+\dfrac{5}{2} $.\\

  
 \end{itemize}


 \item Calcule as integrais triplas:
 \begin{itemize}
  \item[a)]$\displaystyle\iiint_B xyz^2dxdydz$ onde $B=[0,1]\times[0,2]\times[1,3].$\\
 \item[b)]$\displaystyle\iiint_B 2y\sen(yz)dxdydz$ onde $B$ \'e o paralelep\'ipedo limitado por $x=\pi$, $y=\frac{\pi}{2}$, $z=\frac{\pi}{3}$ e os planos coordenados.\\
 \item[c)]$\displaystyle\int_{1}^{3}\int_{x}^{x^2}\int_{0}^{\ln z}xe^{y}dydzdx.$\\
 \item[d)]$\displaystyle\int_{1/3}^{1/2}\int_{0}^{\pi}\int_{0}^{1}zx\sen(xy)dzdydx.$\\
 \item[e)]$\displaystyle\iiint_B xydxdydz$ onde $B$ \'e o s\'olido limitado pelos cilindros parab\'olicos $x=y^2$ e $y=x^2$ e pelos planos $z=0$ e $z=x+y$.\\
  
 \end{itemize}

 
 \end{enumerate}
 


\begin{flushright}
 \textit{ Bons estudos!}
\end{flushright}
\begin{center}
 \textbf{Bibliografia:}\\ Stewart, J. - C\'alculo Vol II\\ Flemming, D. - C\'alculo B \\ Howard, A. - C\'alculo Vol II.\\ 
\end{center}
\end{document}
