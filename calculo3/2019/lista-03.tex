\documentclass[a4paper,5pt]{amsbook}
%%%%%%%%%%%%%%%%%%%%%%%%%%%%%%%%%%%%%%%%%%%%%%%%%%%%%%%%%%%%%%%%%%%%%

\usepackage{booktabs}
\usepackage{graphicx}
\usepackage{multicol}
\usepackage{textcomp}
\usepackage{systeme}
\usepackage{amssymb}
\usepackage[]{amsmath}
\usepackage{subcaption}
\usepackage[inline]{enumitem}
\usepackage{gensymb}

%%%%%%%%%%%%%%%%%%%%%%%%%%%%%%%%%%%%%%%%%%%%%%%%%%%%%%%%%%%%%%

\newcommand{\sen}{\,\mbox{sen}}
\newcommand{\tg}{\,\mbox{tg}\,}
\newcommand{\cosec}{\,\mbox{cosec}\,}
\newcommand{\cotg}{\,\mbox{cotg}\,}
\newcommand{\tr}{\,\mbox{tr}\,}
\newcommand{\ds}{\displaystyle}
\newcommand{\ra}{\rightarrow}

%%%%%%%%%%%%%%%%%%%%%%%%%%%%%%%%%%%%%%%%%%%%%%%%%%%%%%%%%%%%%%%%%%%%%%%%

\setlength{\textwidth}{16cm} %\setlength{\topmargin}{-1.3cm}
\setlength{\textheight}{25cm}
\setlength{\leftmargin}{1.2cm} \setlength{\rightmargin}{1.2cm}
\setlength{\oddsidemargin}{0cm}\setlength{\evensidemargin}{0cm}

%%%%%%%%%%%%%%%%%%%%%%%%%%%%%%%%%%%%%%%%%%%%%%%%%%%%%%%%%%%%%%%%%%%%%%%%

% \renewcommand{\baselinestretch}{1.6}
% \renewcommand{\thefootnote}{\fnsymbol{footnote}}
% \renewcommand{\theequation}{\thesection.\arabic{equation}}
% \setlength{\voffset}{-50pt}
% \numberwithin{equation}{chapter}

%%%%%%%%%%%%%%%%%%%%%%%%%%%%%%%%%%%%%%%%%%%%%%%%%%%%%%%%%%%%%%%%%%%%%%%

\begin{document}
\thispagestyle{empty}
\pagestyle{empty}
\begin{minipage}[h]{0.14\textwidth}
	\includegraphics[scale=0.24]{../../ufgd.png}
\end{minipage}
\begin{minipage}[h]{\textwidth}
\begin{tabular}{c}
{{\bf UNIVERSIDADE FEDERAL DA GRANDE DOURADOS}}\\
{{\bf C\'alculo Diferencial e Integral III --- Lista 3}}\\
{{\bf Prof.\ Adriano Barbosa}}\\
\end{tabular}
\vspace{-0.45cm}
%
\end{minipage}

%------------------------

\vspace{1cm}
%%%%%%%%%%%%%%%%%%%%%%%%%%%%%%%%   formulario  inicio  %%%%%%%%%%%%%%%%%%%%%%%%%%%%%%%%
\begin{enumerate}
    \setlength\itemsep{0.5cm}
    \item Encontre a equa\c{c}\~ao do plano tangente ao gr\'afico das fun\c{c}\~oes nos
    pontos dados.
        \begin{enumerate}
            \setlength\itemsep{0.5cm}
            \item $z=3y^2-2x^2+x$, $(2,-1,-3)$
            \item $z=\sqrt{xy}$, $(1,1,1)$
            \item $z=x \sen{(x+y)}$, $(-1,1,0)$
        \end{enumerate}

    \item Determine se as fun\c{c}\~oes abaixo s\~ao diferenci\'aveis no ponto dado e
    calcule a aproxima\c{c}\~ao linear $L(x,y)$ de $f$ naquele ponto.
        \begin{enumerate}
            \setlength\itemsep{0.5cm}
            \item $f(x,y) = 1+x\ln{(xy-5)}$, $(2,3)$
            \item $f(x,y) = \ds\frac{x}{x+y}$, $(2,1)$
        \end{enumerate}

    \item Sabendo que $f$ \'e diferenci\'avel e que $f(2,5)=6$, $\ds\frac{\partial
    f}{\partial x}(2,5)=1$, $\ds\frac{\partial f}{\partial y}(2,5)=-1$,
    encontre uma aproxima\c{c}\~ao para o valor de $f(1.02, 0.97)$.

    \item Use a regra da cadeia para calcular $\ds\frac{\partial z}{\partial t}$.
        \begin{enumerate}
            \setlength\itemsep{0.5cm}
            \item $z=x^2+y^2+xy$, onde $x=\sen{t}$ e $y=e^t$
            \item $z=\sqrt{1+x^2+y^2}$, onde $x=\ln{t}$ e $y=\cos{t}$
            \item $z=xe^{y/z}$, onde $x=t^2$, $y=1-t$ e $z=1+2t$
        \end{enumerate}

    \item Use a regra da cadeia para calcular $\ds\frac{\partial z}{\partial
    s}$ e $\ds\frac{\partial z}{\partial t}$.
        \begin{enumerate}
            \setlength\itemsep{0.5cm}
            \item $z=x^2y^3$, onde $x=s\cos{t}$ e $y=s \sen{t}$
            \item $z=\sen{\theta}\cos{\phi}$, onde $\theta=st^2$ e $\phi=s^2t$
            \item $z=e^r \cos{\theta}$, onde $r=st$ e $\theta=\sqrt{s^2+t^2}$
        \end{enumerate}

    \item Se $z=f(x,y)$, com $f$ diferenci\'avel e $x=g(t)$, $y=h(t)$, $g(3)=2$,
    $h(3)=7$, $g'(3)=5$, $h'(3)=-4$, $\ds\frac{\partial f}{\partial x}(2,7)=6$,
    $\ds\frac{\partial f}{\partial y}(2,7)=-8$, calcule $\ds\frac{\partial
    z}{\partial t}$ quando $t=3$.
\end{enumerate}

\end{document}
