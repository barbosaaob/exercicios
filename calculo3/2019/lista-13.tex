\documentclass[a4paper,5pt]{amsbook}
%%%%%%%%%%%%%%%%%%%%%%%%%%%%%%%%%%%%%%%%%%%%%%%%%%%%%%%%%%%%%%%%%%%%%

\usepackage{booktabs}
\usepackage{graphicx}
\usepackage{multicol}
\usepackage{textcomp}
\usepackage{systeme}
\usepackage{amssymb}
\usepackage[]{amsmath}
\usepackage{subcaption}
\usepackage[inline]{enumitem}
\usepackage{gensymb}

%%%%%%%%%%%%%%%%%%%%%%%%%%%%%%%%%%%%%%%%%%%%%%%%%%%%%%%%%%%%%%

\newcommand{\sen}{\,\mbox{sen}}
\newcommand{\tg}{\,\mbox{tg}\,}
\newcommand{\cosec}{\,\mbox{cosec}\,}
\newcommand{\cotg}{\,\mbox{cotg}\,}
\newcommand{\tr}{\,\mbox{tr}\,}
\newcommand{\ds}{\displaystyle}
\newcommand{\ra}{\rightarrow}

%%%%%%%%%%%%%%%%%%%%%%%%%%%%%%%%%%%%%%%%%%%%%%%%%%%%%%%%%%%%%%%%%%%%%%%%

\setlength{\textwidth}{16cm} %\setlength{\topmargin}{-1.3cm}
\setlength{\textheight}{25cm}
\setlength{\leftmargin}{1.2cm} \setlength{\rightmargin}{1.2cm}
\setlength{\oddsidemargin}{0cm}\setlength{\evensidemargin}{0cm}

%%%%%%%%%%%%%%%%%%%%%%%%%%%%%%%%%%%%%%%%%%%%%%%%%%%%%%%%%%%%%%%%%%%%%%%%

% \renewcommand{\baselinestretch}{1.6}
% \renewcommand{\thefootnote}{\fnsymbol{footnote}}
% \renewcommand{\theequation}{\thesection.\arabic{equation}}
% \setlength{\voffset}{-50pt}
% \numberwithin{equation}{chapter}

%%%%%%%%%%%%%%%%%%%%%%%%%%%%%%%%%%%%%%%%%%%%%%%%%%%%%%%%%%%%%%%%%%%%%%%

\begin{document}
\thispagestyle{empty}
\pagestyle{empty}
\begin{minipage}[h]{0.14\textwidth}
	\includegraphics[scale=0.24]{../../ufgd.png}
\end{minipage}
\begin{minipage}[h]{\textwidth}
\begin{tabular}{c}
{{\bf UNIVERSIDADE FEDERAL DA GRANDE DOURADOS}}\\
{{\bf C\'alculo Diferencial e Integral III --- Lista 13}}\\
{{\bf Prof.\ Adriano Barbosa}}\\
\end{tabular}
\vspace{-0.45cm}
%
\end{minipage}

%------------------------

\vspace{1cm}
%%%%%%%%%%%%%%%%%%%%%%%%%%%%%%%%   formulario  inicio  %%%%%%%%%%%%%%%%%%%%%%%%%%%%%%%%
\begin{enumerate}
    \setlength\itemsep{0.5cm}
    \item Calcule a integral de linha diretamente e utilizando o Teorema de
    Green.
        \begin{enumerate}
            \setlength\itemsep{0.3cm}
            \item $\int_C xy\ dx + x^2\ dy$, $C$ \'e o ret\^angulo com v\'ertices
            $(0,0)$, $(3,0)$, $(3,1)$ e $(0,1)$
            \item $\ds\int_C xy\ dx + x^2y^3\ dy$, onde $C$ \'e o tri\^angulo com
            v\'ertices $(0,0)$, $(1,0)$ e $(1,2)$
        \end{enumerate}

    \item Use o Teorema de Green para calcular a integral de linha ao longo da
    curva dada com orienta\c{c}\~ao positiva.
        \begin{enumerate}
            \setlength\itemsep{0.3cm}
            \item $\ds\int_C xy^2\ dx + 2x^2y\ dy$, $C$ \'e o tri\^angulo com
            v\'ertices $(0,0)$, $(2,2)$ e $(2,4)$
            \item $\ds\int_C (y+e^{\sqrt{x}})\ dx + (2x+\cos{y^2})\ dy$, $C$ \'e
            o limite da regi\~ao englobada pelas par\'abolas $y=x^2$ e $x=y^2$
        \end{enumerate}

    \item Use o Teorema de Green para calcular o trabalho realizado pela for\c{c}a
    $F(x,y)=(x(x+y), xy^2)$ ao mover uma part\'{\i}cula da origem ao longo do eixo
    $x$ para $(1,0)$, em seguida ao longo de um segmento de reta at\'e $(0,1)$, e
    ent\~ao de volte \`a origem ao longo do eixo $y$.

    \item Calcule a \'area da regi\~ao acima da curva
    $r(t)=(\cos{t},\sen{t}-1),0\le t\le \frac{\pi}{2}$ e abaixo do eixo $x$.
\end{enumerate}

\end{document}
