\documentclass[a4paper,5pt]{amsbook}
%%%%%%%%%%%%%%%%%%%%%%%%%%%%%%%%%%%%%%%%%%%%%%%%%%%%%%%%%%%%%%%%%%%%%

\usepackage{booktabs}
\usepackage{graphicx}
\usepackage{multicol}
\usepackage{textcomp}
\usepackage{systeme}
\usepackage{amssymb}
\usepackage[]{amsmath}
\usepackage{subcaption}
\usepackage[inline]{enumitem}
\usepackage{gensymb}

%%%%%%%%%%%%%%%%%%%%%%%%%%%%%%%%%%%%%%%%%%%%%%%%%%%%%%%%%%%%%%

\newcommand{\sen}{\,\mbox{sen}}
\newcommand{\tg}{\,\mbox{tg}\,}
\newcommand{\cosec}{\,\mbox{cosec}\,}
\newcommand{\cotg}{\,\mbox{cotg}\,}
\newcommand{\tr}{\,\mbox{tr}\,}
\newcommand{\ds}{\displaystyle}
\newcommand{\ra}{\rightarrow}

%%%%%%%%%%%%%%%%%%%%%%%%%%%%%%%%%%%%%%%%%%%%%%%%%%%%%%%%%%%%%%%%%%%%%%%%

\setlength{\textwidth}{16cm} %\setlength{\topmargin}{-1.3cm}
\setlength{\textheight}{25cm}
\setlength{\leftmargin}{1.2cm} \setlength{\rightmargin}{1.2cm}
\setlength{\oddsidemargin}{0cm}\setlength{\evensidemargin}{0cm}

%%%%%%%%%%%%%%%%%%%%%%%%%%%%%%%%%%%%%%%%%%%%%%%%%%%%%%%%%%%%%%%%%%%%%%%%

% \renewcommand{\baselinestretch}{1.6}
% \renewcommand{\thefootnote}{\fnsymbol{footnote}}
% \renewcommand{\theequation}{\thesection.\arabic{equation}}
% \setlength{\voffset}{-50pt}
% \numberwithin{equation}{chapter}

%%%%%%%%%%%%%%%%%%%%%%%%%%%%%%%%%%%%%%%%%%%%%%%%%%%%%%%%%%%%%%%%%%%%%%%

\begin{document}
\thispagestyle{empty}
\pagestyle{empty}
\begin{minipage}[h]{0.14\textwidth}
	\includegraphics[scale=0.24]{../../ufgd.png}
\end{minipage}
\begin{minipage}[h]{\textwidth}
\begin{tabular}{c}
{{\bf UNIVERSIDADE FEDERAL DA GRANDE DOURADOS}}\\
{{\bf C\'alculo Diferencial e Integral III --- Lista 9}}\\
{{\bf Prof.\ Adriano Barbosa}}\\
\end{tabular}
\vspace{-0.45cm}
%
\end{minipage}

%------------------------

\vspace{1cm}
%%%%%%%%%%%%%%%%%%%%%%%%%%%%%%%%   formulario  inicio  %%%%%%%%%%%%%%%%%%%%%%%%%%%%%%%%
\begin{enumerate}
    \setlength\itemsep{0.5cm}
    \item Calcule a integral dada, colocando-a em coordenadas polares.
        \begin{enumerate}
            \setlength\itemsep{0.3cm}
            \item $\ds\iint_R \sen{(x^2+y^2)}\ dA$, onde $R$ \'e a regi\~ao do
            primeiro quadrante entre os c\'{\i}rculos com centro na origem e reios 1
            e 3.
            \item $\ds\iint_D e^{-x^2-y^2}\ dA$, onde $D$ \'e a regi\~ao limitada
            pelo semic\'{\i}rculo $x=\sqrt{4-y^2}$ e o eio $y$.
            \item $\ds\iint_R \arctan{(y/x)}\ dA$, onde $R=\{(x,y)\ |\ 1\le
            x^2+y^2\le 4, 0\le y\le x\}$.
        \end{enumerate}

    \item Utilize coordenadas polares para determinar o volume do s\'olido.
        \begin{enumerate}
            \setlength\itemsep{0.3cm}
            \item Abaixo do cone $z=\sqrt{x^2+y^2}$ e acima do disco
            $x^2+y^2\le 4$.
            \item Limitado pelo hiperboloide $-x^2-y^2+z^2=1$ e pelo plano
            $z=2$.
        \end{enumerate}

    \item Calcule a integral iterada, convertendo-a antes para coordenadas
    polares.
        \begin{enumerate}
            \setlength\itemsep{0.3cm}
            \item $\ds\int_{-3}^{3} \int_0^{\sqrt{9-x^2}} \sen{(x^2+y^2)}\ dydx$
            \item $\ds\int_0^1 \int_y^{\sqrt{2-y^2}} x+y\ dxdy$
        \end{enumerate}
\end{enumerate}

\end{document}
