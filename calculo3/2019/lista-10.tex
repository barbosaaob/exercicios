\documentclass[a4paper,5pt]{amsbook}
%%%%%%%%%%%%%%%%%%%%%%%%%%%%%%%%%%%%%%%%%%%%%%%%%%%%%%%%%%%%%%%%%%%%%

\usepackage{booktabs}
\usepackage{graphicx}
\usepackage{multicol}
\usepackage{textcomp}
\usepackage{systeme}
\usepackage{amssymb}
\usepackage[]{amsmath}
\usepackage{subcaption}
\usepackage[inline]{enumitem}
\usepackage{gensymb}

%%%%%%%%%%%%%%%%%%%%%%%%%%%%%%%%%%%%%%%%%%%%%%%%%%%%%%%%%%%%%%

\newcommand{\sen}{\,\mbox{sen}}
\newcommand{\tg}{\,\mbox{tg}\,}
\newcommand{\cosec}{\,\mbox{cosec}\,}
\newcommand{\cotg}{\,\mbox{cotg}\,}
\newcommand{\tr}{\,\mbox{tr}\,}
\newcommand{\ds}{\displaystyle}
\newcommand{\ra}{\rightarrow}

%%%%%%%%%%%%%%%%%%%%%%%%%%%%%%%%%%%%%%%%%%%%%%%%%%%%%%%%%%%%%%%%%%%%%%%%

\setlength{\textwidth}{16cm} %\setlength{\topmargin}{-1.3cm}
\setlength{\textheight}{25cm}
\setlength{\leftmargin}{1.2cm} \setlength{\rightmargin}{1.2cm}
\setlength{\oddsidemargin}{0cm}\setlength{\evensidemargin}{0cm}

%%%%%%%%%%%%%%%%%%%%%%%%%%%%%%%%%%%%%%%%%%%%%%%%%%%%%%%%%%%%%%%%%%%%%%%%

% \renewcommand{\baselinestretch}{1.6}
% \renewcommand{\thefootnote}{\fnsymbol{footnote}}
% \renewcommand{\theequation}{\thesection.\arabic{equation}}
% \setlength{\voffset}{-50pt}
% \numberwithin{equation}{chapter}

%%%%%%%%%%%%%%%%%%%%%%%%%%%%%%%%%%%%%%%%%%%%%%%%%%%%%%%%%%%%%%%%%%%%%%%

\begin{document}
\thispagestyle{empty}
\pagestyle{empty}
\begin{minipage}[h]{0.14\textwidth}
	\includegraphics[scale=0.24]{../../ufgd.png}
\end{minipage}
\begin{minipage}[h]{\textwidth}
\begin{tabular}{c}
{{\bf UNIVERSIDADE FEDERAL DA GRANDE DOURADOS}}\\
{{\bf C\'alculo Diferencial e Integral III --- Lista 10}}\\
{{\bf Prof.\ Adriano Barbosa}}\\
\end{tabular}
\vspace{-0.45cm}
%
\end{minipage}

%------------------------

\vspace{1cm}
%%%%%%%%%%%%%%%%%%%%%%%%%%%%%%%%   formulario  inicio  %%%%%%%%%%%%%%%%%%%%%%%%%%%%%%%%
\begin{enumerate}
    \setlength\itemsep{0.5cm}
    \item Esboce o s\'olido descrito pelas desigualdades dadas.
        \begin{enumerate}
            \setlength\itemsep{0.3cm}
            \item $0\le r\le 2$, $-\pi/2\le \theta\le \pi/2$, $0\le z\le 1$
            \item $0\le \theta\le \pi/2$, $r\le z\le 2$
        \end{enumerate}

    \item Esboce o s\'olido cujo volume \'e dade pela integral
    $\ds\int_{-\pi/2}^{\pi/2} \int_0^2 \int_0^{r^2} r\ dz\ dr\ d\theta$ e
    calcule-a.

    \item Calcule $\ds\iiint_E \sqrt{x^2+y^2}\ dV$, onde $E$ \'e a regi\~ao que
    est\'a dentro do cilindro $x^2+y^2=16$ e entre os planos $z=-5$ e $z=4$.

    \item Calcule $\ds\iiint_E x+y+z\ dV$, onde $E$ \'e o s\'olido do primeiro
    octante que est\'a abaixo do paraboloide $z=4-x^2-y^2$.

    %\item Calcule a integral $\ds\int_{-2}^2
    %\int_{-\sqrt{4-y^2}}^{\sqrt{4-y^2}} \int_{\sqrt{x^2+y^2}}^2 xz\ dzdxdy$,
    %transformando para coordenadas cil\'{\i}ndricas.

    \item Esboce o s\'olido descrito pelas desigualdades.
        \begin{enumerate}
            \setlength\itemsep{0.3cm}
            \item $2\le \rho\le 4$, $0\le \phi\le \pi/3$, $0\le \theta\le \pi$
            \item $\rho\le 1$, $3\pi/4\le \phi\le \pi$
        \end{enumerate}
        
    \item Calcule $\ds\iiint_B (x^2+y^2+z^2)^2\ dV$, onde $B$ \'e a bola com
    centro na origem e raio 5.

    \item Calcule $\iiint_E x^2+y^2\ dV$, onde $E$ est\'a entre as esferas
    $x^2+y^2+z^2=4$ e $x^2+y^2+z^2=9$.

    \item Esboce o s\'olido cujo volume \'e dado pela integral $\ds\int_0^{\pi/6}
    \int_0^{\pi/2} \int_0^3 \rho^2\sen\phi\ d\rho\ d\theta\ d\phi$.

\end{enumerate}

\end{document}
