\documentclass[a4paper,5pt]{amsbook}
%%%%%%%%%%%%%%%%%%%%%%%%%%%%%%%%%%%%%%%%%%%%%%%%%%%%%%%%%%%%%%%%%%%%%

\usepackage{graphics}
\usepackage[]{float}
\usepackage{amssymb}
\usepackage{amsfonts}
\usepackage[]{amsmath}
\usepackage[]{epsfig}
%\usepackage[brazil]{babel}
\usepackage[utf8]{inputenc}
\usepackage{verbatim}
%\usepackage[]{pstricks}
%\usepackage[notcite,notref]{showkeys}

%%%%%%%%%%%%%%%%%%%%%%%%%%%%%%%%%%%%%%%%%%%%%%%%%%%%%%%%%%%%%%%%%%%%%%

\newtheorem{theorem}{Teorema}[chapter]
\newtheorem{lemma}{Lema}[chapter]
\newtheorem{definition}{Defini\c{c}\~{a}o}[chapter]
\newtheorem{remark}{Observa\c{c}\~{a}o}[chapter]
\newtheorem{proposition}{Proposi\c{c}\~{a}o}[chapter]
\newtheorem{corollary}{Corolario}[chapter]
\newtheorem{example}{Exemplo}[chapter]

%%%%%%%%%%%%%%%%%%%%%%%%%%%%%%%%%%%%%%%%%%%%%%%%%%%%%%%%%%%%%%

\newcommand{\R}{\mathbb R}
\newcommand{\Q}{\mathbb Q}
\newcommand{\Z}{\mathbb Z}
\newcommand{\N}{\mathbb N}
\newcommand{\lan}{\langle}
\newcommand{\ran}{\rangle}
\newcommand{\sen}{\text{sen}}
\newcommand{\I}{\infty}
\newcommand{\vet}[1]{\ensuremath{\overrightarrow{#1}}}
%%%%%%%%%%%%%%%%%%%%%%%%%%%%%%%%%%%%%%%%%%%%%%%%%%%%%%%%%%%%%%%%%%%%%%%%
%---------------------------------------------------------
\setlength{\textwidth}{16cm} %\setlength{\topmargin}{-0.1cm}
\setlength{\leftmargin}{1.2cm} \setlength{\rightmargin}{1.2cm}
\setlength{\oddsidemargin}{0cm}\setlength{\evensidemargin}{0cm}
%---------------------------------------------------------
%%%%%%%%%%%%%%%%%%%%%%%%%%%%%%%%%%%%%%%%%%%%%%%%%%%%%%%%%%%%%%%%%%%%%%%%

%\setlength{\textwidth}{16cm}
%\addtolength{\oddsidemargin}{-1.7cm}
%\addtolength{\evensidemargin}{-1.7cm}
%\setlength{\textheight}{20cm}
%%%%%%%%%%%%%%%%%%%%%%%%%%%%%%%%%%%%%%%%%%%%%%%%%%%%%%%%%%%%%%%%%%%%%%%
%%
\renewcommand{\baselinestretch}{1.6}
\renewcommand{\thefootnote}{\fnsymbol{footnote}}
\renewcommand{\theequation}{\thesection.\arabic{equation}}
\setlength{\voffset}{-50pt}
\numberwithin{equation}{chapter}
%%
%%%%%%%%%%%%%%%%%%%%%%%%%%%%%%%%%%%%%%%%%%%%%%%%%%%%%%%%%%%%%%%%%%%%%%%


%%%%%%%%%%%%%%%%%%%%%%%%%%%%%%%%%%%%%%%%%%%%%%%%%%%%%%%%%%%%%%%%%%%%%%%


%%%%%%%%%%%%%%%%%%%%%%%%%%%%%%%%%%%%%%%%%%%%%%%%%%%%%%%%%%%%%%%%%%%%%%%
\begin{document}
\thispagestyle{empty}
\begin{minipage}[b]{0.45\linewidth}
\begin{tabular}{c}
\hline \hline
{{\bf UNIVERSIDADE FEDERAL DA GRANDE DOURADOS}}\\

{{\bf FACET}} \\

\hline
Lista 03\hspace{12 cm}  12/09/2016  \\
\hline \hline
\end{tabular}
%
\end{minipage} 

\vspace{0.2cm}
%%%%%%%%%%%%%%%%%%%%%%%%%%%%%%%%   formulario  in\'icio  %%%%%%%%%%%%%%%%%%%%%%%%%%%%%%%%
 
 
\begin{enumerate}
\item Considere o campo de for\c{c}as $\overrightarrow{F}(x,y)=\left(\dfrac{x}{x^2+y^2}, \dfrac{y}{x^2+y^2}\right)$.\\
\begin{itemize}
 \item[a)]Calcule o trabalho realizado pelo campo $\overrightarrow{F}$ numa part\'icula que se move ao longo da curva C, que consiste do arco da par\'abola $y=x^2-1$ com $-1\leq x \leq 2$, seguido do segmento da reta que une os pontos $(2,3)$ e $(-1,0)$.\\
 \item[b)] Mostre que $ \displaystyle\oint_{C}\overrightarrow{F}\cdot d\overrightarrow{r} = 0$ para toda curva fechada simples $C$, suave por partes, que circunda a origem.\\
\end{itemize}

\item Calcule o trabalho realizado pelo campo de for\c{c}as $\overrightarrow{F}$ numa part\'icula que se move ao longo de uma curva lisa C, do ponto $A$ ao ponto $B$ dados:
\begin{itemize}
 \item[a)]$F(x,y)=3y\textbf{i}+3x\textbf{j}$ do ponto $A=(1,2)$ ao ponto $B=(4,0)$.\\
 \item[b)]$F(x,y)=ye^{xy}\textbf{i}+xe^{xy}\textbf{j}$ do ponto $A=(-1,1)$ ao ponto $B=(2,0)$.\\
 \item[c)]$F(x,y,z)=2xy\textbf{i}+x^2\textbf{j}+2\textbf{k}$ do ponto $A=(0,1,1)$ ao ponto $B=(1,0,1)$.\\
 \item[d)]$F(x,y,z)=2x\sen z\textbf{i}+(z^3-e^y)\textbf{j}+(x^2\cos z+3yz^2)\textbf{k}$ do ponto $A=(1,1,1)$ ao ponto $B=(1,2,3)$.\\
\end{itemize}

\item Considere as fun\c{c}\~oes $P(x,y)=\dfrac{-y}{x^2+y^2}$ e $Q(x,y)=\dfrac{x}{x^2+y^2}$, definidas para $(x,y)\neq (0,0)$. Considere ainda D a regi\~ao descrita por $0<x^2+y^2\leq R$ e $\partial D$ a curva fronteira desta regi\~ao.
 \begin{itemize}
  \item[a)] Mostre que $\displaystyle\oint_{\partial D}Pdx+Qdy = 2\pi;$\\
  \item[b)]Mostre que $\displaystyle\int\int_D\left(\frac{\partial Q}{\partial x} - \frac{\partial P}{\partial y}\right)dxdy = 0.$ Por que isto n\~ao contradiz o Teorema de Green?\\
  \item[c)]  Mostre que $\displaystyle\oint_{C}Pdx+Qdy = 2\pi$ para toda curva fechada simples, suave por partes, orientada no sentindo anti-hor\'ario que circunda a origem.\\
 \end{itemize}
 \item Calcule as integrais de linha:
\begin{itemize}
 \item[a)]$\displaystyle\oint_{C}ydx - xdy$, onde C \'e o tri\^angulo definido pelos pontos $A=(0,0)$, $B=(2,0)$ e $C=(0,4)$, no sentido hor\'ario.\\
  \item[b)]$\displaystyle\oint_{C}ydx - xdy$, onde C \'e a cardi\'oide de equa\c{c}\~ao polar $$r(\theta)=2(1+cos\theta)\quad (0\leq \theta\leq 2\pi)$$ e equa\c{c}\~ao param\'etrica  $$\overrightarrow{r}(\theta)=(2\cos t +\cos 2t+1,2\sen t +\sen 2t).$$
  
\end{itemize}


\end{enumerate}



 


\begin{flushright}
 \textit{ Bons estudos!}
\end{flushright}
\begin{center}
 \textbf{Bibliografia:}\\ Stewart, J. - C\'alculo Vol II\\ Flemming, D. - C\'alculo B \\ Howard, A. - C\'alculo Vol II\\ Guidorizzi, H. - Um curso de c\'alculo Vol 3.
\end{center}
\end{document}
