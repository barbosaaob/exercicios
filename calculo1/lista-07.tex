\documentclass[a4paper,5pt]{amsbook}
%%%%%%%%%%%%%%%%%%%%%%%%%%%%%%%%%%%%%%%%%%%%%%%%%%%%%%%%%%%%%%%%%%%%%

\usepackage{booktabs}
\usepackage{graphicx}
\usepackage{multicol}
\usepackage{textcomp}
\usepackage{systeme}
\usepackage{amssymb}
\usepackage[]{amsmath}
\usepackage{subcaption}
\usepackage[inline]{enumitem}
\usepackage{gensymb}

%%%%%%%%%%%%%%%%%%%%%%%%%%%%%%%%%%%%%%%%%%%%%%%%%%%%%%%%%%%%%%

\newcommand{\sen}{\,\mbox{sen}\,}
\newcommand{\tg}{\,\mbox{tg}\,}
\newcommand{\cosec}{\,\mbox{cosec}\,}
\newcommand{\cotg}{\,\mbox{cotg}\,}
\newcommand{\tr}{\,\mbox{tr}\,}
\newcommand{\ds}{\displaystyle}
\newcommand{\ra}{\rightarrow}

%%%%%%%%%%%%%%%%%%%%%%%%%%%%%%%%%%%%%%%%%%%%%%%%%%%%%%%%%%%%%%%%%%%%%%%%

\setlength{\textwidth}{16cm} \setlength{\topmargin}{-1.7cm}
\setlength{\textheight}{25cm}
\setlength{\leftmargin}{1.2cm} \setlength{\rightmargin}{1.2cm}
\setlength{\oddsidemargin}{0cm}\setlength{\evensidemargin}{0cm}

%%%%%%%%%%%%%%%%%%%%%%%%%%%%%%%%%%%%%%%%%%%%%%%%%%%%%%%%%%%%%%%%%%%%%%%%

% \renewcommand{\baselinestretch}{1.6}
% \renewcommand{\thefootnote}{\fnsymbol{footnote}}
% \renewcommand{\theequation}{\thesection.\arabic{equation}}
% \setlength{\voffset}{-50pt}
% \numberwithin{equation}{chapter}

%%%%%%%%%%%%%%%%%%%%%%%%%%%%%%%%%%%%%%%%%%%%%%%%%%%%%%%%%%%%%%%%%%%%%%%

\begin{document}
\thispagestyle{empty}
\pagestyle{empty}
\begin{minipage}[h]{0.14\textwidth}
	\includegraphics[scale=0.24]{../ufgd.png}
\end{minipage}
\begin{minipage}[h]{\textwidth}
\begin{tabular}{c}
{{\bf UNIVERSIDADE FEDERAL DA GRANDE DOURADOS}}\\
{{\bf C\'alculo Diferencial e Integral --- Lista 7}}\\
{{\bf Prof.\ Adriano Barbosa}}\\
\end{tabular}
\vspace{-0.45cm}
%
\end{minipage}

%------------------------

\vspace{1cm}
%%%%%%%%%%%%%%%%%%%%%%%%%%%%%%%%   formulario  inicio  %%%%%%%%%%%%%%%%%%%%%%%%%%%%%%%%
\begin{enumerate}
    \vspace{0.5cm}
    \item Calcule a derivada das fun\c{c}\~oes abaixo:
        \begin{enumerate}
            \vspace{0.3cm}
            \item $f(x)=3x^2-2\cos{x}$
            \vspace{0.3cm}
            \item $f(x)=\ds\frac{x}{2-\tg{x}}$
            \vspace{0.3cm}
            \item $f(\theta)=\sen{\theta} \cos{\theta}$
            \vspace{0.3cm}
            \item $f(x)=\sqrt{x}\sen{x}$
        \end{enumerate}

    \vspace{0.5cm}
    \item Mostre que $\ds\frac{d}{dx}(\cosec{x}) = -\cosec{x} \cotg{x}$.

    \vspace{0.5cm}
    \item Mostre que $\ds\frac{d}{dx}(\sec{x}) = \sec{x}\tg{x}$.

    \vspace{0.5cm}
    \item Mostre que $\ds\frac{d}{dx}(\cotg{x}) = -\cosec^2{x}$.

    \vspace{0.5cm}
    \item Calcule os limites abaixo:
        \begin{enumerate}
            \vspace{0.3cm}
            \item $\ds\lim_{x\ra 0} \frac{\sen{3x}}{x}$
            \vspace{0.3cm}
            \item $\ds\lim_{x\ra 0} \frac{\tg{6x}}{\sen{2x}}$
            \vspace{0.3cm}
            \item $\ds\lim_{x\ra 0} \frac{\sen{x^2}}{x}$
        \end{enumerate}
\end{enumerate}

\end{document}
