\documentclass[a4paper,5pt]{amsbook}
%%%%%%%%%%%%%%%%%%%%%%%%%%%%%%%%%%%%%%%%%%%%%%%%%%%%%%%%%%%%%%%%%%%%%

%\usepackage{booktabs}
\usepackage{graphicx}
%\usepackage{multicol}
%\usepackage{textcomp}
%\usepackage{systeme}
%\usepackage{amssymb}
%\usepackage[]{amsmath}
%\usepackage{subcaption}
%\usepackage[inline]{enumitem}
%\usepackage{gensymb}

%%%%%%%%%%%%%%%%%%%%%%%%%%%%%%%%%%%%%%%%%%%%%%%%%%%%%%%%%%%%%%

\newcommand{\sen}{\,\mbox{sen}\,}
\newcommand{\tg}{\,\mbox{tg}\,}
\newcommand{\cosec}{\,\mbox{cosec}\,}
\newcommand{\cotg}{\,\mbox{cotg}\,}
\newcommand{\tr}{\,\mbox{tr}\,}
\newcommand{\ds}{\displaystyle}
\newcommand{\ra}{\rightarrow}

%%%%%%%%%%%%%%%%%%%%%%%%%%%%%%%%%%%%%%%%%%%%%%%%%%%%%%%%%%%%%%%%%%%%%%%%

\setlength{\textwidth}{16cm} \setlength{\topmargin}{-1.7cm}
\setlength{\textheight}{25cm}
\setlength{\leftmargin}{1.2cm} \setlength{\rightmargin}{1.2cm}
\setlength{\oddsidemargin}{0cm}\setlength{\evensidemargin}{0cm}

%%%%%%%%%%%%%%%%%%%%%%%%%%%%%%%%%%%%%%%%%%%%%%%%%%%%%%%%%%%%%%%%%%%%%%%%

% \renewcommand{\baselinestretch}{1.6}
% \renewcommand{\thefootnote}{\fnsymbol{footnote}}
% \renewcommand{\theequation}{\thesection.\arabic{equation}}
% \setlength{\voffset}{-50pt}
% \numberwithin{equation}{chapter}

%%%%%%%%%%%%%%%%%%%%%%%%%%%%%%%%%%%%%%%%%%%%%%%%%%%%%%%%%%%%%%%%%%%%%%%

\begin{document}
\thispagestyle{empty}
\pagestyle{empty}
\begin{minipage}[h]{0.14\textwidth}
	\includegraphics[scale=0.24]{../ufgd.png}
\end{minipage}
\begin{minipage}[h]{\textwidth}
\begin{tabular}{c}
{{\bf UNIVERSIDADE FEDERAL DA GRANDE DOURADOS}}\\
{{\bf C\'alculo Diferencial e Integral --- Lista 11}}\\
{{\bf Prof.\ Adriano Barbosa}}\\
\end{tabular}
\vspace{-0.45cm}
%
\end{minipage}

%------------------------

\vspace{1cm}
%%%%%%%%%%%%%%%%%%%%%%%%%%%%%%%%   formulario  inicio  %%%%%%%%%%%%%%%%%%%%%%%%%%%%%%%%
\begin{enumerate}
    \vspace{0.5cm}
    \item Verifique as hip\'oteses do Teorema de Rolle para as fun\c{c}\~oes abaixo
        nos intervalos dados. Em seguida encontre todos os n\'umeros $c$ que
        satisfazem o Teorema.
        \begin{enumerate}
            \item $f(x)=5-12x+3x^2$, $[1,3]$
            \item $f(x)=\sqrt{x}-\ds\frac{1}{3}x$, $[0,9]$
        \end{enumerate}

    \vspace{0.5cm}
    \item Verifique as hip\'oteses do Teorema do Valor M\'edio para as
        fun\c{c}\~oes abaixo nos intervalos dados. Em seguida encontre todos os n\'umeros $c$
        que satisfazem o Teorema.
        \begin{enumerate}
            \item $f(x) = 2x^2-3x+1$, $[0,2]$
            \item $f(x) = \sqrt[3]{x}$, $[0,1]$
        \end{enumerate}

    \vspace{0.5cm}
    \item Seja $f(x) = {(x-3)}^{-2}$. Mostre que n\~ao existe $c\in(1,4)$ tal
        que $f'(c) = \ds\frac{f(4)-f(1)}{4-1}$. Por que isso n\~ao contradiz o Teorema
        do Valor M\'edio?

    \vspace{0.5cm}
    \item
        \begin{enumerate}
            \item Como determinar quando $f$ \'e crescente ou decrescente?
            \item Como podemos determinar se o gr\'afico de $f$ \'e c\^oncavo para
                cima ou para baixo?
            \item Como localizar um ponto de inflex\~ao?
        \end{enumerate}

    \vspace{0.5cm}
    \item Desenhe o gr\'afico de fun\c{c}\~oes que satisfa\c{c}am as condi\c{c}\~oes abaixo:
        \begin{enumerate}
            \item $f'(0)=f'(2)=f'(4)=0$, $f'(x)>0$ se $x<0$ ou se $2<x<4$,
                $f'(x)<0$ se $0<x<2$ ou $x>4$, $f''(x)>0$ se $1<x<3$,
                $f''(x)<0$ se $x<1$ ou $x>3$.
            \item $f'(x)>0$ se $|x|<2$, $f'(x)<0$ se $|x|>2$, $f'(-2)=0$,
                $\ds\lim_{x\ra2}|f'(x)|=\infty$, $f''(x)>0$ se $x\neq2$.
        \end{enumerate}

    \vspace{0.5cm}
    \item Para cada item abaixo encontre:
        \begin{enumerate}
            \item[i)] Os intervalos de crescimento e decrescimento da fun\c{c}\~ao.
            \item[ii)] Onde a fun\c{c}\~ao atinge seus m\'aximos e m\'{\i}nimos e seus valores.
            \item[iii)] Os intervalos onde o gr\'afico da fun\c{c}\~ao tem concavidade para cima e para baixo.
            \item[iv)] Esboce o gr\'afico das fun\c{c}\~oes.
        \end{enumerate}
        \vspace{0.3cm}
        \begin{enumerate}
            \item $f(x)=x^3-12x+2$
            \item $f(x)={(x+1)}^5-5x-2$
            \item $f(x)=x^{1/3}(x+4)$
            \item $f(x)=2\cos{x}+\cos^2{x}$, $0\le x \le 2\pi$
        \end{enumerate}

\end{enumerate}

\end{document}
