\documentclass[a4paper,5pt]{amsbook}
%%%%%%%%%%%%%%%%%%%%%%%%%%%%%%%%%%%%%%%%%%%%%%%%%%%%%%%%%%%%%%%%%%%%%

\usepackage{booktabs}
\usepackage{graphicx}
\usepackage{multicol}
\usepackage{textcomp}
\usepackage{systeme}
\usepackage{amssymb}
\usepackage[]{amsmath}
\usepackage{subcaption}
\usepackage[inline]{enumitem}
\usepackage{gensymb}

%%%%%%%%%%%%%%%%%%%%%%%%%%%%%%%%%%%%%%%%%%%%%%%%%%%%%%%%%%%%%%

\newcommand{\sen}{\,\mbox{sen}\,}
\newcommand{\tg}{\,\mbox{tg}\,}
\newcommand{\cosec}{\,\mbox{cosec}\,}
\newcommand{\cotg}{\,\mbox{cotg}\,}
\newcommand{\tr}{\,\mbox{tr}\,}
\newcommand{\ds}{\displaystyle}
\newcommand{\ra}{\rightarrow}

%%%%%%%%%%%%%%%%%%%%%%%%%%%%%%%%%%%%%%%%%%%%%%%%%%%%%%%%%%%%%%%%%%%%%%%%

\setlength{\textwidth}{16cm} \setlength{\topmargin}{-1.7cm}
\setlength{\textheight}{25cm}
\setlength{\leftmargin}{1.2cm} \setlength{\rightmargin}{1.2cm}
\setlength{\oddsidemargin}{0cm}\setlength{\evensidemargin}{0cm}

%%%%%%%%%%%%%%%%%%%%%%%%%%%%%%%%%%%%%%%%%%%%%%%%%%%%%%%%%%%%%%%%%%%%%%%%

% \renewcommand{\baselinestretch}{1.6}
% \renewcommand{\thefootnote}{\fnsymbol{footnote}}
% \renewcommand{\theequation}{\thesection.\arabic{equation}}
% \setlength{\voffset}{-50pt}
% \numberwithin{equation}{chapter}

%%%%%%%%%%%%%%%%%%%%%%%%%%%%%%%%%%%%%%%%%%%%%%%%%%%%%%%%%%%%%%%%%%%%%%%

\begin{document}
\thispagestyle{empty}
\pagestyle{empty}
\begin{minipage}[h]{0.14\textwidth}
	\includegraphics[scale=0.24]{../../ufgd.png}
\end{minipage}
\begin{minipage}[h]{\textwidth}
\begin{tabular}{c}
{{\bf UNIVERSIDADE FEDERAL DA GRANDE DOURADOS}}\\
{{\bf C\'alculo Diferencial e Integral --- Lista 3}}\\
{{\bf Prof.\ Adriano Barbosa}}\\
\end{tabular}
\vspace{-0.45cm}
%
\end{minipage}

%------------------------

\vspace{1cm}
%%%%%%%%%%%%%%%%%%%%%%%%%%%%%%%%   formulario  inicio  %%%%%%%%%%%%%%%%%%%%%%%%%%%%%%%%
\begin{enumerate}
    \vspace{0.5cm}
    \item Derive as fun\c{c}\~oes:
        \begin{enumerate}
            \vspace{0.3cm}
            \item $f(x)=e^5$
            \vspace{0.3cm}
            \item $f(x)=(x^3+2x)e^x$
            \vspace{0.3cm}
            \item $y=e^{ax^3}$
            \vspace{0.3cm}
            \item $f(x)=x\ln{x}-x$
            \vspace{0.3cm}
            \item $f(x)=\sen{(\ln{x})}$
            \vspace{0.3cm}
            \item $f(x)=\ln{\left(\ds\frac{1}{x}\right)}$
            \vspace{0.3cm}
            \item $f(x)=\log_{10}(x^3+1)$
            \vspace{0.3cm}
            \item $y=2x\log_{10}\sqrt{x}$
            \vspace{0.3cm}
            \item $y=\log_2(e^{-x}\cos{\pi x})$
            \vspace{0.3cm}
            \item $f(t)=10^{\sqrt{t}}$
            \vspace{0.3cm}
            \item $F(t)=3^{\cos{2t}}$
        \end{enumerate}

    \vspace{0.5cm}
    \item Se $f(x)=e^{2x}$, encontre a f\'ormula para $f^{(n)}(x)$ em fun\c{c}\~ao de $n$.

    \vspace{0.5cm}
    \item Encontre a equa\c{c}\~ao da reta tangente ao gr\'afico de $y=\sen(2\ln{x})$
        no ponto $(1,0)$.

    \vspace{0.5cm}
    \item Seja $I_{f}(x)$ um operador tal que sua derivada \'e $I_{f}'(x)=f(x)$.
        Calcule a derivada de $I_{f}(\ln{x})$, onde $f(t)=e^{t}$.

    \vspace{0.5cm}
    \item Seja $f(x)=\ds\log_{10}{\left(1+\frac{1}{x}\right)}$:
        \begin{enumerate}
            \vspace{0.3cm}
            \item Calcule a fun\c{c}\~ao inversa de $f$.
            \vspace{0.3cm}
            \item Calcule a derivada de $f^{-1}$.
        \end{enumerate}
\end{enumerate}

\end{document}
