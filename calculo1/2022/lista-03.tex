\documentclass[a4paper,5pt]{amsbook}
%%%%%%%%%%%%%%%%%%%%%%%%%%%%%%%%%%%%%%%%%%%%%%%%%%%%%%%%%%%%%%%%%%%%%

\usepackage{booktabs}
\usepackage{graphicx}
\usepackage{multicol}
\usepackage{textcomp}
\usepackage{systeme}
\usepackage{amssymb}
\usepackage[]{amsmath}
\usepackage{subcaption}
\usepackage[inline]{enumitem}
\usepackage{gensymb}

%%%%%%%%%%%%%%%%%%%%%%%%%%%%%%%%%%%%%%%%%%%%%%%%%%%%%%%%%%%%%%

\newcommand{\sen}{\,\mbox{sen}\,}
\newcommand{\tg}{\,\mbox{tg}\,}
\newcommand{\cosec}{\,\mbox{cosec}\,}
\newcommand{\cotg}{\,\mbox{cotg}\,}
\newcommand{\tr}{\,\mbox{tr}\,}
\newcommand{\ds}{\displaystyle}
\newcommand{\ra}{\rightarrow}

%%%%%%%%%%%%%%%%%%%%%%%%%%%%%%%%%%%%%%%%%%%%%%%%%%%%%%%%%%%%%%%%%%%%%%%%

\setlength{\textwidth}{16cm} \setlength{\topmargin}{-1.7cm}
\setlength{\textheight}{25cm}
\setlength{\leftmargin}{1.2cm} \setlength{\rightmargin}{1.2cm}
\setlength{\oddsidemargin}{0cm}\setlength{\evensidemargin}{0cm}

%%%%%%%%%%%%%%%%%%%%%%%%%%%%%%%%%%%%%%%%%%%%%%%%%%%%%%%%%%%%%%%%%%%%%%%%

% \renewcommand{\baselinestretch}{1.6}
% \renewcommand{\thefootnote}{\fnsymbol{footnote}}
% \renewcommand{\theequation}{\thesection.\arabic{equation}}
% \setlength{\voffset}{-50pt}
% \numberwithin{equation}{chapter}

%%%%%%%%%%%%%%%%%%%%%%%%%%%%%%%%%%%%%%%%%%%%%%%%%%%%%%%%%%%%%%%%%%%%%%%

\begin{document}
\thispagestyle{empty}
\pagestyle{empty}
\begin{minipage}[h]{0.14\textwidth}
	\includegraphics[scale=0.24]{../../ufgd.png}
\end{minipage}
\begin{minipage}[h]{\textwidth}
\begin{tabular}{c}
{{\bf UNIVERSIDADE FEDERAL DA GRANDE DOURADOS}}\\
{{\bf C\'alculo Diferencial e Integral --- Lista 3}}\\
{{\bf Prof.\ Adriano Barbosa}}\\
\end{tabular}
\vspace{-0.45cm}
%
\end{minipage}

%------------------------

\vspace{1cm}
%%%%%%%%%%%%%%%%%%%%%%%%%%%%%%%%   formulario  inicio  %%%%%%%%%%%%%%%%%%%%%%%%%%%%%%%%
\begin{enumerate}
    \vspace{0.5cm}
    \item Escreva as fun\c{c}\~oes abaixo na forma $f(g(x))$ identificando as fun\c{c}\~oes
        $f$ e $g$:

        \begin{enumerate*}
            \item $y=\sqrt[3]{1+4x}$
            \hspace{1cm}
            \hspace{1cm}
            \item $y=\tg(\pi x)$
            \hspace{1cm}
            \hspace{1cm}
            \item $y=\sqrt{\sen{x}}$
        \end{enumerate*}

    \vspace{0.5cm}
    \item Calcule a derivada das fun\c{c}\~oes:
        \begin{enumerate}
            \vspace{0.3cm}
            \item $F(x)={(x^4+3x^2-2)}^5$
            \vspace{0.3cm}
            \item $y=\sen{(x \cos{x})}$
            \vspace{0.3cm}
            \item $y=\sen\sqrt{1+x^2}$
            \vspace{0.3cm}
            \item $y=\sqrt{x+\sqrt{x}}$
            \vspace{0.3cm}
            \item $F(x)=\cos\left(\sqrt{\sen{(\tg{(\pi x)})}}\right)$
            \vspace{0.3cm}
            \item $y = \sec^2x + \tg^2x$
            \vspace{0.3cm}
            \item $y = \left[x+(x+\sen^2x)^3\right]^4$
            \vspace{0.3cm}
            \item $y = \ds\left[\frac{1-\cos(2x)}{1+\cos(2x)}\right]^4$
        \end{enumerate}

    \vspace{0.5cm}
    \item Seja $r(x)=f(g(h(x)))$, onde $h(1)=2$, $g(2)=3$, $h'(1)=4$, $g'(2)=5$
        e $f'(3)=6$. Calcule $r'(1)$.

    \vspace{0.5cm}
    \item Se $g$ \'e uma fun\c{c}\~ao duas vezes deriv\'avel e $f(x)=xg(x^2)$, calcule
        $f''$ em fun\c{c}\~ao de $g$, $g'$ e $g''$.
\end{enumerate}

\end{document}
