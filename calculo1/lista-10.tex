\documentclass[a4paper,5pt]{amsbook}
%%%%%%%%%%%%%%%%%%%%%%%%%%%%%%%%%%%%%%%%%%%%%%%%%%%%%%%%%%%%%%%%%%%%%

%\usepackage{booktabs}
\usepackage{graphicx}
%\usepackage{multicol}
%\usepackage{textcomp}
%\usepackage{systeme}
%\usepackage{amssymb}
%\usepackage[]{amsmath}
%\usepackage{subcaption}
%\usepackage[inline]{enumitem}
%\usepackage{gensymb}

%%%%%%%%%%%%%%%%%%%%%%%%%%%%%%%%%%%%%%%%%%%%%%%%%%%%%%%%%%%%%%

\newcommand{\sen}{\,\mbox{sen}\,}
\newcommand{\tg}{\,\mbox{tg}\,}
\newcommand{\cosec}{\,\mbox{cosec}\,}
\newcommand{\cotg}{\,\mbox{cotg}\,}
\newcommand{\tr}{\,\mbox{tr}\,}
\newcommand{\ds}{\displaystyle}
\newcommand{\ra}{\rightarrow}

%%%%%%%%%%%%%%%%%%%%%%%%%%%%%%%%%%%%%%%%%%%%%%%%%%%%%%%%%%%%%%%%%%%%%%%%

\setlength{\textwidth}{16cm} \setlength{\topmargin}{-1.7cm}
\setlength{\textheight}{25cm}
\setlength{\leftmargin}{1.2cm} \setlength{\rightmargin}{1.2cm}
\setlength{\oddsidemargin}{0cm}\setlength{\evensidemargin}{0cm}

%%%%%%%%%%%%%%%%%%%%%%%%%%%%%%%%%%%%%%%%%%%%%%%%%%%%%%%%%%%%%%%%%%%%%%%%

% \renewcommand{\baselinestretch}{1.6}
% \renewcommand{\thefootnote}{\fnsymbol{footnote}}
% \renewcommand{\theequation}{\thesection.\arabic{equation}}
% \setlength{\voffset}{-50pt}
% \numberwithin{equation}{chapter}

%%%%%%%%%%%%%%%%%%%%%%%%%%%%%%%%%%%%%%%%%%%%%%%%%%%%%%%%%%%%%%%%%%%%%%%

\begin{document}
\thispagestyle{empty}
\pagestyle{empty}
\begin{minipage}[h]{0.14\textwidth}
	\includegraphics[scale=0.24]{../ufgd.png}
\end{minipage}
\begin{minipage}[h]{\textwidth}
\begin{tabular}{c}
{{\bf UNIVERSIDADE FEDERAL DA GRANDE DOURADOS}}\\
{{\bf C\'alculo Diferencial e Integral --- Lista 10}}\\
{{\bf Prof.\ Adriano Barbosa}}\\
\end{tabular}
\vspace{-0.45cm}
%
\end{minipage}

%------------------------

\vspace{1cm}
%%%%%%%%%%%%%%%%%%%%%%%%%%%%%%%%   formulario  inicio  %%%%%%%%%%%%%%%%%%%%%%%%%%%%%%%%
\begin{enumerate}
    \vspace{0.5cm}
    \item Explique a diferen\c{c}a ente m\'aximo local e absoluto e entre m\'{\i}nimo
	    local e absoluto.

    \vspace{0.5cm}
    \item Desenhe o gr\'afico de uma fun\c{c}\~ao cont\'{\i}nua no intervalo $[1, 5]$ tal que:
        \begin{enumerate}
            %\hspace{0.3cm}
            \item Possui um m\'{\i}nimo absoluto em 2, um m\'aximo absoluto em 3 e
		    um m\'{\i}nimo local em 4.
            %\hspace{0.3cm}
            \item Possui um m\'aximo absoluto em 5, um m\'{\i}nimo absoluto em 2, um
		    m\'aximo local em 3 e m\'{\i}nimos locais em 2 e 4.
        \end{enumerate}

    \vspace{0.5cm}
    \item Calcule os pontos cr\'{\i}ticos das fun\c{c}\~oes abaixo:
        \begin{enumerate}
            %\hspace{0.3cm}
            \item $f(x) = 4+\ds\frac{x}{3}-\frac{x^2}{2}$
            %\hspace{0.3cm}
            \item $f(x) = 2x^3-3x^2-36x$
            %\hspace{0.3cm}
            \item $f(x) = \ds\frac{x-1}{x^2-x+1}$
            %\hspace{0.3cm}
            \item $f(x) = x^{\frac{4}{5}}{(x-4)}^2$
        \end{enumerate}

    \vspace{0.5cm}
    \item Encontre o m\'aximo absoluto e m\'{\i}nimo absoluto das fun\c{c}\~oes abaixo nos
	    intervalos dados:
        \begin{enumerate}
            %\hspace{0.3cm}
            \item $f(x) = 12+4x-x^2$ em $[0, 5]$
            %\hspace{0.3cm}
            \item $f(x) = {(x^2-1)}^3$ em $[-1, 2]$
            %\hspace{0.3cm}
            \item $f(x) = x+\ds\frac{1}{x}$ em $0,2 \le x \le 4$
        \end{enumerate}

    \vspace{0.5cm}
    \item O modelo para o pre\c{c}o m\'edio norte-americano para o a\c{c}\'ucar entre 1993
	    e 2003 \'e dado pela fun\c{c}\~ao \[P(t) = -0,00003237t^5 + 0,0009037t^4
		- 0,008956t^3 + 0,03629t^2 - 0,04458t + 0,4074\] onde $t$ \'e
		medido em anos desde agosto de 1993. Estime os instantes nos
		quais o a\c{c}\'ucar esteve mais barato e mais caro entre 1993 e
		2003.
\end{enumerate}

\end{document}
