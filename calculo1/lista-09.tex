\documentclass[a4paper,5pt]{amsbook}
%%%%%%%%%%%%%%%%%%%%%%%%%%%%%%%%%%%%%%%%%%%%%%%%%%%%%%%%%%%%%%%%%%%%%

\usepackage{booktabs}
\usepackage{graphicx}
\usepackage{multicol}
\usepackage{textcomp}
\usepackage{systeme}
\usepackage{amssymb}
\usepackage[]{amsmath}
\usepackage{subcaption}
\usepackage[inline]{enumitem}
\usepackage{gensymb}

%%%%%%%%%%%%%%%%%%%%%%%%%%%%%%%%%%%%%%%%%%%%%%%%%%%%%%%%%%%%%%

\newcommand{\sen}{\,\mbox{sen}\,}
\newcommand{\tg}{\,\mbox{tg}\,}
\newcommand{\cosec}{\,\mbox{cosec}\,}
\newcommand{\cotg}{\,\mbox{cotg}\,}
\newcommand{\tr}{\,\mbox{tr}\,}
\newcommand{\ds}{\displaystyle}
\newcommand{\ra}{\rightarrow}

%%%%%%%%%%%%%%%%%%%%%%%%%%%%%%%%%%%%%%%%%%%%%%%%%%%%%%%%%%%%%%%%%%%%%%%%

\setlength{\textwidth}{16cm} \setlength{\topmargin}{-1.7cm}
\setlength{\textheight}{25cm}
\setlength{\leftmargin}{1.2cm} \setlength{\rightmargin}{1.2cm}
\setlength{\oddsidemargin}{0cm}\setlength{\evensidemargin}{0cm}

%%%%%%%%%%%%%%%%%%%%%%%%%%%%%%%%%%%%%%%%%%%%%%%%%%%%%%%%%%%%%%%%%%%%%%%%

% \renewcommand{\baselinestretch}{1.6}
% \renewcommand{\thefootnote}{\fnsymbol{footnote}}
% \renewcommand{\theequation}{\thesection.\arabic{equation}}
% \setlength{\voffset}{-50pt}
% \numberwithin{equation}{chapter}

%%%%%%%%%%%%%%%%%%%%%%%%%%%%%%%%%%%%%%%%%%%%%%%%%%%%%%%%%%%%%%%%%%%%%%%

\begin{document}
\thispagestyle{empty}
\pagestyle{empty}
\begin{minipage}[h]{0.14\textwidth}
	\includegraphics[scale=0.24]{../ufgd.png}
\end{minipage}
\begin{minipage}[h]{\textwidth}
\begin{tabular}{c}
{{\bf UNIVERSIDADE FEDERAL DA GRANDE DOURADOS}}\\
{{\bf C\'alculo Diferencial e Integral --- Lista 9}}\\
{{\bf Prof.\ Adriano Barbosa}}\\
\end{tabular}
\vspace{-0.45cm}
%
\end{minipage}

%------------------------

\vspace{1cm}
%%%%%%%%%%%%%%%%%%%%%%%%%%%%%%%%   formulario  inicio  %%%%%%%%%%%%%%%%%%%%%%%%%%%%%%%%
\begin{enumerate}
    \vspace{0.5cm}
    \item Calcule os limites abaixo. Use a regra de L'Hospital quando
        poss\'{\i}vel. Se existir uma solu\c{c}\~ao mais elementar, d\^e
        prefer\^encia a essa solu\c{c}\~ao.

        \noindent{}
        \begin{enumerate*}
            \item $\ds\lim_{x\ra1} \frac{x^2-1}{x^2-x}$
            \hspace{0.3cm}
            \hspace{0.3cm}
            \item $\ds\lim_{x\ra1} \frac{x^3-2x^2+1}{x^3-1}$
            \hspace{0.3cm}
            \hspace{0.3cm}
            \item $\ds\lim_{x\ra\frac{\pi}{2}^+} \frac{\cos{x}}{1-\sen{x}}$
            \hspace{0.3cm}
            \hspace{0.3cm}
            \item $\ds\lim_{x\ra0} \frac{\sqrt{1+2x}-\sqrt{1-4x}}{x}$
            \hspace{0.3cm}
            \hspace{0.3cm}
            \item $\ds\lim_{x\ra1} \frac{x^8-1}{x^5-1}$
            \hspace{0.3cm}
            \hspace{0.3cm}
            \item $\ds\lim_{x\ra0} \frac{x3^x}{3^x-1}$
            \hspace{0.3cm}
            \hspace{0.3cm}
            \item $\ds\lim_{x\ra\infty} x\sen{\left(\frac{\pi}{x}\right)}$
            \hspace{0.3cm}
            \hspace{0.3cm}
            \item $\ds\lim_{x\ra0} {(1-2x)}^\frac{1}{x}$
            \hspace{0.3cm}
            \hspace{0.3cm}
            \item $\ds\lim_{x\ra1^+}x^{\frac{1}{1-x}}$
            \hspace{0.3cm}
            \hspace{0.3cm}
            \item $\ds\lim_{x\ra0^+} x^{\sqrt{x}}$
        \end{enumerate*}

    \vspace{0.5cm}
    \item Cada lado de um quadrado est\'a aumentando a uma taxa de 6 cm/s. A
        que taxa a \'area do quadrado est\'a aumentando quando sua \'area for 16
        cm$^2$?

    \vspace{0.5cm}
    \item Um tanque cil\'{\i}ndrico com raio de 5 m est\'a sendo enchido com \'agua a
        uma taxa de 3 m$^3$/min. Qu\~ao r\'apido a altura da \'agua est\'a
        aumentando?

    \vspace{0.5cm}
    \item Uma luz de rua \'e colocada no topo de um poste de 6 metros de
        altura. Um homem com 2 m de altura anda, afastando-se do poste com
        velocidade de 1,5 m/s ao longo de uma trajet\'oria reta. Com que
        velocidade se move a ponta de sua sombra quando ele est\'a a 10 m do
        poste?

    \vspace{0.5cm}
    \item Est\'a vazando \'agua de um tanque c\^onico invertido a uma taxa de
        10000cm/min. Ao mesmo tempo, \'agua est\'a sendo bombeada para dentro do
        tanque a uma taxa constante. O tanque tem 6 m de altura e o di\^ametro
        no topo \'e de 4 m. Se o n\'{\i}vel da \'agua estiver subindo a uma taxa de 20
        cm/min quando a altura da \'agua for 2 m, encontre a taxa segundo a
        qual a \'agua est\'a sendo bombeada dentro do tanque.

    \vspace{0.5cm}
    \item Suponha $y=\sqrt{2x+1}$, onde $x$ e $y$ s\~ao fun\c{c}\~oes de $t$.  Se
        $\ds\frac{dx}{dt}=3$, encontre $\ds\frac{dy}{dt}$ quando $x=4$.

    \vspace{0.5cm}
    \item Dado que $4x^2+9y^2=36$, onde $x$ e $y$ s\~ao fun\c{c}\~oes de $t$. Calcule
        $\ds\frac{dx}{dt}$ quando $x=2$, $y=\ds\frac{2}{3}\sqrt{5}$ e
        $\ds\frac{dy}{dt}=\frac{1}{3}$.

    \vspace{0.5cm}
    \item Uma part\'{\i}cula se move ao longo da curva $y=2\ds\sen{\left(\frac{\pi
                    x}{2}\right)}$. Quando a part\'{\i}cula passa pelo ponto
        $\left(\ds\frac{1}{3},1\right)$, sua coordenada $x$ cresce a uma taxa
        de $\sqrt{10}$ cm/s. Qu\~ao r\'apido a dist\^ancia da part\'{\i}cula \`a origem do
        sistema de coordenadas est\'a variando nesse momento?

    \vspace{0.5cm}
    \item Um homem come\c{c}a a andar para o norte a 1,2 m/s a partir de um ponto
        $P$. Cinco minutos depois uma mulher come\c{c}a a andar para o sul a
        1,6m/s de um ponto 200 m a leste de $P$. A que taxa as pessoas est\~ao
        de distanciando 15 minutos ap\'os a mulher come\c{c}ar a andar?

    \vspace{0.5cm}
    \item Dois lados de um tri\^angulo t\^em 4 m e 5 m, e o \^angulo entre eles
        est\'a crescendo a uma taxa de 0,06 rad/s. Encontre a taxa segundo a
        qual sua \'area est\'a crescendo quando o \^angulo entre os lados de
        comprimento fixo for $\ds\frac{\pi}{3}$.
\end{enumerate}

\end{document}
