\documentclass[a4paper,5pt]{amsbook}
%%%%%%%%%%%%%%%%%%%%%%%%%%%%%%%%%%%%%%%%%%%%%%%%%%%%%%%%%%%%%%%%%%%%%

\usepackage{booktabs}
\usepackage{graphicx}
\usepackage{multicol}
\usepackage{textcomp}
\usepackage{systeme}
\usepackage{amssymb}
\usepackage[]{amsmath}
\usepackage{subcaption}
\usepackage[inline]{enumitem}
\usepackage[utf8]{inputenc}

%%%%%%%%%%%%%%%%%%%%%%%%%%%%%%%%%%%%%%%%%%%%%%%%%%%%%%%%%%%%%%

\newcommand{\sen}{\,\mbox{sen}\,}
\newcommand{\tg}{\,\mbox{tg}\,}
\newcommand{\cosec}{\,\mbox{cosec}\,}
\newcommand{\cotg}{\,\mbox{cotg}\,}
\newcommand{\tr}{\,\mbox{tr}\,}
\newcommand{\ds}{\displaystyle}

%%%%%%%%%%%%%%%%%%%%%%%%%%%%%%%%%%%%%%%%%%%%%%%%%%%%%%%%%%%%%%%%%%%%%%%%

\setlength{\textwidth}{16cm} %\setlength{\topmargin}{-1.3cm}
\setlength{\textheight}{30cm}
\setlength{\leftmargin}{1.2cm} \setlength{\rightmargin}{1.2cm}
\setlength{\oddsidemargin}{0cm}\setlength{\evensidemargin}{0cm}

%%%%%%%%%%%%%%%%%%%%%%%%%%%%%%%%%%%%%%%%%%%%%%%%%%%%%%%%%%%%%%%%%%%%%%%%

% \renewcommand{\baselinestretch}{1.6}
% \renewcommand{\thefootnote}{\fnsymbol{footnote}}
% \renewcommand{\theequation}{\thesection.\arabic{equation}}
% \setlength{\voffset}{-50pt}
% \numberwithin{equation}{chapter}

%%%%%%%%%%%%%%%%%%%%%%%%%%%%%%%%%%%%%%%%%%%%%%%%%%%%%%%%%%%%%%%%%%%%%%%

\begin{document}
\thispagestyle{empty}
\pagestyle{empty}
\begin{minipage}[h]{0.14\textwidth}
	\includegraphics[scale=0.24]{../ufgd.png}
\end{minipage}
\begin{minipage}[h]{\textwidth}
\begin{tabular}{c}
{{\bf UNIVERSIDADE FEDERAL DA GRANDE DOURADOS}}\\
{{\bf Fundamentos de Matem\'{a}tica III --- Lista 9}}\\
{{\bf Prof.\ Adriano Barbosa}}\\
\end{tabular}
\vspace{-0.45cm}
%
\end{minipage}

%------------------------

\vspace{1cm}
%%%%%%%%%%%%%%%%%%%%%%%%%%%%%%%%   formulario  inicio  %%%%%%%%%%%%%%%%%%%%%%%%%%%%%%%%
\begin{enumerate}
	\vspace{0.5cm}
    \item Esboce os gr\'aficos das fun\c{c}\~oes polinomiais abaixo:
        \begin{enumerate}
                \vspace{0.3cm}
            \item $f(x)=3x-2$
                \vspace{0.3cm}
            \item $f(x)=-x-1$
                \vspace{0.3cm}
            \item $f(x)=-2x+3$
                \vspace{0.3cm}
            \item $f(x)=2(x-1)$
                \vspace{0.3cm}
            \item $f(x)=x^2-2$
                \vspace{0.3cm}
            \item $f(x)=-3x^2+1$
                \vspace{0.3cm}
            \item $\ds f(x)=-\frac{1}{2}{(x-1)}^2$
                \vspace{0.3cm}
            \item $f(x)=3{(x+2)}^2-1$
        \end{enumerate}

    \vspace{0.5cm}
    \item Escreva as fun\c{c}\~oes quadr\'aticas abaixo na forma can\^onica:
        \begin{enumerate}
            \vspace{0.3cm}
            \item $f(x)=x^2+x+1$
            \vspace{0.3cm}
            \item $f(x)=x^2-2x+3$
            \vspace{0.3cm}
            \item $f(x)=2x^2-x+2$
            \vspace{0.3cm}
            \item $f(x)=-x^2-x+1$
            \vspace{0.3cm}
            \item $\ds f(x)=\frac{1}{2}x^2+x+\frac{1}{3}$
        \end{enumerate}

    \vspace{0.5cm}
    \item Quantas s\~ao as ra\'{\i}zes reais da equa\c{c}\~ao $x^3-10x^2+5x-1=0$ no
        intervalo $(0, 3)$?

    \vspace{0.5cm}
    \item Determine $\alpha$ de modo que a equa\c{c}\~ao $x^3+x^2+5x+\alpha=0$ tenha
        ao menos uma raiz real no intervalo $(-2,0)$.
\end{enumerate}
\end{document}
