\documentclass[a4paper,5pt]{amsbook}
%%%%%%%%%%%%%%%%%%%%%%%%%%%%%%%%%%%%%%%%%%%%%%%%%%%%%%%%%%%%%%%%%%%%%

\usepackage{booktabs}
\usepackage{graphicx}
\usepackage{multicol}
\usepackage{textcomp}
\usepackage{systeme}
\usepackage{amssymb}
\usepackage[]{amsmath}
\usepackage{subcaption}
\usepackage[inline]{enumitem}
\usepackage[utf8]{inputenc}

%%%%%%%%%%%%%%%%%%%%%%%%%%%%%%%%%%%%%%%%%%%%%%%%%%%%%%%%%%%%%%

\newcommand{\sen}{\,\mbox{sen}\,}
\newcommand{\tg}{\,\mbox{tg}\,}
\newcommand{\cosec}{\,\mbox{cosec}\,}
\newcommand{\cotg}{\,\mbox{cotg}\,}
\newcommand{\tr}{\,\mbox{tr}\,}
\newcommand{\ds}{\displaystyle}

%%%%%%%%%%%%%%%%%%%%%%%%%%%%%%%%%%%%%%%%%%%%%%%%%%%%%%%%%%%%%%%%%%%%%%%%

\setlength{\textwidth}{16cm} %\setlength{\topmargin}{-1.3cm}
\setlength{\textheight}{30cm}
\setlength{\leftmargin}{1.2cm} \setlength{\rightmargin}{1.2cm}
\setlength{\oddsidemargin}{0cm}\setlength{\evensidemargin}{0cm}

%%%%%%%%%%%%%%%%%%%%%%%%%%%%%%%%%%%%%%%%%%%%%%%%%%%%%%%%%%%%%%%%%%%%%%%%

% \renewcommand{\baselinestretch}{1.6}
% \renewcommand{\thefootnote}{\fnsymbol{footnote}}
% \renewcommand{\theequation}{\thesection.\arabic{equation}}
% \setlength{\voffset}{-50pt}
% \numberwithin{equation}{chapter}

%%%%%%%%%%%%%%%%%%%%%%%%%%%%%%%%%%%%%%%%%%%%%%%%%%%%%%%%%%%%%%%%%%%%%%%

\begin{document}
\thispagestyle{empty}
\pagestyle{empty}
\begin{minipage}[h]{0.14\textwidth}
	\includegraphics[scale=0.24]{../ufgd.png}
\end{minipage}
\begin{minipage}[h]{\textwidth}
\begin{tabular}{c}
{{\bf UNIVERSIDADE FEDERAL DA GRANDE DOURADOS}}\\
    {{\bf Fundamentos de Matem\'{a}tica III --- Lista 10}}\\
{{\bf Prof.\ Adriano Barbosa}}\\
\end{tabular}
\vspace{-0.45cm}
%
\end{minipage}

%------------------------

\vspace{1cm}
%%%%%%%%%%%%%%%%%%%%%%%%%%%%%%%%   formulario  inicio  %%%%%%%%%%%%%%%%%%%%%%%%%%%%%%%%
\begin{enumerate}
    \vspace{0.5cm}
    \item Liste as poss\'{\i}veis ra\'{\i}zes racionais da equa\c{c}\~ao
        $5x^7+4x^5+2x^3+x+12=0$?

	\vspace{0.5cm}
    \item Verifique se a equa\c{c}\~ao $x^3+x^2-4x+6=0$ possui ra\'{\i}zes racionais e
        determine todas as suas ra\'{\i}zes.

    \vspace{0.5cm}
    \item Resolva a equa\c{c}\~ao $5x^3-37x^2+90x-72=0$, sabendo que admite ra\'{\i}zes
        inteiras positivas.

    \vspace{0.5cm}
    \item As equa\c{c}\~oes $x^4+2x^3+3x^2+4x+2=0$ e $(x-a)(x-b)(x-c)(x-d)=0$, onde
        $a$, $b$, $c$ e $d$ s\~ao n\'umeros racionais,  podem ter ra\'{\i}zes em comum?

    \vspace{0.5cm}
    \item Dado $x^2+ax+b$ um trin\^omio m\^onico (coeficiente l\'{\i}der igual a $1$) do
        segundo grau com $a$ e $b$ n\'umeros racionais e que possui uma raiz
        irracional $\alpha$. Mostre que este \'e o \'unico trin\^omio m\^onico do
        segundo grau em que $\alpha$ \'e raiz.

    \vspace{0.5cm}
    \item Use o m\'etodo da bisse\c{c}\~ao para encontrar as ra\'{\i}zes da equa\c{c}\~ao
        $x^3-7x^2+14x-6=0$ com precis\~ao de $10^{-2}$ em cada intervalo
        abaixo:
        \vspace{0.3cm}

        \begin{enumerate*}
            \item $(0,1)$
            \hspace{1cm}
            \hspace{1cm}
            \item $\left(1, \ds\frac{16}{5}\right)$
            \hspace{1cm}
            \hspace{1cm}
            \item $\left(\ds\frac{16}{5}, 4\right)$
        \end{enumerate*}
\end{enumerate}
\end{document}
