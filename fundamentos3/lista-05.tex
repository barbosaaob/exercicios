\documentclass[a4paper,5pt]{amsbook}
%%%%%%%%%%%%%%%%%%%%%%%%%%%%%%%%%%%%%%%%%%%%%%%%%%%%%%%%%%%%%%%%%%%%%

\usepackage{booktabs}
\usepackage{graphicx}
\usepackage{multicol}
\usepackage{textcomp}
\usepackage{systeme}
\usepackage{amssymb}
\usepackage[]{amsmath}
\usepackage{subcaption}
\usepackage[inline]{enumitem}
\usepackage[utf8]{inputenc}

%%%%%%%%%%%%%%%%%%%%%%%%%%%%%%%%%%%%%%%%%%%%%%%%%%%%%%%%%%%%%%

\newcommand{\sen}{\,\mbox{sen}\,}
\newcommand{\tg}{\,\mbox{tg}\,}
\newcommand{\cosec}{\,\mbox{cosec}\,}
\newcommand{\cotg}{\,\mbox{cotg}\,}
\newcommand{\tr}{\,\mbox{tr}\,}
\newcommand{\ds}{\displaystyle}

%%%%%%%%%%%%%%%%%%%%%%%%%%%%%%%%%%%%%%%%%%%%%%%%%%%%%%%%%%%%%%%%%%%%%%%%

\setlength{\textwidth}{16cm} %\setlength{\topmargin}{-1.3cm}
\setlength{\textheight}{30cm}
\setlength{\leftmargin}{1.2cm} \setlength{\rightmargin}{1.2cm}
\setlength{\oddsidemargin}{0cm}\setlength{\evensidemargin}{0cm}

%%%%%%%%%%%%%%%%%%%%%%%%%%%%%%%%%%%%%%%%%%%%%%%%%%%%%%%%%%%%%%%%%%%%%%%%

% \renewcommand{\baselinestretch}{1.6}
% \renewcommand{\thefootnote}{\fnsymbol{footnote}}
% \renewcommand{\theequation}{\thesection.\arabic{equation}}
% \setlength{\voffset}{-50pt}
% \numberwithin{equation}{chapter}

%%%%%%%%%%%%%%%%%%%%%%%%%%%%%%%%%%%%%%%%%%%%%%%%%%%%%%%%%%%%%%%%%%%%%%%

\begin{document}
\thispagestyle{empty}
\pagestyle{empty}
\begin{minipage}[h]{0.14\textwidth}
	\includegraphics[scale=0.24]{../ufgd.png}
\end{minipage}
\begin{minipage}[h]{\textwidth}
\begin{tabular}{c}
{{\bf UNIVERSIDADE FEDERAL DA GRANDE DOURADOS}}\\
{{\bf Fundamentos de Matem\'{a}tica III --- Lista 5}}\\
{{\bf Prof.\ Adriano Barbosa}}\\
\end{tabular}
\vspace{-0.45cm}
%
\end{minipage}

%------------------------

\vspace{1cm}
%%%%%%%%%%%%%%%%%%%%%%%%%%%%%%%%   formulario  inicio  %%%%%%%%%%%%%%%%%%%%%%%%%%%%%%%%
\begin{enumerate}
	\vspace{0.5cm}
	\item Dada a fun\c{c}\~ao polinomial $f(x) = x^{15} + x^{14} + \cdots + x^2 + x +
		1$, calcule $f(-1)$, $f(0)$ e $f(1)$.

	\vspace{0.5cm}
	\item Determine os n\'umeros reais $a$, $b$ e $c$ de modo que $f = (a-2)x^3 +
		(b+2)x + (3-c)$ seja o polin\^omio nulo.

	\vspace{0.5cm}
	\item Determine $a$, $b$ e $c$ de modo que
		\[\ds\frac{ax^2+bx-5}{3x^2+7x+c} = 3\]
		para todo $x\in\mathbb{R}$.

	\vspace{0.5cm}
	\item Dados $f(x) = 2+3x-4x^2$, $g(x) = 7+x^2$ e $h(x) = 2x-3x^2+x^3$. Calcule:

		\vspace{0.3cm}
		\begin{enumerate*}
			\item $(f+g)(x)$
			\hspace{0.3cm}
			\hspace{0.3cm}
			\item $(f-h)(x)$
			\hspace{0.3cm}
			\hspace{0.3cm}
			\item $(fg)(x)$
			\hspace{0.3cm}
			\hspace{0.3cm}
			\item $(gh)(x)$
		\end{enumerate*}

	\vspace{0.5cm}
	\item Calcule os valores de $\alpha\in\mathbb{R}$ tais que $f=g^2$, onde $f
		= x^4+2\alpha x^3-4\alpha x+4$ e $g = x^2+2x+2$.

	\vspace{0.5cm}
	\item Detemrine o polin\^omio de grau dois tal que $f(0)=1$, $f(1)=4$ e
		$f(-1)=0$.

	\vspace{0.5cm}
	\item Determine $f(x)$ de modo que $\partial f=2$, $f(1)=0$ e
		$f(x)=f(x-1),\ \forall x\in\mathbb{R}$.
\end{enumerate}

\end{document}
