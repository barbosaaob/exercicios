\documentclass[a4paper,5pt]{amsbook}
%%%%%%%%%%%%%%%%%%%%%%%%%%%%%%%%%%%%%%%%%%%%%%%%%%%%%%%%%%%%%%%%%%%%%

\usepackage{booktabs}
\usepackage{graphicx}
\usepackage{multicol}
\usepackage{textcomp}
\usepackage{systeme}
\usepackage{amssymb}
\usepackage[]{amsmath}
\usepackage{subcaption}
\usepackage[inline]{enumitem}
\usepackage[utf8]{inputenc}

%%%%%%%%%%%%%%%%%%%%%%%%%%%%%%%%%%%%%%%%%%%%%%%%%%%%%%%%%%%%%%

\newcommand{\sen}{\,\mbox{sen}\,}
\newcommand{\tg}{\,\mbox{tg}\,}
\newcommand{\cosec}{\,\mbox{cosec}\,}
\newcommand{\cotg}{\,\mbox{cotg}\,}
\newcommand{\tr}{\,\mbox{tr}\,}
\newcommand{\ds}{\displaystyle}

%%%%%%%%%%%%%%%%%%%%%%%%%%%%%%%%%%%%%%%%%%%%%%%%%%%%%%%%%%%%%%%%%%%%%%%%

\setlength{\textwidth}{16cm} %\setlength{\topmargin}{-1.3cm}
\setlength{\textheight}{30cm}
\setlength{\leftmargin}{1.2cm} \setlength{\rightmargin}{1.2cm}
\setlength{\oddsidemargin}{0cm}\setlength{\evensidemargin}{0cm}

%%%%%%%%%%%%%%%%%%%%%%%%%%%%%%%%%%%%%%%%%%%%%%%%%%%%%%%%%%%%%%%%%%%%%%%%

% \renewcommand{\baselinestretch}{1.6}
% \renewcommand{\thefootnote}{\fnsymbol{footnote}}
% \renewcommand{\theequation}{\thesection.\arabic{equation}}
% \setlength{\voffset}{-50pt}
% \numberwithin{equation}{chapter}

%%%%%%%%%%%%%%%%%%%%%%%%%%%%%%%%%%%%%%%%%%%%%%%%%%%%%%%%%%%%%%%%%%%%%%%

\begin{document}
\thispagestyle{empty}
\pagestyle{empty}
\begin{minipage}[h]{0.14\textwidth}
	\includegraphics[scale=0.24]{../ufgd.png}
\end{minipage}
\begin{minipage}[h]{\textwidth}
\begin{tabular}{c}
{{\bf UNIVERSIDADE FEDERAL DA GRANDE DOURADOS}}\\
{{\bf Fundamentos de Matem\'{a}tica III --- Lista 3}}\\
{{\bf Prof.\ Adriano Barbosa}}\\
\end{tabular}
\vspace{-0.45cm}
%
\end{minipage}

%------------------------

\vspace{1cm}
%%%%%%%%%%%%%%%%%%%%%%%%%%%%%%%%   formulario  inicio  %%%%%%%%%%%%%%%%%%%%%%%%%%%%%%%%
\begin{enumerate}
	\vspace{0.5cm}
	\item Seja $z=1+i$, determine o módulo e o argumento de $z^4$.

	\vspace{0.5cm}
	\item Calcule

		\begin{enumerate*}
			\item $\displaystyle{\left(-\frac{\sqrt{3}}{2}-\frac{i}{2}\right)}^{100}$
			\hspace{0.3cm}
			\hspace{0.3cm}
			\item ${(-1+i)}^{6}$
			\hspace{0.3cm}
			\hspace{0.3cm}
			\item ${(1+\sqrt{3}i)}^{-5}$
			\hspace{0.3cm}
			\hspace{0.3cm}

			\item $\displaystyle\frac{i}{{\left(-\frac{1}{2}-\frac{\sqrt{3}}{2}i\right)}^6}$
		\end{enumerate*}

	\vspace{0.5cm}
	\item Determine, se possível, o menor valor de $n\in\mathbb{N}$ tal que ${(\sqrt{3}+i)}^n$ é:

		\begin{enumerate*}
			\item real positivo
			\hspace{0.5cm}
			\hspace{0.5cm}
			\item real negativo
			\hspace{0.5cm}
			\hspace{0.5cm}
			\item imaginário puro
		\end{enumerate*}

	\vspace{0.5cm}
	\item
		\begin{enumerate}
			\item Encontre $z$ tal que $iz+2\bar{z} + 1 - i = 0$.
			\item Calcule o módulo e o argumento de $z$.
			\item Calcule $z^{1004}$.
		\end{enumerate}

	\vspace{0.5cm}
	\item Calcule

		\begin{enumerate*}
			\item $\sqrt{-7+24i}$
			\hspace{0.3cm}
			\hspace{0.3cm}
			\item $\sqrt[3]{-11-2i}$
			\hspace{0.3cm}
			\hspace{0.3cm}
			\item $\sqrt{5+12i}$
			\hspace{0.3cm}
			\hspace{0.3cm}
			\item $\displaystyle\frac{1}{\sqrt{-4i}}$
		\end{enumerate*}

	\vspace{0.5cm}
	\item Sabendo que $\displaystyle\frac{\sqrt{2}}{2} + \frac{\sqrt{2}}{2}i$ é uma das raízes quartas de $z$, calcule todas as raízes quartas de $z$.

	\vspace{0.5cm}
	\item Determine graficamente as raízes quartas de $81$.

	\vspace{0.5cm}
	\item Mostre que a soma das raízes de índice $2n$ de um número complexo qualquer é zero.
\end{enumerate}

\end{document}
