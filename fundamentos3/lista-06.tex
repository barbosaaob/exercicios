\documentclass[a4paper,5pt]{amsbook}
%%%%%%%%%%%%%%%%%%%%%%%%%%%%%%%%%%%%%%%%%%%%%%%%%%%%%%%%%%%%%%%%%%%%%

\usepackage{booktabs}
\usepackage{graphicx}
\usepackage{multicol}
\usepackage{textcomp}
\usepackage{systeme}
\usepackage{amssymb}
\usepackage[]{amsmath}
\usepackage{subcaption}
\usepackage[inline]{enumitem}
\usepackage[utf8]{inputenc}

%%%%%%%%%%%%%%%%%%%%%%%%%%%%%%%%%%%%%%%%%%%%%%%%%%%%%%%%%%%%%%

\newcommand{\sen}{\,\mbox{sen}\,}
\newcommand{\tg}{\,\mbox{tg}\,}
\newcommand{\cosec}{\,\mbox{cosec}\,}
\newcommand{\cotg}{\,\mbox{cotg}\,}
\newcommand{\tr}{\,\mbox{tr}\,}
\newcommand{\ds}{\displaystyle}

%%%%%%%%%%%%%%%%%%%%%%%%%%%%%%%%%%%%%%%%%%%%%%%%%%%%%%%%%%%%%%%%%%%%%%%%

\setlength{\textwidth}{16cm} %\setlength{\topmargin}{-1.3cm}
\setlength{\textheight}{30cm}
\setlength{\leftmargin}{1.2cm} \setlength{\rightmargin}{1.2cm}
\setlength{\oddsidemargin}{0cm}\setlength{\evensidemargin}{0cm}

%%%%%%%%%%%%%%%%%%%%%%%%%%%%%%%%%%%%%%%%%%%%%%%%%%%%%%%%%%%%%%%%%%%%%%%%

% \renewcommand{\baselinestretch}{1.6}
% \renewcommand{\thefootnote}{\fnsymbol{footnote}}
% \renewcommand{\theequation}{\thesection.\arabic{equation}}
% \setlength{\voffset}{-50pt}
% \numberwithin{equation}{chapter}

%%%%%%%%%%%%%%%%%%%%%%%%%%%%%%%%%%%%%%%%%%%%%%%%%%%%%%%%%%%%%%%%%%%%%%%

\begin{document}
\thispagestyle{empty}
\pagestyle{empty}
\begin{minipage}[h]{0.14\textwidth}
	\includegraphics[scale=0.24]{../ufgd.png}
\end{minipage}
\begin{minipage}[h]{\textwidth}
\begin{tabular}{c}
{{\bf UNIVERSIDADE FEDERAL DA GRANDE DOURADOS}}\\
{{\bf Fundamentos de Matem\'{a}tica III --- Lista 6}}\\
{{\bf Prof.\ Adriano Barbosa}}\\
\end{tabular}
\vspace{-0.45cm}
%
\end{minipage}

%------------------------

\vspace{1cm}
%%%%%%%%%%%%%%%%%%%%%%%%%%%%%%%%   formulario  inicio  %%%%%%%%%%%%%%%%%%%%%%%%%%%%%%%%
\begin{enumerate}
	\vspace{0.5cm}
	\item Numa divis\~ao de polin\^omios onde o divisor tem grau 4, o quociente tem
		grau 2 e o resto tem grau 1, qual \'e o grau do dividendo? E se o grau do
		resto fosse 2?

	\vspace{0.5cm}
	\item Efetue a divis\~ao de $f$ por $g$ usando o m\'etodo dos coeficientes a
		determinar:
		\begin{enumerate}
			\item $f=3x^5-x^4+2x^3+4x-3$ e $g=x^3-2x+1$
			\item $f=2x^5-3x+12$ e $g=x^2+1$
			\item $f=x^4-2x+13$ e $g=x^2+x+1$
		\end{enumerate}

	\vspace{0.5cm}
	\item Determine $p\in\mathbb{R}$ e $q\in\mathbb{R}$ de modo que $x^4+1$
		seja divis\'{\i}vel por $x^2+px+q$.

	\vspace{0.5cm}
	\item Aplicando o m\'etodo da chave, efetue a divis\~ao de $f$ por $g$:
		\begin{enumerate}
			\item $f=x^2+5x+1$ e $g=2x^2+4x-3$
			\item $f=5x+1$ e $g=x^3+5$
			\item $f=x^3+x^2+x+1$ e $g=2x^2+3$
		\end{enumerate}

	\vspace{0.5cm}
	\item Calcule o quociente e o resto da divis\~ao de $x^n-a^n$ e $x^n+a^n$ por
		$x-a$.

	\vspace{0.5cm}
	\item Determine o quociente e o resto da divis\~ao de $f$ por $g$:
		\begin{enumerate}
			\item $f=x^4-81$ e $g=x+3$
			\item $f=x^5-32$ e $g=x-2$
			\item $f=x^6-1$ e $g=x+1$
		\end{enumerate}

	\vspace{0.5cm}
	\item Determine $a$ de modo que a divis\~ao de $f=x^4-2ax^3 + (a+2)x^2 + 3a
		+1$ por $g=x-2$ tenha resto igual a 7.

	\vspace{0.5cm}
	\item Obtenha um polin\^omio $f$ com coeficiente l\'{\i}der igual a 1 de grau 2
		tal que $f$ \'e divis\'{\i}vel por $x-1$ e os restos das divis\~oes de $f$ por
		$x-2$ e $x-3$ s\~ao iguais.

	\vspace{0.5cm}
	\item Mostre que $f=x^4+2x^3-x-2$ \'e divis\'{\i}vel por $x+2$ e $x+1$. Conclua
		que $f$ tamb\'em \'e divis\'{\i}vel por $x^2+3x+2$.

	\vspace{0.5cm}
	\item Se as divis\~oes de $f$ por $x-1$, $x-2$ e $x-3$ s\~ao exatas, o que
		podemos afirmar sobre o grau de $f$?
\end{enumerate}

\end{document}
