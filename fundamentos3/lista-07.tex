\documentclass[a4paper,5pt]{amsbook}
%%%%%%%%%%%%%%%%%%%%%%%%%%%%%%%%%%%%%%%%%%%%%%%%%%%%%%%%%%%%%%%%%%%%%

\usepackage{booktabs}
\usepackage{graphicx}
\usepackage{multicol}
\usepackage{textcomp}
\usepackage{systeme}
\usepackage{amssymb}
\usepackage[]{amsmath}
\usepackage{subcaption}
\usepackage[inline]{enumitem}
\usepackage[utf8]{inputenc}

%%%%%%%%%%%%%%%%%%%%%%%%%%%%%%%%%%%%%%%%%%%%%%%%%%%%%%%%%%%%%%

\newcommand{\sen}{\,\mbox{sen}\,}
\newcommand{\tg}{\,\mbox{tg}\,}
\newcommand{\cosec}{\,\mbox{cosec}\,}
\newcommand{\cotg}{\,\mbox{cotg}\,}
\newcommand{\tr}{\,\mbox{tr}\,}
\newcommand{\ds}{\displaystyle}

%%%%%%%%%%%%%%%%%%%%%%%%%%%%%%%%%%%%%%%%%%%%%%%%%%%%%%%%%%%%%%%%%%%%%%%%

\setlength{\textwidth}{16cm} %\setlength{\topmargin}{-1.3cm}
\setlength{\textheight}{30cm}
\setlength{\leftmargin}{1.2cm} \setlength{\rightmargin}{1.2cm}
\setlength{\oddsidemargin}{0cm}\setlength{\evensidemargin}{0cm}

%%%%%%%%%%%%%%%%%%%%%%%%%%%%%%%%%%%%%%%%%%%%%%%%%%%%%%%%%%%%%%%%%%%%%%%%

% \renewcommand{\baselinestretch}{1.6}
% \renewcommand{\thefootnote}{\fnsymbol{footnote}}
% \renewcommand{\theequation}{\thesection.\arabic{equation}}
% \setlength{\voffset}{-50pt}
% \numberwithin{equation}{chapter}

%%%%%%%%%%%%%%%%%%%%%%%%%%%%%%%%%%%%%%%%%%%%%%%%%%%%%%%%%%%%%%%%%%%%%%%

\begin{document}
\thispagestyle{empty}
\pagestyle{empty}
\begin{minipage}[h]{0.14\textwidth}
	\includegraphics[scale=0.24]{../ufgd.png}
\end{minipage}
\begin{minipage}[h]{\textwidth}
\begin{tabular}{c}
{{\bf UNIVERSIDADE FEDERAL DA GRANDE DOURADOS}}\\
{{\bf Fundamentos de Matem\'{a}tica III --- Lista 7}}\\
{{\bf Prof.\ Adriano Barbosa}}\\
\end{tabular}
\vspace{-0.45cm}
%
\end{minipage}

%------------------------

\vspace{1cm}
%%%%%%%%%%%%%%%%%%%%%%%%%%%%%%%%   formulario  inicio  %%%%%%%%%%%%%%%%%%%%%%%%%%%%%%%%
\begin{enumerate}
	\vspace{0.5cm}
    \item Determine $m$ de modo que $-2$ seja raiz da equa\c{c}\~ao
        $x^3+(m+2)x^2+(1+m)x-2=0$.

    \vspace{0.5cm}
    \item Resolva as equa\c{c}\~oes polinomiais:
        \begin{enumerate}
            \vspace{0.3cm}
            \item $(x+1)(x^2-x+1) = (x-1)^3$
            \vspace{0.3cm}
            \item $(x+2)(x+3)+(x-2)(1-x) = 4(1+2x)$
        \end{enumerate}

    \vspace{0.5cm}
    \item Resolva a equa\c{c}\~ao $6x^3+7x^2-14x-15 = 0$ sabendo que uma das ra\'{\i}zes
        \'e $-1$.

    \vspace{0.5cm}
    \item Determine o polin\^omio $p(x)$ de grau 3  cujas ra\'{\i}zes s\~ao 0, 1 e 2
        sabendo que $p\left(\frac{1}{2}\right) = -\frac{3}{2}$.

    \vspace{0.5cm}
    \item Determine todas as ra\'{\i}zes e suas multiplicidades nas equa\c{c}\~oes abaixo:
        \begin{enumerate}
            \vspace{0.3cm}
            \item $3(x+4)(x^2+1) = 0$
            \vspace{0.3cm}
            \item $4{(x-10)}^5(2x-3) = 4{(x-10)}^5(x-1)$
            \vspace{0.3cm}
            \item ${(x^2+x+1)}^3{(7x-14i)}^5=0$
        \end{enumerate}

    \vspace{0.5cm}
    \item Determine uma equa\c{c}\~ao polinomial cuja ra\'{\i}zes sejam 1, $i$ e $-i$
        com multiplicidade 1, 2 e 2.

    \vspace{0.5cm}
    \item Calcule a soma e o produto das ra\'{\i}zes das equa\c{c}\~oes abaixo:
        \begin{enumerate}
            \vspace{0.3cm}
            \item $x^3-2x^2+3x-5=0$
            \vspace{0.3cm}
            \item $2x^3+4x^2+7x+10i=0$
            \vspace{0.3cm}
            \item $x^2-7x+2=0$
        \end{enumerate}

    \vspace{0.5cm}
    \item Resolva a equa\c{c}\~ao $x^3-4x^2+x+6=0$ sabendo que uma rais \'e igual a
        soma das outras duas.

\end{enumerate}
\end{document}
