\documentclass[a4paper,5pt]{amsbook}
%%%%%%%%%%%%%%%%%%%%%%%%%%%%%%%%%%%%%%%%%%%%%%%%%%%%%%%%%%%%%%%%%%%%%

\usepackage{booktabs}
\usepackage{graphicx}
\usepackage{multicol}
\usepackage{textcomp}
\usepackage{systeme}
\usepackage{amssymb}
\usepackage[]{amsmath}
\usepackage{subcaption}
\usepackage[inline]{enumitem}

%%%%%%%%%%%%%%%%%%%%%%%%%%%%%%%%%%%%%%%%%%%%%%%%%%%%%%%%%%%%%%

\newcommand{\sen}{\,\mbox{sen}\,}
\newcommand{\tg}{\,\mbox{tg}\,}
\newcommand{\cosec}{\,\mbox{cosec}\,}
\newcommand{\cotg}{\,\mbox{cotg}\,}
\newcommand{\tr}{\,\mbox{tr}\,}
\newcommand{\ds}{\displaystyle}

%%%%%%%%%%%%%%%%%%%%%%%%%%%%%%%%%%%%%%%%%%%%%%%%%%%%%%%%%%%%%%%%%%%%%%%%

\setlength{\textwidth}{16cm} %\setlength{\topmargin}{-1.3cm}
\setlength{\textheight}{30cm}
\setlength{\leftmargin}{1.2cm} \setlength{\rightmargin}{1.2cm}
\setlength{\oddsidemargin}{0cm}\setlength{\evensidemargin}{0cm}

%%%%%%%%%%%%%%%%%%%%%%%%%%%%%%%%%%%%%%%%%%%%%%%%%%%%%%%%%%%%%%%%%%%%%%%%

% \renewcommand{\baselinestretch}{1.6}
% \renewcommand{\thefootnote}{\fnsymbol{footnote}}
% \renewcommand{\theequation}{\thesection.\arabic{equation}}
% \setlength{\voffset}{-50pt}
% \numberwithin{equation}{chapter}

%%%%%%%%%%%%%%%%%%%%%%%%%%%%%%%%%%%%%%%%%%%%%%%%%%%%%%%%%%%%%%%%%%%%%%%

\begin{document}
\thispagestyle{empty}
\pagestyle{empty}
\begin{minipage}[h]{0.14\textwidth}
	\includegraphics[scale=0.24]{../ufgd.png}
\end{minipage}
\begin{minipage}[h]{\textwidth}
\begin{tabular}{c}
{{\bf UNIVERSIDADE FEDERAL DA GRANDE DOURADOS}}\\
{{\bf Fundamentos de Matem\'{a}tica III --- Lista 2}}\\
{{\bf Prof.\ Adriano Barbosa}}\\
\end{tabular}
\vspace{-0.45cm}
%
\end{minipage}

%------------------------

\vspace{1cm}
%%%%%%%%%%%%%%%%%%%%%%%%%%%%%%%%   formulario  inicio  %%%%%%%%%%%%%%%%%%%%%%%%%%%%%%%%
\begin{enumerate}
	\vspace{0.5cm}
	\item Efetue as divis\~oes e escreva os n\'umeros na forma alg\'ebrica:

		\begin{enumerate*}
			\item $\ds\frac{1}{2+i}$
			\hspace{0.5cm}
			\hspace{0.5cm}
			\item $\ds\frac{3+4i}{3-i}$
			\hspace{0.5cm}
			\hspace{0.5cm}
			\item $\ds\frac{i^3-i^2+i^{17}-i^{35}}{i^{16}-i^{13}+i^{30}}$
			\hspace{0.5cm}
			\hspace{0.5cm}
			\item $\ds\frac{1+i}{{(1-i)}^2}$
		\end{enumerate*}

	\vspace{0.5cm}
	\item Determine $x \in \mathbb{R}$ tal que:
		\begin{enumerate}
			\vspace{0.3cm}
			\item $\ds\frac{2-xi}{1+2xi}$ seja imagin\'ario puro
			\vspace{0.3cm}
			\item $\ds\frac{1+2i}{2+xi}$ seja real
		\end{enumerate}

	\vspace{0.5cm}
	\item Mostre que $\overline{(z^n)} = {(\overline{z})}^n$ para todo $n$ natural.

	\vspace{0.5cm}
	\item Determine $z \in \mathbb{C}$ tal que:

		\begin{enumerate*}
			\item $z^3=\overline{z}$
			\hspace{0.5cm}
			\hspace{0.5cm}
			\item $z^2=i$
			\hspace{0.5cm}
			\hspace{0.5cm}
			\item $z^2 = 1+\sqrt{3}i$
		\end{enumerate*}

	\vspace{0.5cm}
	\item Calcule $z\in\mathbb{C}$ tal que $z\ \overline{z} +
		(z-\overline{z})=13+6i$.

	\vspace{0.5cm}
	\item Use as propriedades de m\'odulo para calcular:
		\begin{enumerate}
			\vspace{0.3cm}
			\item $|z_1z_2|$, onde $z_1=1-i$ e $z_2=2+2i$
			\vspace{0.3cm}
			\item $\ds\left|\frac{z_1}{z_2}\right|$, onde $z_1=3+3i$ e
				$z_2=1+2i$
			\vspace{0.3cm}
			\item $|z^6|$, onde $z=1+\sqrt{3}i$
		\end{enumerate}

	\vspace{0.5cm}
	\item Escreva os n\'umeros abaixo na forma trigonom\'etrica:

		\begin{enumerate*}
			\item $3+3i$
			\hspace{0.5cm}
			\hspace{0.5cm}
			\item $i^3$
			\hspace{0.5cm}
			\hspace{0.5cm}
			\item $2i(1-i)$
			\hspace{0.5cm}
			\hspace{0.5cm}
			\item $5-5\sqrt{3}i$
			\hspace{0.5cm}
			\hspace{0.5cm}
			\item $2i$
		\end{enumerate*}

	\vspace{0.5cm}
	\item Represente no plano os n\'umeros abaixo e indique graficamente o m\'odulo
		e o argumento principal de cada um deles

		\begin{enumerate*}
			\item $2+5i$
			\hspace{0.5cm}
			\hspace{0.5cm}
			\item $-2-3i$
			\hspace{0.5cm}
			\hspace{0.5cm}
			\item $1-4i$
		\end{enumerate*}

	\vspace{0.5cm}
	\item Represente geometricamente no plano os conjuntos abaixo:
		\begin{enumerate}
			\item $A=\{z\in\mathbb{C}\ |\ Re(z)=0\}$
			\item $B=\{z\in\mathbb{C}\ |\ Im(z)=0\}$
			\item $C=\{z\in\mathbb{C}\ |\ |z|=1\}$
			\item $D=\{z\in\mathbb{C}\ |\ Re(z)\ge 1\ e\ Im(z)\ge 2\}$
		\end{enumerate}

	\vspace{0.5cm}
	\item Represente geometricamente no plano o conjunto dos n\'umeros complexos
		$z$ tais que

		\noindent{}$|z-(1+i)| \le 1$.
\end{enumerate}

\end{document}
